\chapter{Podsumowanie pracy i wnioski końcowe}\label{cha:pierwszyDokument}

Jakość wyników otrzymywanych za pomocą algorytmów genetycznych zależy
od wielkości populacji, czasu przeznaczonego na poszukiwanie rozwiązania, doboru
sposobu selekcji, zastosowanych operatorów krzyżowania i mutacji oraz prawdopodobieństwa, z jakim te operacje są przeprowadzane. W algorytmach genetycznych
stosowane są także inne operatory krzyżowania i mutacji niż przytoczone powyżej. \\
\par
Im wiekszy rozmiar problemu tym trudniej jest o dobre rozwiazanie bo ten wykres n kształtuje jest eksponencjalnie.\\
\par
 Ponadto,
często wymaga się, aby indeksy osobników biorących udział w mutacji były parami
różne, jakkolwiek to założenie nie jest zwykle uwzględnianie w analizach teoretycznych
i nie ma także zastosowania w badaniach przedstawionych w rozprawie. Pominięcie
warunku (2.2) powoduje, że położenie mutanta 
jest niezależne od położenia jego
rodzica \cite{diff2}
. Uwzględnienie tego wymogu wprowadziłoby jedynie słabą zależność,
gdyż dla liczniejszych populacji i tak jest on zazwyczaj spełniony
\par
Przedwczesna zbieżność algorytmu polega na tym, że najlepsze ale jeszcze nie optymalne chromosomy
dominują w populacji. 
\par
Moze ze istnieje jeszcze wiele metod do zbadania, wiele modyfikacji np. przekształcenia liniowe.
\par
Wprowadzenie dynamicznej cos tam parametrów moze uniknac trudnego etapu strojenia algorytmu, bo jakby robi sie to samo. W literaturze [87, 149] podane są
optymalne wartości współczynników mutacji czy krzyżowania, należy jednak pamiętać, że najczęściej zbiorem testowym jest standardowy zestaw funkcji. W przypadku bardziej złożonych problemów najczęściej wymagane jest ustalenie innych
wartości parametrów.
\par
mutacja jest duzo wazniejsza od krzyzowania, co jest odwrtone w stosunku do algorytmow genetycznych