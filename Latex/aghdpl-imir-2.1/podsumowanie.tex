\chapter{Podsumowanie pracy i wnioski końcowe}\label{cha:pierwszyDokument}

Jakość wyników otrzymywanych za pomocą algorytmów genetycznych zależy
od wielkości populacji, czasu przeznaczonego na poszukiwanie rozwiązania, doboru
sposobu selekcji, zastosowanych operatorów krzyżowania i mutacji oraz prawdopodobieństwa, z jakim te operacje są przeprowadzane. W algorytmach genetycznych
stosowane są także inne operatory krzyżowania i mutacji niż przytoczone powyżej. \\
\par
Im wiekszy rozmiar problemu tym trudniej jest o dobre rozwiazanie bo ten wykres n kształtuje jest eksponencjalnie.\\
\par
 Ponadto,
często wymaga się, aby indeksy osobników biorących udział w mutacji były parami
różne, jakkolwiek to założenie nie jest zwykle uwzględnianie w analizach teoretycznych
i nie ma także zastosowania w badaniach przedstawionych w rozprawie. Pominięcie
warunku (2.2) powoduje, że położenie mutanta 
jest niezależne od położenia jego
rodzica \cite{diff2}
. Uwzględnienie tego wymogu wprowadziłoby jedynie słabą zależność,
gdyż dla liczniejszych populacji i tak jest on zazwyczaj spełniony