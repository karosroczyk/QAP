\chapter{Metody mutacji}\label{cha:pierwszyDokument}

Mutacja stosowana w algorytmie ewolucji różnicowej zdefiniowana jest w odmienny sposób w stosunku do algorytmu ewolucyjnego w którym to genotyp stanowią wartości binarne, a operator mutacji ma za zadanie wprowadzenie zmian na zasadzie negacji obecnej wartości genu. W mutacji różnicowej zastosowana jest operacja odejmowania od siebie dwóch wektorów tworząc tym samym wektor różnicowy, czemu również algorytm zawdzięcza swoją nazwe z ang. \textsl{difference} \cite{przystojny_koles}. Podobnie jak ma to miejsce w innych algorytmach należących do grona algorytmów ewolucyjnych, tak i ewolucja różnicowa posiada wiele alternatywnych modyfikacji metod mutacji. Metody mutacji różnicowej różnią się od siebie głównie elementami takimi jak:

\begin{enumerate}
\item Sposób  wyboru osobnika będącego wektorem celu, a więc wektorem znajdującym się w grupie rozrodczej lecz nie wchodzącym w skład wektora różnicowego,
\item Liczba wektorów różnicowych
\end{enumerate}

W związku z powyższym, metody mutacji mogą przyjąć następującą nomenklature \textsl{DE/1/2/}. Wybór danej metody jest zależny od specyfiki rozwiązywanego problemu i dla różnych jego instancji może dawać odmienne rezultaty. Poniżej zestawione zostały strategie najczęściej spotykane w literaturze i opisane szczegółowo w \cite{doktorat}:
%---------------------------------------------------------------------------

\section{Strategia I: DE/rand/1}\label{sec:strukturaDokumentu}

Podstawową strategią  stosowaną jako operator mutacji jest strategia zakładająca wylosowanie spośród grupy rozrodczej osobnika mającego stanowić wektor celu. Jako, że grupa rozrodcza skłąda się z trzech osobników, dwa pozostałe są odpowiedzialne za utworzenie wektora różnicowego. Jest to zgodne z poniższym wzorem:
$$
 \forall U_{i} =S_{r_{1}i} + F \cdot (S_{r_{2}i} - S_{r_{3}i})
$$
Gdzie:\\
$$

%---------------------------------------------------------------------------

\section{Strategia II: DE/best/1}\label{sec:kompilacja}

%---------------------------------------------------------------------------

\section{Strategia III: DE/rand/$n_{v}$}\label{sec:narzedzia}

%---------------------------------------------------------------------------

\section{Strategia IV: DE/rand to best/$n_{v}$}\label{sec:narzedzia}

%---------------------------------------------------------------------------

\section{Strategia V: DE/current to best/$n_{v} +1$}\label{sec:narzedzia}