\chapter{Wstęp}


Cel pracy stanowi opracowanie modelu matematycznego dla kwadratowego zagadnenia przydziału (Quadratic Assigment Problem). Ze względu na fakt, iż QAP należy do klasy problemów NP-trudnych, a więc problemów dla których nie istnieje jednoznacznie określony algorytm dający optymalne rezultaty, istnieje konieczność opracowania modelu matematycznego dającego satysfakcjonujące efekty w przypadku konretnie postawionego zagadnienia. Bardzo dobrych wyników optymalizacji można spodziewać się wówczas gdy opracowany algorytm wykorzystuje specyficzne cechy oraz regularności rozwiązywanego problemu. 

W celu uzyskania satysfakcjonujących wyników dla problemów NP-trudnych, można posłużyć się metodami metaheurystycznymi, a więc metodami inspirującymi się zjawiskami zachodzącymi w naturze i dającymi rozwiązania przybliżone. Jedną z metod często stosowaną w celu rozwiązywania problemów optymalizacji globalnej jest algorytm ewolucji różnicowej. Algorytm ten w dużej mierze oparty jest na losowości oraz rozwiązaniach o charakterze intuicyjnym, niemniej jednak dajacych oczekiwane rezultaty. W poniżej pracy model matematyczny powstanie właśnie w oparciu o algorytm ewolucji różnicowej przy uwzględnieniu koniecznych modyfikacji wynikających bezpośrednio z charakteru problemu kwadratowego zagadnienia przydziału. Ze względu na konieczne modyfikacje należy zastosować zmodyfikowany algorytm ewolucji różnicowej. 


 W ramach pracy zostaną wykonane eksperymenty obliczeniowe a następnie przeprowadzona będzie analiza uzyskanych wyników dla rzeczywistych oraz testowych instancji zagadnienia. Testowe instancje kwadratowego zagadnienia przydziału zostały określone w bibliotece QAPLIB i w oparciu o nią będą przeprowadzane testy algorytmu. Eksperymenty będą prowadzone w oparciu o implementacje kilku różnych wersji poszczególnych fragmentów zmodyfikowanego algorytmu ewolucji różnicowej , a więc selekcje, mutacje, krzyżowanie.