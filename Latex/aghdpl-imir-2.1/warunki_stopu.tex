\chapter{Dodatkowe modyfikacje w obrębie algorytmu ewolucji różnicowej}\label{cha:pierwszyDokument}
\section{Sekwencyjny i równoległy algorytm ewolucji różnicowej }
Równoległy opiera się na zastosowaniu 3 populacji osobników o jednakowej liczności. W każdej iteracji, na podstawie populacji rodziców tworzona
jest populacja osobników próbnych a następnie potomków. Z kolei w sekwencyjnej
ewolucji różnicowej, w danej iteracji stosowana jest tylko jedna populacja, w której potomek porównywany jest od razu ze swoim rodzicem - oczywiście osobnik
o wyższej wartości funkcji przystosowania zastępuje swojego rodzica.

\section{Krzyżowa ewolucja różnicowa}