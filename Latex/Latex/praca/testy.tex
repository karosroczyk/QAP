\chapter{Testy}\label{cha:pierwszyDokument}


%Ogolnie :

%zlozonosc algorytmu, zajmowana pamiec


%Poszczegolnych:
%szybkosc funkcji, odchylenie standardowe , srednie itp z np 20 runów, runy zrobic tez dla roznych ilosci iteracji bo moze sie cos nie wyrabia i potem jest lepiej 
%---------------------------------------------------------------------------

\section{Dane wejściowe instancja I}

Dane wejściowe zostały zaczerpnięte z~biblioteki \cite{qaplib}. Znajdowały się~tam zarówno macierze trasy, kosztu, jak~i~informacje o~najmniejszej wartości funkcji celu, jaką udało się~uzyskać dla~danej instancji danych wejściowych. Dane rozważane w~tym rozdziale są~rozmiaru równego 12. Macierz testowa jest~macierzą symetryczną względem przekątnej.
\par
$$
\mathbf{Macierz\_trasy} =
\left( \begin{array}{cccccccccccc}
0& 1& 2& 2& 3& 4& 4& 5& 3& 5& 6& 7\\
1& 0& 1& 1& 2& 3& 3& 4& 2& 4& 5& 6\\
2& 1& 0& 2& 1& 2& 2& 3& 1& 3& 4& 5\\
2& 1& 2& 0&1& 2& 2& 3& 3& 3& 4& 5\\
3& 2& 1& 1& 0& 1& 1& 2& 2& 2& 3& 4\\
4& 3& 2& 2& 1& 0& 2& 3& 3& 1& 2& 3\\
4& 3& 2& 2& 1& 2& 0& 1& 3& 1& 2 & 3\\
5& 4& 3& 3& 2& 3& 1& 0& 4& 2& 1& 2\\
3& 2& 1& 3& 2& 3& 3& 4& 0& 4& 5& 6\\
5& 4& 3& 3& 2& 1& 1& 2& 4& 0& 1& 2\\
6& 5& 4&4& 3& 2& 2& 1& 5& 1& 0& 1\\
7& 6& 5& 5& 4& 3& 3& 2& 6& 2& 1& 0 \\
\end{array} \right)
$$

\par
$$
\mathbf{Macierz\_kosztu} =
\left( \begin{array}{cccccccccccc}
0&3&4&6&8&5&6&6&5&1&4&6\\
3&0&6&3&7&9&9&2&2&7&4&7\\
4&6&0&2&6&4&4&4&2&6&3&6\\
6&3&2&0&5&5&3&3&9&4&3&6\\
8&7&6&5&0&4&3&4&5&7&6&7\\
5&9&4&5&4&0&8&5&5&5&7&5\\
6&9&4&3&3&8&0&6&8&4&6&7\\
6&2&4&3&4&5&6&0&1&5&5&3\\
5&2&2&9&5&5&8&1&0&4&5&2\\
1&7&6&4&7&5&4&5&4&0&7&7\\
4&4&3&3&6&7&6&5&5&7&0&9\\
6&7&6&6&7&5&7&3&2&7&9&0\\
\end{array} \right)
$$

\subsection{Metody reprodukcji}\label{reprodukcja}

Analiza metod reprodukcji w~obrębie jednej instancji została przeprowadzona przy założeniu stałych takich jak:
\begin{itemize}
\item
macierz odległości, 
\item 
macierz przepływu,
\item
metoda reprodukcji,
\item
metoda krzyżowania,
\item
wielkość populacji
\end{itemize}
\par
\vspace{0,4cm}
Testy zostały przeprowadzone dla~instancji danych wejściowych, zaczerpniętych z~\cite{qaplib}, gdzie również znajduje się~informacja o~najlepszym otrzymanym wyniku funkcji celu dla~danej instancji danych, wynosił on 1652. Dzięki temu możliwe jest~określenie jak~blisko rozwiązania znajduje się~wynik działania algorytmu dla~każdej z~metod. W~celu uzyskania ostatecznego rozwiązania, a~więc rozwiązania niezmieniającego się~dalej pod~wpływem kolejnych iteracji algorytmu, analiza została przeprowadzona dla~odpowiednio 20 000 oraz~50 000 iteracji.\\



\subsubsection{20 000 iteracji}
\par
W~\ref{instancja1} zostały zestawione wartości błędu względnego funkcji celu w~stosunku do~wartości, która~zgodnie z~\cite{qaplib} jest~najmniejszą znaną wartością dla~danej instancji danych. Testy zostały przeprowadzone dwudziestokrotnie dla~poszczególnych metod reprodukcji. Metody te~zostały szczegółowo opisane w~rozdziale 3. W~metodach takich jak~metoda Losowa 2, Rankingowa 2 oraz~Turniejowa 2 nie~istnieje warunek co do~niepowtarzalności się~osobników w~obrębie jednej grupy rozrodczej. Dla~każdej z~metod został policzony średni błąd względny wartości funkcji celu, a~także odchylenie standardowe. Za~pomocą tych narzędzi statystycznych można określić zachowanie poszczególnych metod oraz~wyodrębnić metodę dającą w~tym kontekście najbardziej satysfakcjonujący wynik.\\
\par
\begin{table}[h!]
\begin{center}
\caption{Wartości błędu względnego funkcji celu dla poszczególnych metod reprodukcji, 20 000 iteracji.}
\scalebox{0.6}{
\begin{tabular}{|c|c|c|c|c|c|c|c|c|}
\hline
\textbf{Iteracja}  &\textbf{Losowa}  & \textbf{Rankingowa} & \textbf{Ruletka} & \textbf{Turniejowa} & \textbf{Elitarna} & \textbf{Losowa 2} & \textbf{Rankingowa 2} & \textbf{Turniejowa 2}\\
\hline
 \textbf{1}&1,45 \%&0,60\%&2,9\%&0,84 \%&2,05\%&2,17\%&0,96\%&6,53\% \\
\hline
 \textbf{2}&1,69\%&0,60\%&0,72\%&0,84\%&1,45\%&1,81\%&1,81\%&6,29\% \\
\hline
 \textbf{3}&0,96\%&0,72\%&3,51\%&1,08\%&1,69\%&1,81\%&1,21\%&8,71\%  \\
\hline
 \textbf{4}&0,84\%&0,72\%&2,54\%&2,66\%&1,81\%&1,69\%&3,02\%&5,44\%  \\
\hline
 \textbf{5}&0,84\%&1,21\%&2,42\%&3,26\%&2,42\%&1,21\%&1,08\%&10,16\%  \\
\hline
 \textbf{6}&1,08\%&0,48\%&3,26\%&1,69\%&1,08\%&1,81\%&1,08\%&8,83\%\\
\hline
 \textbf{7}&0,84\%&0,84\%&3,38\%&2,3\%&1,08\%&1,69\%&3,02\%&6,9\% \\
\hline
 \textbf{8}&1,08\%&0,24\%&4,23\%&1,45\%&0,96\%&2,17\%&2,05\%&2,9\%\\
\hline
 \textbf{9}&2,42\%&084\%&0,60\%&0,24\%&1,69\%&3,38\%&1,57\%&3,26\%\\
\hline
 \textbf{10}&0,96\%&0,24\%&1,08\%&1,45\%&1,33\%&3,87\%&3,26\%&10,53\%\\
\hline
 \textbf{11}&1,81\%&0,48\%&0,96\%&1,08\%&2,3\%&3,63\%&1,81\%&7,62\% \\
\hline
 \textbf{12}&2,66\%&0,48\%&3,87\%&2,17\%&1,57\%&4,6\%&0,48\%&7,14\%\\
\hline
 \textbf{13}&0,96\%&0,12\%&2,42\%&1,69\%&1,93\%&3,02\%&1,21\%&7,38\% \\
\hline
 \textbf{14}&0,96\%&0,48\%&0,96\%&1,21\%&3,26\%&2,66\%&2,78\%&4,6\%\\
\hline
 \textbf{15}&2,66\%&0,72\%&1,21\%&2,17\%&1,93\%&1,93\%&1,93\%&9,32\%\\
\hline
 \textbf{16}&3,75\%&0,48\%&2,05\%&1,45\%&1,33\%&1,45\%&2,3\%&7,38\%\\
\hline
 \textbf{17}&0,60\%&0,24\%&2,9\%&1,33\%&1,45\%&1,93\%&1,93\%&6,41\% \\
\hline
 \textbf{18}&1,57\%&0,60\%&1,21\%&1,08\%&2,3\%&2,42\%&1,69\%&5.69\%\\
\hline
 \textbf{19}&1,09\%&0,84\%&3,99\%&1,21\%&1,69\%&3,63\%&0,60\%&6,65\%\\
\hline
 \textbf{20}&1,93\%&0,48\%&1,81\%&0,84\%&1,33\%&0,84\%&1,08\%&4,11\%\\
\hline
 \textbf{ŚREDNIA}&1,51\%&0,58\%&2,31\%&1,51\%&1,74\%&2,39\%&1,75\%&6,8\%\\
\hline
 \textbf{ODCHYLENIE STANDARDOWE}&0,8\%&0,25\%&1,15\%&0,7\%&0,54\%&0,97\%&0,79\%&6,8\%\\
\hline
\end{tabular}}
\label{instancja1}
\end{center}
\end{table}


Na~podstawie powyższej tabeli zauważalny staje się~fakt, iż~najbardziej satysfakcjonujące wyniki w~przeciągu 20 000 iteracji otrzymuje metoda rankingowa. W~17 na~20 przypadków uzyskała ona najmniejsze wartości funkcji celu. W~pozostałych 3 przypadkach znajduje się~na~drugim bądź~też trzecim miejscu. Posiada ona również najmniejszy współczynnik odchylenia standardowego oraz~współczynnik zmienności co świadczy o~stosunkowo najbliższym spośród wszystkich metod położeniu wyników wokół wartości średniej. Poniżej zestawiony został ranking metod selekcji utworzony w~zależności od~uzyskanej wartości średniej funkcji celu.\\

\begin{table}[h!]
\begin{center}
\caption{Ranking metod reprodukcji na podstawie średniej wartości błędu względnego funkcji celu, 20 000 iteracji.}
\scalebox{0.8}{
\begin{tabular}{|c|c|c|c|}
\hline
\textbf{Miejsce}  &\textbf{Metoda}  & \textbf{Średni błąd względny funkcji celu} & \textbf{Liczba osiągnietych wartośći 0\%}\\
\hline
 \textbf{1}&Rankingowa&0,58\%& 0 \\
\hline
 \textbf{2}&Turniejowa&1,51 \%& 0\\
\hline
 \textbf{3}&Losowa&1,51 \%& 0\\
\hline
 \textbf{4}&Elitarna&1,74 \%& 0\\
\hline
 \textbf{5}&Rankingowa 2&1,75\%& 0\\
\hline
 \textbf{6}&Ruletki&2,31\%& 0\\
\hline
 \textbf{7}&Losowa 2&2,39\%& 0\\
\hline
 \textbf{8}&Turniejowa 2&6,8\%& 0\\
\hline
\end{tabular}}
\label{ranking_1}
\end{center}
\end{table}

Widoczny staje się~więc fakt, iż~lepiej działają metody, w~których stosujemy ograniczenia co do~konieczności niepowtarzania się~danego rodzica w~grupie rodzicielskiej. A~więc~metody Rankingowa, Turniejowa, Losowa oraz~Elitarna. W~gronie metod niestosujących się~do~zasady indywidualności osobnika najwyższe miejsce w~rankingu otrzymuje również metoda Rankingowa. Przebieg działania algorytmu dla~poszczególnych metod można przenalizować na~podstawie wykresu \ref{ranking}. Na~wykresie tym został zaprezentowany przebieg, dla~którego~to wartość końcowa uzyskana przez~metodę rankingową jest~jednocześnie najmniejszą wartością funkcji celu otrzymaną w~we wszystkich 20 iteracjach.\\
\begin{figure}[ht]
		\includegraphics[scale=0.6]{../../../../Screeny/matody_1654_legend.png}
		\caption{Przebieg działania algorytmu dla~poszczególnych metod reprodukcji, 20 000 iteracji.}
		\label{ranking}			
\end{figure}

Na~podstawie wykresu \ref{instancja1} zauważalne staje się~zmniejszanie wartości funkcji celu w~kolejnych iteracjach algorytmu. W~przypadku metod takich jak~metoda ruletki czy~też metoda turniejowa, która~dopuszcza powtórzeń wśród osobników w~grupie rodzicielskiej, stabilizacja poziomu następuje w~początkowych iteracjach i~utrzymuje się~tak do~końca. Powodem tak~szybkiej zbieżności tych metod jest~brak różnorodności w~kolejnych grupach rodzicielskich. W~metodzie ruletki spowodowane jest~to~dużym prawdopodobieństwem wylosowania osobnika najlepszego, co prowadzi do~jego dominacji nad~innymi. Natomiast w~metodzie turniejowej turniej jest~w~większości przypadków wygrywany przez~tego samego osobnika, co przy dodatkowym założeniu, iż~w~jednej grupie rodzicielskiej mogą znajdować się~te~same osobniki prowadzi do~sytuacji w~której na~grupę rodzicielską składają się~3 te~same osobniki. Pomimo wrażenia, iż~rozwiązania osiągnęły stan stabilny już w~20 000 iteracji, należy sprawdzić, czy~nie nastąpi jeszcze spadek wartości funkcji celu, gdy liczba iteracji zostanie zwiększona. W~tym celu ponowione zostają badania metod dla~50 000 iteracji.\\

\subsubsection{50 000 iteracji}

 W~tabeli \ref{instancja2} zostały zestawione wartości błędu względnego funkcji celu w~kolejnych iteracjach dla~poszczególnych metod. W~metodach takich jak~metoda Losowa 2, Rankingowa 2 oraz~Turniejowa 2 nie~istnieje warunek co do~niepowtarzalności się~osobników w~obrębie jednej grupy rozrodczej. W~tym przypadku zwiększona została liczba iteracji działania algorytmu dla~każdej z~metod z~20 000 na~50 000. Kolorem zielonym zostały zaznaczone wyniki działania algorytmu osiągające wartość błędu względnego równą 0\%.

\begin{table}[h!]
\begin{center}
\caption{Wartości błędu względnego funkcji celu dla poszczególnych metod reprodukcji, 50 000 iteracji.}
\scalebox{0.6}{
\begin{tabular}{|c|c|c|c|c|c|c|c|c|}
\hline
\textbf{Iteracja}  &\textbf{Losowa}  & \textbf{Rankingowa} & \textbf{Ruletka} & \textbf{Turniejowa} & \textbf{Elitarna} & \textbf{Losowa 2} & \textbf{Rankingowa 2} & \textbf{Turniejowa 2}\\
\hline
 \textbf{1}&2,3 \%&0,6\%&4,47\%&0,24\%&1,21\%&2,66\%&1,93\%&4,6\% \\
\hline
 \textbf{2}&1,93\%&0,6\%&2,05\%&0,6\%&1,33\%&1,45\%&2,42\%&4,47\% \\
\hline
 \textbf{3}&0,24\%&1,81\%&1,93\%&1,33\%&1,93\%&1,21\%&1,21\%&6,29\%  \\
\hline
 \textbf{4}&3,99\%&0,24\%&1,81\%&1,21\%&1,45\%&2,42\%&0,48\%&7,99\%  \\
\hline
 \textbf{5}&2,3\%&0,6\%&1,21\%&0,96\%&2,05\%&4,47\%&1,45\%&8,35\%  \\
\hline
 \textbf{6}&1,69\%&0,48\%&4,47\%&2,3\%&1,45\%&2,54\%&1,21\%&9,2\%\\
\hline
 \textbf{7}&1,33\%&0,24\%&1,93\%&0,6\%&1,45\%&1,21\%&1,08\%&7,26\% \\
\hline
 \textbf{8}&2,42\%&0,84\%&1,81\%&0,96\%&0,72\%&1,39\%&2,05\%&6,77\%\\
\hline
 \textbf{9}&0,84\%&\color{green}\textbf{0\%}&0,6\%&0,72\%&1,08\%&2,05\%&2,42\%&8,83\%\\
\hline
 \textbf{10}&2,54\%&0,48\%&2,78\%&0,72\%&2,78\%&1,93\%&2,66\%&5,69\%\\
\hline
 \textbf{11}&2,54\%&0,24\%&1,69\%&1,33\%&1,45\%&3,14\%&3,14\%&3,87\%\\
\hline
 \textbf{12}&0,6 \%&0,48\%&2,05\%&0,48\%&0,6\%&2,66\%&0,96\%&10,53\%\\
\hline
 \textbf{13}&1,33\%&0,84\%&3,14\%&1,08\%&1,21\%&2,66\%&1,45\%&8,59\%\\
\hline
 \textbf{14}&0,84\%&0,24\%&2,3\%&1,45\%&1,21\%&4,47\%&1,21\%&5,69\%\\
\hline
 \textbf{15}&2,54\%&1,45\%&1,33\%&0,6\%&2,3\%&3,63\%&1,81\%&6,29\%\\
\hline
 \textbf{16}&1,08\%&0,6\%&2,66\%&0,96\%&1,81\%&1,45\%&1,08\%&10,53\%\\
\hline
 \textbf{17}&1,81\%&\color{green}\textbf{0\%}&1,33\%&0,72\%&1,21\%&3,99\%&0,48\%&9,2\%\\
\hline
 \textbf{18}&1,57\%&0,48\%&3,26\%&1,93\%&1,93\%&2,17\%&1,45\%&7,14\%\\
\hline
 \textbf{19}&1,08\%&0,48\%&1,57\%&0,84\%&2,3\%&3,99\%&0,96\%&6,53\%\\
\hline
 \textbf{20}&1,93\%&0,6\%&3,14\%&2,42\%&2,05\%&1,45\%&1,93\%&10,53\%\\
\hline
 \textbf{ŚREDNIA}&1,75\%&0,57\%&2,4\%&1,08\%&1,47\%&2,55\%&1,57\%&7,42\%\\
\hline
 \textbf{ODCHYLENIE STANDARDOWE}&0,85\%&0,42\%&0,93\%&0,57\%&0,51\%&1,06\%&0,7\%&1,99\%\\
\hline
\end{tabular}}
\label{instancja2}
\end{center}
\end{table}

W~przypadku 50 000 iteracji zauważalne jest~uzyskiwanie lepszych średnich wyników przez~metody takie jak: metoda rankingowa, metoda elitarna oraz~turniejowa dopuszczające istnienie powtórzeń w~grupie rodzicielskiej. Dzieje się~tak, ponieważ~metody te~nie kończą stabilizacji na~poziomie 20 000, a~spadek wartości funkcji celu następuje również w~dalszych iteracjach. Pozostałe metody utrzymały średnią na~tym samym bądź~też nieco wyższym poziomie w~stosunku to~próby testowej zakładającej 20 000 iteracji. Poniżej w~tabeli \ref{ranking2} zestawiony został ranking metod uporządkowany od~metody działającej najlepiej, na~pierwszym miejscu, do~metody działającej najmniej korzystnie, na~ostatnim.

\begin{table}[h!]
\begin{center}
\caption{Ranking metod reprodukcji na podstawie średniej wartości błędu względnego funkcji celu, 50 000 iteracji.}
\scalebox{0.8}{
\begin{tabular}{|c|c|c|c|}
\hline
\textbf{Miejsce}  &\textbf{Metoda}  & \textbf{Średni błąd względny funkcji celu} & \textbf{Liczba osiągnietych wartośći 0\%}\\
\hline
 \textbf{1}&Rankingowa&0,57\%& 2\\
\hline
 \textbf{2}&Turniejowa&1,08\%& 0\\
\hline
 \textbf{3}&Elitarna&1,47\%& 0\\
\hline
 \textbf{4}&Rankingowa 2&1,57\%& 0\\
\hline
 \textbf{5}&Losowa&1,75\%& 0\\
\hline
 \textbf{6}&Ruletki&2,4\%& 0\\
\hline
 \textbf{7}&Losowa 2&2,55\%& 0\\
\hline
 \textbf{8}&Turniejowa 2&7,42\%& 0\\
\hline
\end{tabular}}
\label{ranking2}
\end{center}
\end{table}

Na~podstawie tabeli \ref{ranking2} można wysunąć wniosek, iż~po raz kolejny najlepsze wyniki uzyskuje metoda rankingowa, a~zaraz po~niej powtórnie najkorzystniej zachowuje się~metoda turniejowa. Niemniej jednak przodowanie tych metod może mieć miejsce w~przypadku założonej instancji danych wejściowych, dlatego należy przeprowadzić testy przy użyciu innej macierzy odległości oraz~przepływu. Przebieg funkcji celu we wszystkich 50 000 iteracjach widoczny jest~na~wykresie \ref{ranking2picun}, na~którym to~zauważyć można również spadek dla~metody turniejowej, w~której istnieje zastrzeżenie co do~niepowtarzalności się~osobników w~grupie rozrodczej. Wyjątkowo niekorzystnym działaniem odznacza się~metoda Turniejowa 2, gdyż spadek następuje jedynie w~początkowych iteracjach i~na tym poziomie utrzymuje się~już do~końca.

\begin{figure}[h!]
		\includegraphics[scale=0.6]{../../../../Screeny/50_tys.png}
		\caption{Przebieg działania algorytmu dla~poszczególnych metod reprodukcji, 50 000 iteracji.}
		\label{ranking2picun}			
\end{figure}

Możliwe jest~zatem ustalenie optymalnej ilości iteracji na~liczbę 30 000, wówczas wartości funkcji celu w~poszczególnych metodach osiągną już stan stabilny, a~dodatkowe obliczenia dla~liczby równej aż 50 000 iteracji nie~będą musiały być~wykonywane.

\subsection{Metody mutacji}\label{mutacja}

Metody mutacji o~których mowa w~podrozdziale zostały szczegółowo opisane w~rozdziale 4. Testy metod mutacji w~obrębie jednej instancji danych zostały przeprowadzone przy uprzednim określeniu:
\begin{itemize}
\item
macierzy odległości,
\item
macierzy przepływu,
\item
metod reprodukcji,
\item
metoda krzyżowania,
\item
wielkość populacji,
\end{itemize}
\par
W~klasycznych metodach mutacji stosowany jest~wybór osobników do~grupy rozrodczej poprzez~metodę losową. Niemniej jednak ze~względu na~wyjątkowo dobre wyniki uzyskiwane przez~metodę rankingową w~testach przeprowadzanych w~rozdziale 6.1.1, postanowiono sprawdzić jej działanie w~połączeniu z~różnymi wariantami metod mutacji. Liczba iteracji algorytmu była równa, podobnie jak~w~powyższych testach, 20 000 iteracji. Następnie, w~celu zbadania czy~dalej niż granica 20 000 następuje spadek wartości funkcji celu, również dla~50 000 iteracji. Zgodnie z~podrozdziałem 4.2.4 metody mutacji zostaną dodatkowo przetestowane pod~kątem wpływu na~ich działanie parametru $\leftthreetimes$. Również dynamiczna zmiana parametru mutacji poddawana jest~testom.

%----------------------------------------------------------------------------------------------------------------------------------------------------------------------------------------------------

\subsubsection{Reprodukcja losowa}
W~rozdziale tym testy metod mutacji zostały przeprowadzone z~użyciem metody reprodukcji wykorzystywanej w~klasycznej wersji algorytmu ewolucji różnicowej. Metodą decydującą o~osobnikach wchodzących w~skład grupy rozrodczej jest~więc metoda losowa.\\
\par

\begin{itemize}
\item  \textbf{20 000 iteracji}\\

Wyniki zostały zaprezentowane w~formie tabeli zawierającej wartości błędu względnego funkcji celu. Testy przeprowadzono dla~20 powtórzeń algorytmu. W~wyniku tego działania możliwe było policzenie wartości średniej oraz~odchylenia standardowego z~kolejnych 20 prób.

\begin{table}[h!]
\begin{center}
\caption{Wartości błędu względnego funkcji celu dla poszczególnych metod mutacji, metoda losowa, 20 000 iteracji.}
\scalebox{0.55}{
\begin{tabular}{|c|c|c|c|c|c|c|c|c|}
\hline
\textbf{Iteracja}  &\textbf{DE/rand/1}  & \textbf{DE/rand/1+$\leftthreetimes$} & \textbf{DE/best/1} & \textbf{DE/best/1+$\leftthreetimes$} & \textbf{DE/rand/$n_{v}$} & \textbf{DE/rand/$n_{v}$+$\leftthreetimes$} & \textbf{ DE/current to best/$n_{v}+1$} & \textbf{ DE/current to best/$n_{v}+1$+$\leftthreetimes$}\\\hline
\textbf{1}&1,21\%&\color{green}\textbf{0}\%&3,39\%&1,94\%&0,85\%&0,48\%&0,73\%&0,61\% \\ \hline
\textbf{2}&0,97\%&0,61\%&4,84\%&0,85\%&1,21\%&1,33\%&0,61\%&0,73\% \\ \hline
\textbf{3}&2,91\%&0,61\%&2,66\%&1,21\%&1,33\%&0,85\%&0,12\%&0,12\% \\ \hline
\textbf{4}&2,18\%&1,33\%&1,33\%&1,45\%&1,82\%&0,61\%&0,61\%&1,09\% \\ \hline
\textbf{5}&2,06\%&2,42\%&2,42\%&3,51\%&1,57\%&1,69\%&1,09\%&0,24\% \\ \hline
\textbf{6}&1,57\%&1,94\%&2,54\%&1,82\%&0,85\%&0,24\%&0,85\%&0,24\% \\ \hline
\textbf{7}&3,15\%&1,69\%&3,75\%&0,85\%&2,54\%&1,09\%&1,09\%&0,48\% \\ \hline
\textbf{8}&1,57\%&1,21\%&3,39\%&0,85\%&1,45\%&1,69\%&1,21\%&0,48\% \\ \hline
\textbf{9}&1,82\%&4,84\%&5,08\%&1,82\%&0,97\%&0,48\%&0,61\%&0,61\% \\ \hline
\textbf{10}&1,33\%&0,61\%&4,24\%&1,69\%&1,33\%&2,18\%&0,48\%&\color{green}\textbf{0}\% \\ \hline
\textbf{11}&2,42\%&0,48\%&3,51\%&0,73\%&1,21\%&0,24\%&0,48\%&0,24\% \\ \hline
\textbf{12}&3,15\%&1,69\%&1,82\%&1,69\%&2,18\%&1,82\%&0,48\%&0,61\% \\ \hline
\textbf{13}&1,21\%&0,48\%&1,33\%&0,85\%&4,60\%&0,48\%&0,48\%&0,48\% \\ \hline
\textbf{14}&1,21\%&\color{green}\textbf{0}\%&3,03\%&1,21\%&1,33\%&0,85\%&0,97\%&0,24\% \\ \hline
\textbf{15}&1,21\%&0,48\%&1,45\%&0,97\%&1,45\%&0,48\%&1,21\%&0,61\% \\ \hline
\textbf{16}&0,61\%&1,33\%&2,30\%&1,21\%&0,48\%&2,91\%&0,85\%&0,12\% \\ \hline
\textbf{17}&1,94\%&0,48\%&4,24\%&3,39\%&0,61\%&\color{green}\textbf{0}\%&1,21&\color{green}\textbf{0}\% \\ \hline
\textbf{18}&3,15\%&0,48\%&2,42\%&2,06\%&2,18\%&2,42\%&0,48\%&0,85\% \\ \hline
\textbf{19}&4,24\%&1,94\%&6,17\%&1,45\%&0,85\%&3,75\%&0,61&\color{green}\textbf{0}\% \\ \hline
\textbf{20}&0,48\%&0,24\%&1,45\%&0,73\%&1,33\%&2,78\%&0,48\%&\color{green}\textbf{0}\% \\ \hline
\textbf{ŚREDNIA}&1,92\%&1,14\%&3,07\%&1,51\%&1,51\%&1,32\%&0,73\%&0,39\% \\ \hline
\textbf{ODCHYLENIE}&0,96\%&1,09\%&1,33\%&0,77\%&0,88\%&1,02\%&0,30\%&0,31\% \\ \hline
\end{tabular}}
\label{losowa20}
\end{center}
\end{table}

Sumaryczna liczba osiągnięcia przez~algorytm wartości błędu względnego równego 0\% jest~równa 7, z~czego 4 są~osiągane w~wyniku zastosowania w~algorytmie metody DE/current to~best/$n_{v}+1$.

\begin{table}[h!]
\begin{center}
\caption{Ranking metod mutacji na podstawie średniej wartości błędu względnego funkcji celu, metoda losowa, 20 000 iteracji.}
\scalebox{0.8}{
\begin{tabular}{|c|c|c|c|}
\hline
\textbf{Miejsce}  &\textbf{Metoda}  & \textbf{Średni błąd względny funkcji celu} & \textbf{Liczba osiągnietych wartośći 0\%}\\\hline
 \textbf{1}&DE/current to~best/$n_{v}+1$+$\leftthreetimes$&0,39\%& 4\\\hline
 \textbf{2}& DE/current to~best/$n_{v}+1$ &0,73\%& 0\\\hline
 \textbf{3}&DE/rand/$n_{v}$+$\leftthreetimes$&1,14\%& 1\\\hline
 \textbf{4}&DE/rand/1+$\leftthreetimes$&1,32\%& 2\\\hline
 \textbf{5}&DE/rand/$n_{v}$&1,51\%&0\\\hline
 \textbf{6}&DE/best/1+$\leftthreetimes$1&1,51\%& 0\\\hline
 \textbf{7}&DE/rand/1&1,92\%& 0\\\hline
 \textbf{8}&DE/best/1&3,07\%& 0\\\hline
\end{tabular}}
\label{ranking5}
\end{center}
\end{table}

Na~podstawie tabeli \ref{ranking5} widoczny staje się~fakt, iż~najmniejszą średnią wartością błędu względnego, o~wartości 0,39\%, charakteryzuje się~metoda DE/current to~best/$n_{v}+1$+$\leftthreetimes$. W~zestawieniu z~pozostałymi metodami mutacji przyjmuje ona około dwukrotnie lepsze rezultaty, jako, że~pozostałe metody uzyskują wartości błędu względnego wahające się~w~zakresie 0,73\% - 3,07\%. Wartość odchylenia standardowego dla~tej metody wynosi 0,31\% co w~porównaniu z~pozostałymi testowanymi metodami mutacji jest~wartością stosunkowo niewielką. Metoda DE/current to~best/$n_{v}+1$+$\leftthreetimes$ uzyskuje korzystniejsze rezultaty w~porównaniu do~jej odpowiednika bez modyfikatora DE/current to~best/$n_{v}+1$. W~przypadku pozostałych metod, DE/rand/$n_{v}$, DE/rand/1 oraz~DE/best/1, modyfikacja współczynnikiem $\leftthreetimes$ również pozwala na~uzyskiwanie przez~te metody bardziej zadowalających efektów.\\


\item  \textbf{50 000 iteracji}\\

Poniższe testy w~stosunku do~poprzednich zestawionych w~tabelach \ref{losowa20} oraz~\ref{ranking5} różnią się~od~siebie liczbą iteracji algorytmu, zwiększoną z~20 000 do~50 000. Działanie to~wprowadzone jest~w~celu zbadania czy~nie następuje spadek wartości funkcji celu w~iteracjach dalszych niż 20 000. Dodatkowym atutem jest~możliwość konfrontacji wyników, a~więc sprawdzenia, czy~uzyskane w~poprzednich testach wartości utrzymują się~na~tym samy bądź~też lepszym poziomie.

\begin{table}[h!]
\begin{center}
\caption{Wartości błędu względnego funkcji celu dla poszczególnych metod mutacji, metoda losowa, 50 000 iteracji.}
\scalebox{0.55}{
\begin{tabular}{|c|c|c|c|c|c|c|c|c|}
\hline
\textbf{Iteracja}  &\textbf{DE/rand/1}  & \textbf{DE/rand/1+$\leftthreetimes$} & \textbf{DE/best/1} & \textbf{DE/best/1+$\leftthreetimes$} & \textbf{DE/rand/$n_{v}$} & \textbf{DE/rand/$n_{v}$+$\leftthreetimes$} & \textbf{ DE/current to best/$n_{v}+1$} & \textbf{ DE/current to best/$n_{v}+1$+$\leftthreetimes$}\\\hline
\textbf{1}&1,09\%&0,85\%&2,06\%&2,91\%&0,85\%&1,57\%&0,12\%&\color{green}\textbf{0}\% \\ \hline
\textbf{2}&1,82\%&0,48\%&2,91\%&2,06\%&0,73\%&1,09\%&0,48\%&\color{green}\textbf{0}\% \\ \hline
\textbf{3}&1,57\%&1,45\%&1,82\%&2,54\%&1,57\%&0,48\%&0,12\%&0,48\% \\ \hline
\textbf{4}&0,97\%&0,85\%&2,54\%&2,54\%&1,45\%&0,61\%&0,24\%&\color{green}\textbf{0}\% \\ \hline
\textbf{5}&2,91\%&0,73\%&2,18\%&0,73\%&0,85\%&0,73\%&\color{green}\textbf{0}\%&0,48\% \\ \hline
\textbf{6}&1,09\%&1,82\%&2,54\%&0,48\%&1,45\%&1,69\%&0,61\%&0,97\% \\ \hline
\textbf{7}&0,73\%&2,06\%&3,27\%&1,94\%&1,21\%&2,42\%&0,48\%&\color{green}\textbf{0}\% \\ \hline
\textbf{8}&2,66\%&8,23\%&2,06\%&2,30\%&2,06\%&0,61\%&0,48\%&0,12\% \\ \hline
\textbf{9}&1,57\%&1,82\%&1,94\%&1,33\%&0,61\%&2,54\%&0,61\%&\color{green}\textbf{0}\% \\ \hline
\textbf{10}&1,94\%&2,42\%&1,21\%&1,09\%&1,45\%&0,97\%&0,12\%&\color{green}\textbf{0}\% \\ \hline
\textbf{11}&2,66\%&1,82\%&2,06\%&2,30\%&0,85\%&0,24\%&0,61\%&0,24\% \\ \hline
\textbf{12}&\color{green}\textbf{0}\%&3,51\%&4,36\%&2,54\%&2,30\%&\color{green}\textbf{0}\%&0,12\%&0,12\% \\ \hline
\textbf{13}&0,48\%&0,12\%&1,82\%&2,18\%&1,45\%&\color{green}\textbf{0}\%&0,48\%&\color{green}\textbf{0}\% \\ \hline
\textbf{14}&0,97\%&0,12\%&1,45\%&2,42\%&1,94\%&1,94\%&0,24\%&0,12\% \\ \hline
\textbf{15}&0,85\%&0,61\%&2,78\%&2,91\%&1,33\%&0,61\%&0,24\%&0,73\% \\ \hline
\textbf{16}&1,09\%&0,48\%&5,08\%&0,85\%&0,61\%&3,27\%&0,48\%&0,24\% \\ \hline
\textbf{17}&2,06\%&\color{green}\textbf{0}\%&2,42\%&1,57\%&0,85\%&1,21\%&0,48\%&0,12\% \\ \hline
\textbf{18}&0,48\%&\color{green}\textbf{0}\%&2,30\%&2,66\%&0,61\%&1,57\%&0,48\%&0,48\% \\ \hline
\textbf{19}&5,57\%&\color{green}\textbf{0}\%&2,78\%&2,30\%&6,90\%&\color{green}\textbf{0}\%&0,48\%&0,24\% \\ \hline
\textbf{20}&1,94\%&0,12\%&3,27\%&1,69\%&0,61\%&1,09\%&0,24\%&\color{green}\textbf{0}\% \\ \hline
\textbf{ŚREDNIA}&1,62\%&1,37\%&2,54\%&1,97\%&1,48\%&1,13\%&0,36\%&0,22\% \\ \hline
\textbf{ODCHYLENIE}&1,19\%&1,83\%&0,90\%&0,72\%&1,34\%&0,88\%&0,19\%&0,27\% \\ \hline
\end{tabular}}
\label{losowa50}
\end{center}
\end{table}

Liczba próbek, w~których algorytm uzyskiwał wartość błędu względnego równą 0\% wzrosła o~9 w~stosunku do~wersji algorytmu z~liczbą iteracji równą 20 000. Świadczy to~o~zmniejszaniu się~wartość funkcji celu w~iteracjach dalszych niż 20 000. W~przypadku większości z~metod następuje spadek średniej wartości błędu bezwzględnego funkcji celu. Niemniej jednak w~przypadku metod takich jak~DE/rand/1+$\leftthreetimes$ oraz~DE/best/1+$\leftthreetimes$ następuje wzrost tej wartości. Zjawisko to~spowodowane jest~zadziałaniem czynnika losowego. 

\begin{table}[h!]
\begin{center}
\caption{Ranking metod mutacji na podstawie średniej wartości błędu względnego funkcji celu, metoda losowa, 50 000 iteracji.}
\scalebox{0.8}{
\begin{tabular}{|c|c|c|c|}
\hline
\textbf{Miejsce}  &\textbf{Metoda}  & \textbf{Średni błąd względny funkcji celu} & \textbf{Liczba osiągnietych wartośći 0\%}\\\hline
 \textbf{1}&DE/current to~best/$n_{v}+1$ +$\leftthreetimes$&0,22\%& 8\\\hline
 \textbf{2}& DE/current to~best/$n_{v}+1$&0,36\%& 1\\\hline
 \textbf{3}&DE/rand/$n_{v}$+$\leftthreetimes$&1,13\%& 3\\\hline
 \textbf{4}&DE/rand/1+$\leftthreetimes$&1,37\%& 3\\\hline
 \textbf{5}&DE/rand/$n_{v}$&1,48\%&0\\\hline
 \textbf{6}&DE/rand/1&1,62\%&1\\\hline
 \textbf{7}&DE/best/1+$\leftthreetimes$&1,97\%& 0\\\hline
 \textbf{8}&DE/best/1&2,54\%& 0\\\hline
\end{tabular}}
\label{ranking3}
\end{center}
\end{table}

Na~podstawie tabeli będącej zestawieniem wszystkich testowanych metod mutacji, tabela \ref{ranking3}, potwierdza się~zauważona w~poprzednich testach zależność, iż~najlepiej działającą metodą jest~metoda DE/current to~best/$n_{v}+1$+$\leftthreetimes$. Średni błąd względny jest~równy 0,22\%, co jest~wartością około dwukrotnie mniejszą od~wyników uzyskiwanych przez~inne metody mutacji.

\end{itemize}

%----------------------------------------------------------------------------------------------------------------------------------------------------------------------------------------------------


\subsubsection{Reprodukcja rankingowa}
W~poniżej przeprowadzonej analizie za~metodę reprodukcji przyjmuje się~metodę rankingową. Testy zostaną przeprowadzone zarówno dla~20 000, jak~i~dla 50 000 iteracji.\\
\par
\begin{itemize}
\item  \textbf{20 000 iteracji}\\
\par
Analiza wyników przeprowadzona zostanie w~oparciu o~sprowadzenie uzyskanych wartości funkcji celu do~postaci ich błędu względnego. Liczba iteracji algorytmu jest~równa 20 000.

\begin{table}[h!]
\begin{center}
\caption{Wartości błędu względnego funkcji celu dla poszczególnych metod mutacji, metoda rankingowa, 20 000 iteracji.}
\scalebox{0.55}{
\begin{tabular}{|c|c|c|c|c|c|c|c|c|}
\hline
\textbf{Iteracja}  &\textbf{DE/rand/1}  & \textbf{DE/rand/1+$\leftthreetimes$} & \textbf{DE/best/1} & \textbf{DE/best/1+$\leftthreetimes$} & \textbf{DE/rand/$n_{v}$} & \textbf{DE/rand/$n_{v}$+$\leftthreetimes$} & \textbf{ DE/current to best/$n_{v}+1$} & \textbf{ DE/current to best/$n_{v}+1$+$\leftthreetimes$}\\
\hline
\textbf{1}&0,85\%&0,73\%&1,21\%&1,33\%&0,61\%&0,24\%&0,24\%&0,48\% \\ \hline
\textbf{2}&1,09\%&0,24\%&1,82\%&1,09\%&0,73\%&0,48\%&0,24\%&0,61\% \\ \hline
\textbf{3}&0,48\%&0,24\%&2,06\%&2,54\%&0,73\%&0,48\%&0,85\%&0,24\% \\ \hline
\textbf{4}&0,48\%&0,24\%&1,33\%&0,48\%&0,24\%&0,48\%&0,48\%&0,48\% \\ \hline
\textbf{5}&0,73\%&0,73\%&0,48\%&0,48\%&0,61\%&0,24\%&0,24\%&0,48\% \\ \hline
\textbf{6}&1,09\%&0,85\%&2,78\%&1,21\%&0,48\%&0,61\%&0,24\%&0,48\% \\ \hline
\textbf{7}&0,85\%&0,48\%&1,33\%&1,21\%&0,85\%&0,48\%&0,24\%&0,12\% \\ \hline
\textbf{8}&1,21\%&0,48\%&0,48\%&1,57\%&1,82\%&0,24\%&0,24\%&0,48\% \\ \hline
\textbf{9}&0,24\%&0,61\%&1,21\%&1,57\%&0,85\%&1,09\%&0,48\%&0,85\% \\ \hline
\textbf{10}&1,33\%&0,73\%&1,33\%&1,21\%&1,09\%&0,48\%&0,61\%&\color{green}\textbf{0}\% \\ \hline
\textbf{11}&0,85\%&\color{green}\textbf{0}\%&0,85\%&1,33\%&0,12\%&0,12\%&1,09\%&\color{green}\textbf{0}\% \\ \hline
\textbf{12}&0,24\%&0,48\%&1,69\%&1,57\%&0,97\%&0,48\%&\color{green}\textbf{0}\%&0,48\% \\ \hline
\textbf{13}&0,73\%&0,85\%&0,61\%&1,57\%&0,61\%&0,48\%&0,24\%&1,09\% \\ \hline
\textbf{14}&0,48\%&0,12\%&0,97\%&1,33\%&0,48\%&0,24\%&1,21\%&0,85\% \\ \hline
\textbf{15}&\color{green}\textbf{0}\%&0,24\%&0,48\%&0,61\%&1,21\%&0,24\%&\color{green}\textbf{0}\%&\color{green}\textbf{0}\% \\ \hline
\textbf{16}&0,48\%&0,61\%&2,30\%&1,69\%&0,61\%&0,73\%&0,48\%&\color{green}\textbf{0}\% \\ \hline
\textbf{17}&0,61\%&0,12\%&2,54\%&2,18\%&0,48\%&0,61\%&0,24\%&0,61\% \\ \hline
\textbf{18}&1,82\%&0,61\%&2,18\%&1,33\%&0,48\%&0,48\%&0,48\%&0,61\% \\ \hline
\textbf{19}&1,09\%&0,24\%&1,21\%&1,21\%&1,57\%&0,48\%&0,61\%&0,48\% \\ \hline
\textbf{20}&0,73\%&0,61\%&2,66\%&0,85\%&0,48\%&0,48\%&0,73\%&0,48\% \\ \hline
\textbf{ŚREDNIA}&0,77\%&0,46\%&1,48\%&1,32\%&0,75\%&0,46\%&0,45\%&0,44\% \\ \hline
\textbf{ODCHYLENIE}&0,41\%&0,25\%&0,73\%&0,49\%&0,41\%&0,21\%&0,32\%&0,30\% \\ \hline
\end{tabular}}
\label{rankingowa20}
\end{center}
\end{table}
\par
Analizując tabele \ref{rankingowa20} widoczny staje się~fakt, iż~najlepsze wyniki uzyskiwane są~dla strategii DE/current to~best/$n_{v}+1$+$\leftthreetimes$, jako że~aż 4 razy algorytm z~udziałem tej metody mutacji osiągał błąd względy równy 0\%. W~pozostałych iteracjach wartości błędu względnego w~przeważającej części również były niewielkie, wartości te~wahają się~w~zakresie (0,1,48\%). Współczynnik odchylenia standardowego dla~wspomnianej powyżej metody jest~równy 0,3 \%, co świadczy o~bliskim skupieniu wartości w~poszczególnych iteracjach wokół średniej. Jest~to~również jedna z~mniejszych wartości odchylenia standardowego, biorąc pod~uwagę wartości uzyskiwane przez~inne metody mutacji.\\
Strategia DE/current to~best/$n_{v}+1$ nie~uwzględniająca zależności opisanych w~podrozdziale 4.2.4, a~więc bez wprowadzenia dodatkowego współczynnika $\leftthreetimes$, także przynosi satysfakcjonujące wyniki. Średni błąd względny wynosi 0,45\%, a~algorytm 2 razy doprowadza funkcję celu do~wartości uznawanej przez~\cite{qaplib} za~najmniejszą znaną wartość. W~tabeli \ref{ranking3} został przedstawiony ranking metod mutacji ułożony w~zależności od~uzyskanej przez~metody te~średniej wartości błędu względnego.

\begin{table}[h!]
\begin{center}
\caption{Ranking metod mutacji na podstawie średniej wartości błędu względnego funkcji celu, metoda rankingowa, 20 000 iteracji.}
\scalebox{0.8}{
\begin{tabular}{|c|c|c|c|}
\hline
\textbf{Miejsce}  &\textbf{Metoda}  & \textbf{Średni błąd względny funkcji celu} & \textbf{Liczba osiągnietych wartośći 0\%}\\\hline
 \textbf{1}&DE/current to~best/$n_{v}+1$+$\leftthreetimes$&0,44\%& 4\\\hline
 \textbf{2}& DE/current to~best/$n_{v}+1$&0,45\%& 2\\\hline
 \textbf{3}&DE/rand/$n_{v}$+$\leftthreetimes$&0,46\%& 0\\\hline
 \textbf{4}&DE/rand/1+$\leftthreetimes$&0,46\%& 1\\\hline
 \textbf{5}&DE/rand/$n_{v}$&0,75\%&0\\\hline
 \textbf{6}&DE/rand/1&0,77\%& 1\\\hline
 \textbf{7}&DE/best/1+$\leftthreetimes$&1,32\%& 0\\\hline
 \textbf{8}&DE/best/1&1,48\%& 0\\\hline
\end{tabular}}
\label{ranking6}
\end{center}
\end{table}

Na~podstawie tabeli \ref{ranking3} można wysnuć wniosek, iż~metody takie jak~ DE/current to~best/$n_{v}+1$+$\leftthreetimes$,  DE/rand/$n_{v}$+$\leftthreetimes$,  DE/rand/1+$\leftthreetimes$ oraz~DE/best/1+$\leftthreetimes$, a~wiec metody, które~zostały zmodyfikowane współczynnikiem $\leftthreetimes$ uzyskują korzystniejsze wyniki w~porównaniu do~ich odpowiedników bez modyfikatora. Widoczna staje się~zatem skuteczność działania czynnika skalującego $\leftthreetimes$.\\

\item  \textbf{50 000 iteracji }

Poniższe testy zostały przeprowadzone w~celu zbadania czy~uzyskane wyniki utrzymają się~na~poziomie, który~osiągają na~etapie 20 000 iteracji, czy~też następuje poprawa w~sytuacji zwiększenia liczby iteracji. Wyniki testów zestawiono w~tabeli \ref{rankingowa50}.\\

\begin{table}[h!]
\begin{center}
\caption{Wartości błędu względnego funkcji celu dla poszczególnych metod mutacji, metoda rankingowa, 50 000 iteracji.}
\scalebox{0.55}{
\begin{tabular}{|c|c|c|c|c|c|c|c|c|}
\hline
\textbf{Iteracja}  &\textbf{DE/rand/1}  & \textbf{DE/rand/1+$\leftthreetimes$} & \textbf{DE/best/1} & \textbf{DE/best/1+$\leftthreetimes$} & \textbf{DE/rand/$n_{v}$} & \textbf{DE/rand/$n_{v}$+$\leftthreetimes$} & \textbf{ DE/current to best/$n_{v}+1$} & \textbf{ DE/current to best/$n_{v}+1$+$\leftthreetimes$}\\
\hline
\textbf{1}&1,33\%&0,85\%&1,33\%&1,33\%&0,24\%&0,48\%&0,12\%&0,48\% \\ \hline
\textbf{2}&0,85\%&\color{green}\textbf{0}\%&1,21\%&0,97\%&0,61\%&0,12\%&0,61\%&0,48\% \\ \hline
\textbf{3}&0,73\%&1,57\%&1,57\%&0,61\%&0,48\%&0,24\%&0,24\%&0,12\% \\ \hline
\textbf{4}&0,97\%&0,48\%&1,33\%&1,45\%&\color{green}\textbf{0}\%&0,48\%&0,97\%&0,12\% \\ \hline
\textbf{5}&0,48\%&0,24\%&0,61\%&1,45\%&0,61\%&0,24\%&0,24\%&\color{green}\textbf{0}\% \\ \hline
\textbf{6}&0,73\%&0,48\%&1,09\%&2,54\%&0,12\%&\color{green}\textbf{0}\%&\color{green}\textbf{0}\%&0,48\% \\ \hline
\textbf{7}&0,24\%&0,24\%&2,78\%&0,85\%&0,48\%&0,48\%&0,85\%&0,12\% \\ \hline
\textbf{8}&0,48\%&0,48\%&1,09\%&2,18\%&0,48\%&0,24\%&\color{green}\textbf{0}\%&0,24\% \\ \hline
\textbf{9}&0,24\%&0,48\%&2,06\%&0,61\%&0,48\%&0,48\%&0,24\%&0,48\% \\ \hline
\textbf{10}&0,85\%&0,48\%&1,57\%&1,33\%&0,48\%&0,48\%&0,24\%&1,09\% \\ \hline
\textbf{11}&2,18\%&0,48\%&1,94\%&2,18\%&0,61\%&0,24\%&0,24\%&0,12\% \\ \hline
\textbf{12}&\color{green}\textbf{0}\%&0,61\%&2,78\%&1,69\%&0,48\%&\color{green}\textbf{0}\%&0,24\%&0,24\% \\ \hline
\textbf{13}&0,73\%&0,73\%&1,21\%&1,45\%&0,48\%&0,48\%&0,24\%&0,48\% \\ \hline
\textbf{14}&\color{green}\textbf{0}\%&0,24\%&3,03\%&1,09\%&0,48\%&0,73\%&0,24\%&0,24\% \\ \hline
\textbf{15}&0,48\%&\color{green}\textbf{0}\%&1,33\%&1,09\%&0,85\%&\color{green}\textbf{0}\%&0,12\%&0,24\% \\ \hline
\textbf{16}&0,48\%&0,61\%&1,09\%&1,45\%&\color{green}\textbf{0}\%&0,24\%&0,48\%&0,24\% \\ \hline
\textbf{17}&0,73\%&0,24\%&1,94\%&1,69\%&0,61\%&0,48\%&0,48\%&\color{green}\textbf{0}\% \\ \hline
\textbf{18}&0,61\%&1,09\%&1,82\%&1,57\%&0,48\%&0,61\%&0,61\%&0,24\% \\ \hline
\textbf{19}&0,73\%&0,48\%&1,09\%&1,82\%&0,12\%&0,12\%&\color{green}\textbf{0}\%&0,24\% \\ \hline
\textbf{20}&0,61\%&0,73\%&2,66\%&1,69\%&\color{green}\textbf{0}\%&\color{green}\textbf{0}\%&0,48\%&0,24\% \\ \hline
\textbf{ŚREDNIA}&0,67\%&0,53\%&1,68\%&1,45\%&0,41\%&0,31\%&0,33\%&0,30\% \\ \hline
\textbf{ODCHYLENIE}&0,46\%&0,35\%&0,67\%&0,49\%&0,23\%&0,22\%&0,26\%&0,24\% \\ \hline
\end{tabular}}
\label{rankingowa50}
\end{center}
\end{table}
\par

Wartości błędu względnego dla~danych metod uległy poprawie w~stosunku do~ograniczenia liczby iteracji do~20 000. Wynika to~z~faktu, iż~po przekroczeniu tej granicy, w~grupach rodzicielskich nadal występują osobniki o~różnym od~siebie genotypie. Zjawisko to~prowadzi do~polepszania się~wartości funkcji dopasowania. Analizując przebiegi \ref{rankingowa50pic} można zauważyć, iż~stan stabilny utrzymuje się~dopiero na~poziomie 30 000 iteracji, co widoczne jest~na~poniższym wykresie.

\begin{figure}[h!]
		\includegraphics[scale=0.6]{../../../../Screeny/50000_2metody_1652.png}
		\caption{Przebieg działania algorytmu dla~poszczególnych metod mutacji.}
		\label{rankingowa50pic}			
\end{figure}

Poniżej zamieszczono zestawienie funkcji uporządkowanych w~kolejności zależnej od~uzyskanej przez~metody te~wartości błedu względnego funkcji celu.

\begin{table}[h!]
\begin{center}
\caption{Ranking metod mutacji na podstawie średniej wartości błędu względnego funkcji celu, metoda rankingowa, 50 000 iteracji.}
\scalebox{0.8}{
\begin{tabular}{|c|c|c|c|}
\hline
\textbf{Miejsce} &\textbf{Metoda}  & \textbf{Średni błąd względny funkcji celu} & \textbf{Liczba osiągnietych wartośći 0\%}\\\hline
 \textbf{1}& DE/current to~best/$n_{v}+1$+$\leftthreetimes$&0,3\%& 2\\\hline
 \textbf{2}& DE/rand/$n_{v}$+$\leftthreetimes$&0,31\%& 4\\\hline
 \textbf{3}& DE/current to~best/$n_{v}+1$&0,33\%& 3\\\hline
 \textbf{4}& DE/rand/$n_{v}$&0,41\%& 3\\\hline
 \textbf{5}& DE/rand/1+$\leftthreetimes$&0,53\%&2\\\hline
 \textbf{6}& DE/rand/1&0,67\%&2\\\hline
 \textbf{7}& DE/best/1+$\leftthreetimes$&1,45\%& 0\\\hline
 \textbf{8}& DE/best/1&1,68\%& 0\\\hline
\end{tabular}}
\label{ranking4}
\end{center}
\end{table}

Ranking dla~50 000 iteracji zmienił się~w~stosunku do~rankingu 20 000 iteracji nie~tylko w~kontekście mniejszych wartości błędu względnego dla~niektórych z~funkcji, ale~również w~kontekście pozycji zajmowanych przez~metody w~rankingu. Nadal najlepsze wyniki uzyskiwane są~dla metody DE/current to~best/$n_{v}+1$+$\leftthreetimes$. Błąd względny tej metody zmniejszył się~o~0,14 \% w~stosunku do~jej działania z~użyciem algorytmu, w~którym liczba iteracji była ustalona na~wartość 20 000. Podobne wyniki uzyskują również metody DE/rand/$n_{v}$+$\leftthreetimes$ orazDE/current to~best/$n_{v}+1$, zajmując tym samym odpowiednio drugie oraz~trzecie miejsce w~rankingu. Niezależnie od~zastosowanej metody reprodukcji metody DE/best/1+$\leftthreetimes$ oraz~DE/best/1 odznaczają się~wyjątkowo niekorzystnymi wynikami w~porównaniu z~pozostałymi metodami mutacji. Uzyskiwane przez~metody te~wyniki są~około dwukrotnie gorsze.\\

\end{itemize}

%----------------------------------------------------------------------------------------------------------------------------------------------------------------------------------

\subsection{Metody krzyżowania}\label{crossover}

Metody krzyżowania, o~których mowa w~podrozdziale zostały szczegółowo opisane w~rozdziale 5. Metody krzyżowania zostały przetestowane w~każdym z~podrozdziałów przy założeniu stałych takich jak:
\begin{itemize}
\item macierz odległości,
\item macierz przepływu,
\item metoda reprodukcji,
\item metoda mutacji,
\item wielkość populacji
\end{itemize}
\par
Testy zostały podzielone na~trzy główne bloki. Pierwszy z~nich zakłada uznanie za~metodę reprodukcji klasyczną metodę losową, natomiast za~metodę mutacji strategie DE/rand/1. W~kolejnych dwóch blokach testowych zostały uwzględnione wyniki otrzymane w~wyżej przeprowadzonych testach. Poprzez~zestawienie z~metodami krzyżowania, możliwe było dokonanie sprawdzenia jaka kombinacja metod z~poszczególnych sekcji algorytmu pozwoli na~osiągnięcie najbardziej satysfakcjonujących wyników. Dane początkowe, dla~których przeprowadzono testy, znajdują się~w~rozdziale 7.1.1. Dodatkowo zgodnie z~informacjami zawartymi w~podrozdziale \ref{dostrajaniecr} zostanie przebadany wpływ współczynnika krzyżowania $C_{r}$ na~działanie poszczególnych metod.\\

\subsubsection{Reprodukcja losowa, mutacja DE/rand/1}

W~przypadku metod krzyżowania testy zostaną przeprowadzone, zgodnie z~przyjętym sposobem prowadzenia testów dla~metod selekcji oraz~metod mutacji, dla~20 000 iteracji, a~następnie dla~50 000 iteracji. Metoda losowa oraz~strategia DE/rand/1 stanowi klasyczną wersję algorytmu ewolucji różnicowej, testy z~użyciem tych metod są~przeprowadzane w~celu konfrontacji wyników działania wersji klasycznej z~wynikami działania metod zmutowanych.\\

\begin{itemize}
\item  \textbf{20 000 iteracji}

Testy dla~każdej z~metod krzyżowania zostały przeprowadzone 20 razy, a~następnie na~podstawie uzyskanych wartości została obliczona wartość podstawowych miar statystycznych takich jak~średnia i~odchylenie standardowe. Na~podstawie tych wartości oceniona została skuteczność działania poszczególnych metod. Wyniki uzyskiwane w~kolejnych iteracjach przez~poszczególne metody zostały zestawione w~poniższej tabeli. Zapisane są~one w~formie wartości błędu względnego z~uzyskanego wyniku.\\
\par
\begin{table}[h!]
\begin{center}
\caption{Wartości błędu względnego funkcji celu dla poszczególnych metod krzyżowania, reprodukcja losowa,  mutacja DE/rand/1, 20 000 iteracji.}
\scalebox{0.55}{
\begin{tabular}{|c|c|c|c|c|}
\hline
\textbf{Iteracja}  &\textbf{Dwumianowe}  & \textbf{OX} & \textbf{CX} & \textbf{PMX}\\
\hline
\textbf{1}&2,30\%&1,09\%&4,36\%&10,53\% \\ \hline
\textbf{2}&1,94\%&1,33\%&2,91\%&4,96\% \\ \hline
\textbf{3}&0,24\%&0,61\%&2,78\%&11,86\% \\ \hline
\textbf{4}&1,33\%&1,21\%&2,18\%&11,50\% \\ \hline
\textbf{5}&1,82\%&0,48\%&2,30\%&7,02\% \\ \hline
\textbf{6}&3,15\%&1,09\%&2,42\%&9,32\% \\ \hline
\textbf{7}&1,21\%&1,45\%&2,66\%&8,11\% \\ \hline
\textbf{8}&3,27\%&0,48\%&3,39\%&8,47\% \\ \hline
\textbf{9}&0,48\%&1,45\%&1,82\%&11,14\% \\ \hline
\textbf{10}&2,91\%&0,12\%&2,91\%&8,72\% \\ \hline
\textbf{11}&0,73\%&0,85\%&2,18\%&9,44\% \\ \hline
\textbf{12}&0,61\%&0,48\%&4,84\%&10,05\% \\ \hline
\textbf{13}&0,97\%&0,24\%&5,21\%&8,84\% \\ \hline
\textbf{14}&1,57\%&1,09\%&3,27\%&9,93\% \\ \hline
\textbf{15}&0,85\%&1,57\%&3,51\%&9,56\% \\ \hline
\textbf{16}&1,21\%&2,18\%&2,54\%&5,21\% \\ \hline
\textbf{17}&1,57\%&0,97\%&1,21\%&9,69\% \\ \hline
\textbf{18}&0,24\%&1,09\%&2,91\%&7,14\% \\ \hline
\textbf{19}&0,97\%&0,73\%&2,42\%&8,96\% \\ \hline
\textbf{20}&0,97\%&0,85\%&4,36\%&8,47\% \\ \hline
\textbf{ŚREDNIA}&1,42\%&0,97\%&3,01\%&8,95\% \\ \hline
\textbf{ODCHYLENIE}&0,89\%&0,49\%&1,00\%&1,79\% \\ \hline
\end{tabular}}
\label{klasyczna20}
\end{center}
\end{table}

Zauważalny staje się~fakt, iż~w~przypadku wykonywania się~programu przez~20 000 iteracji, nie~udało się~żadnej ze~strategii osiągnąć wartości błędu względnego równej 0\%. Spośród dostępnych metod najlepsze rezultaty uzyskuje metoda OX, jako że~wartość błędu jest~równa 0,97 \%. Wynik działania strategii krzyżowania dwumianowego znajduje się~w~niewielkim otoczeniu od~wyniku uzyskanego przez~krzyżowanie dwumianowe. Z~koleii metody CX oraz~PMX odznaczają się~wyjątkowo dużą wartością błędu względnego, rzędu (3,9 \%). Uporządkowany układ od~najmniejszej do~największej wartości średniego błędu względnego został przedstawiony poniżej.

\begin{table}[h!]
\begin{center}
\caption{Ranking metod krzyżowania na podstawie średniej wartości błędu względnego funkcji celu, reprodukcja losowa,  mutacja DE/rand/1, 20 000 iteracji.}
\scalebox{0.8}{
\begin{tabular}{|c|c|c|c|}
\hline
\textbf{Miejsce}  &\textbf{Metoda}  & \textbf{Średni błąd względny funkcji celu} & \textbf{Liczba osiągnietych wartośći 0\%}\\\hline
 \textbf{1}&OX&0,97\%& 0\\\hline
 \textbf{2}&Dwumianowe&1,42\%& 0\\\hline
 \textbf{3}&CX&3,01\%& 0\\\hline
 \textbf{4}&PMX&8,95\%&0\\\hline
\end{tabular}}
\label{rankingcross1}
\end{center}
\end{table}

\item \textbf{50 000 iteracji}
Niezadowalające efekty działania algorytmu uwzględniającego w~swoim działaniu metody krzyżowania CX czy~też PMX mogą poniekąd wynikać z~niewystarczająco dużej liczby iteracji algorytmu. Wraz~ze~zwiększeniem ilości iteracji istnieją większe szanse na~osiąganie przez~metody te~lepszych wyników.
\begin{table}[h!]
\begin{center}
\caption{Wartości błędu względnego funkcji celu dla poszczególnych metod krzyżowania, reprodukcja losowa,  mutacja DE/rand/1, 50 000 iteracji.}
\scalebox{0.55}{
\begin{tabular}{|c|c|c|c|c|}
\hline
\textbf{Iteracja}  &\textbf{Dwumianowe}  & \textbf{OX} & \textbf{CX} & \textbf{PMX}\\
\hline
\textbf{1}&0,85\%&2,91\%&2,18\%&5,57\% \\ \hline
\textbf{2}&1,33\%&0,48\%&4,24\%&9,20\% \\ \hline
\textbf{3}&1,94\%&0,24\%&3,63\%&8,96\% \\ \hline
\textbf{4}&0,61\%&0,24\%&4,12\%&5,45\% \\ \hline
\textbf{5}&1,21\%&\color{green}\textbf{0}\%&1,33\%&10,05\% \\ \hline
\textbf{6}&0,73\%&0,48\%&1,69\%&11,74\% \\ \hline
\textbf{7}&0,24\%&1,57\%&4,84\%&6,30\% \\ \hline
\textbf{8}&2,66\%&0,24\%&1,94\%&6,90\% \\ \hline
\textbf{9}&0,97\%&1,45\%&3,15\%&9,81\% \\ \hline
\textbf{10}&1,57\%&\color{green}\textbf{0}\%&0,61\%&10,90\% \\ \hline
\textbf{11}&0,85\%&\color{green}\textbf{0}\%&3,39\%&11,99\% \\ \hline
\textbf{12}&1,57\%&\color{green}\textbf{0}\%&2,30\%&8,96\% \\ \hline
\textbf{13}&1,09\%&0,12\%&2,30\%&9,44\% \\ \hline
\textbf{14}&2,18\%&\color{green}\textbf{0}\%&4,84\%&5,21\% \\ \hline
\textbf{15}&0,97\%&0,24\%&1,82\%&8,96\% \\ \hline
\textbf{16}&1,09\%&0,85\%&3,51\%&10,77\% \\ \hline
\textbf{17}&0,85\%&0,12\%&2,42\%&2,18\% \\ \hline
\textbf{18}&1,45\%&0,48\%&6,05\%&8,72\% \\ \hline
\textbf{19}&\color{green}\textbf{0}\%&0,24\%&2,91\%&5,81\% \\ \hline
\textbf{20}&0,61\%&0,12\%&2,66\%&5,69\% \\ \hline
\textbf{ŚREDNIA}&1,14\%&0,49\%&3,00\%&8,13\% \\ \hline
\textbf{ODCHYLENIE}&0,62\%&0,71\%&1,32\%&2,53\% \\ \hline
\end{tabular}}
\label{klasyczna50}
\end{center}
\end{table}

Widoczne staje się~osiąganie przez~metodę OX wartości błędu względnego równą 0\% w~5 z~20 iteracji. W~pozostałych 15 iteracjach uzyskiwane wartości są~w~większości przypadków również niewielkie. Strategia ta osiąga ponad dwukrotnie lepsze wyniki niż jej odpowiednik w~postaci metody dwumianowej i~ponad sześciu i~szesnastokrotnie lepsze w~stosunku do~metod, odpowiednio CX i~PMX. Najmniejsze odchylenie wartości od~średniej zauważalne jest~dla~metody krzyżowania dwumianowego, jest~ono równe 0,62\%. Nie~mniej jednak strategia OX osiąga niewiele większą wartość odchylenia standardowego, 0,71 \%.

\begin{table}[h!]
\begin{center}
\caption{Ranking metod krzyżowania na podstawie średniej wartości błędu względnego funkcji celu, reprodukcja losowa,  mutacja DE/rand/1, 50 000 iteracji.}
\scalebox{0.8}{
\begin{tabular}{|c|c|c|c|}
\hline
\textbf{Miejsce}  &\textbf{Metoda}  & \textbf{Średni błąd względny funkcji celu} & \textbf{Liczba osiągnietych wartośći 0\%}\\\hline
 \textbf{1}&OX&0,49\%& 5\\\hline
 \textbf{2}&Dwumianowe&1,14\%& 1\\\hline
 \textbf{3}&CX&3\%& 0\\\hline
 \textbf{4}&PMX&8,13\%&0\\\hline
\end{tabular}}
\label{rankingcross2}
\end{center}
\end{table}



\end{itemize}

\subsubsection{Reprodukcja losowa, DE/current to best/$n_{v}+1$+$\leftthreetimes$}

Kolejny zestaw testów zakłada użycie metody losowej, jako metody reprodukcji, natomiast za~metodę mutacji przyjmuje się~strategię DE/current to~best/$n_{v}+1$+$\leftthreetimes$. Kombinacja dobrana w~ten sposób wynika z~faktu, iż~właśnie dla~takiego połączenia metod, algorytm uzyskiwał najlepsze wyniki. Przy założeniu takiego połączenia zostaną przetestowane rozważane w~ramach niniejszej pracy metody krzyżowania.

\begin{itemize}
\item \textbf{20 000 iteracji}

W~poniższej tabeli \ref{srodkowa20} znajdują się~wartości błędu względnego funkcji celu dla~algorytmów zakładających w~swoim działaniu użycie poszczególnych metod krzyżowania. Liczba iteracji algorytmu dla~pojedynczej instancji testowej jest~równa 20 000.

\begin{table}[h!]
\begin{center}
\caption{Wartości błędu względnego funkcji celu dla poszczególnych metod krzyżowania, reprodukcja losowa,  mutacja  DE/current to best/$n_{v}+1$+$\leftthreetimes$, 20 000 iteracji.}
\scalebox{0.55}{
\begin{tabular}{|c|c|c|c|c|}
\hline
\textbf{Iteracja}  &\textbf{Dwumianowe}  & \textbf{OX} & \textbf{CX} & \textbf{PMX}\\
\hline
\textbf{1}&0,61\%&\color{green}\textbf{0}\%&1,33\%&7,99\% \\ \hline
\textbf{2}&0,48\%&\color{green}\textbf{0}\%&0,73\%&7,87\% \\ \hline
\textbf{3}&0,48\%&\color{green}\textbf{0}\%&1,94\%&7,26\% \\ \hline
\textbf{4}&0,24\%&1,33\%&0,61\%&8,84\% \\ \hline
\textbf{5}&0,48\%&\color{green}\textbf{0}\%&1,57\%&8,35\% \\ \hline
\textbf{6}&0,85\%&\color{green}\textbf{0}\%&1,09\%&8,47\% \\ \hline
\textbf{7}&0,24\%&0,48\%&1,45\%&8,84\% \\ \hline
\textbf{8}&0,12\%&0,12\%&\color{green}\textbf{0}\%&8,84\% \\ \hline
\textbf{9}&0,48\%&0,24\%&1,45\%&8,11\% \\ \hline
\textbf{10}&0,24\%&\color{green}\textbf{0}\%&0,48\%&6,17\% \\ \hline
\textbf{11}&0,48\%&0,48\%&1,33\%&7,38\% \\ \hline
\textbf{12}&0,48\%&0,12\%&0,61\%&8,96\% \\ \hline
\textbf{13}&0,24\%&0,73\%&1,57\%&10,05\% \\ \hline
\textbf{14}&\color{green}\textbf{0}\%&0,24\%&2,18\%&5,81\% \\ \hline
\textbf{15}&0,48\%&0,12\%&\color{green}\textbf{0}\%&6,05\% \\ \hline
\textbf{16}&0,85\%&\color{green}\textbf{0}\%&0,12\%&5,93\% \\ \hline
\textbf{17}&0,61\%&\color{green}\textbf{0}\%&0,85\%&7,99\% \\ \hline
\textbf{18}&0,24\%&\color{green}\textbf{0}\%&1,82\%&5,08\% \\ \hline
\textbf{19}&0,24\%&\color{green}\textbf{0}\%&1,45\%&3,87\% \\ \hline
\textbf{20}&0,48\%&\color{green}\textbf{0}\%&1,09\%&7,26\% \\ \hline
\textbf{ŚREDNIA}&0,42\%&0,19\%&1,08\%&7,46\% \\ \hline
\textbf{ODCHYLENIE}&0,21\%&0,33\%&0,62\%&1,50\% \\ \hline
\end{tabular}}
\label{srodkowa20}
\end{center}
\end{table}

Tabela \ref{srodkowa20} odznacza się~wielokrotnym powtarzaniem się~wartości błędu względnego równego 0\%. Szczególnie wiele razy wartości uznawane przez~\cite{qaplib} za~najmniejsze jakie udało się~uzyskać dla~obranych danych wejściowych, są~otrzymywane przez~metodę OX. Wartość błędu wynosi 0,19 \%. Wynik ten znacznie odbiega od~wartości uzyskiwanych przez~pozostałe metody, szczególnie przez~metody CX oraz~PMX.

\begin{table}[h!]
\begin{center}
\caption{Ranking metod krzyżowania na podstawie średniej wartości błędu względnego funkcji celu, reprodukcja losowa,  mutacja  DE/current to best/$n_{v}+1$+$\leftthreetimes$, 20 000 iteracji.}
\scalebox{0.8}{
\begin{tabular}{|c|c|c|c|}
\hline
\textbf{Miejsce}  &\textbf{Metoda}  & \textbf{Średni błąd względny funkcji celu} & \textbf{Liczba osiągnietych wartośći 0\%}\\\hline
 \textbf{1}&OX&0,19\%& 11\\\hline
 \textbf{2}&Dwumianowe&0,42\%& 1\\\hline
 \textbf{3}&CX&1,08\%& 2\\\hline
 \textbf{4}&PMX&7,46\%&0\\\hline
\end{tabular}}
\label{rankingcross3}
\end{center}
\end{table}

\item \textbf{50 000 iteracji}

Istnieje możliwość, iż~przy zwiększeniu ilości iteracji, wartości uzyskiwane przez~poszczególne metody ulegną zmniejszeniu. W~tym celu testy zostają powtórzone z~liczbą iteracji równą 50 000.

\begin{table}[h!]
\begin{center}
\caption{Wartości błędu względnego funkcji celu dla poszczególnych metod krzyżowania, reprodukcja losowa,  mutacja  DE/current to best/$n_{v}+1$+$\leftthreetimes$, 50 000 iteracji.}
\scalebox{0.55}{
\begin{tabular}{|c|c|c|c|c|}
\hline
\textbf{Iteracja}  &\textbf{Dwumianowe}  & \textbf{OX} & \textbf{CX} & \textbf{PMX}\\
\hline
\textbf{1}&0,48\%&\color{green}\textbf{0}\%&1,33\%&8,60\% \\ \hline
\textbf{2}&0,24\%&\color{green}\textbf{0}\%&0,73\%&4,96\% \\ \hline
\textbf{3}&\color{green}\textbf{0}\%&\color{green}\textbf{0}\%&1,09\%&9,93\% \\ \hline
\textbf{4}&0,48\%&0,12\%&0,85\%&8,23\% \\ \hline
\textbf{5}&0,61\%&\color{green}\textbf{0}\%&1,09\%&6,54\% \\ \hline
\textbf{6}&\color{green}\textbf{0}\%&\color{green}\textbf{0}\%&0,85\%&9,81\% \\ \hline
\textbf{7}&0,12\%&\color{green}\textbf{0}\%&0,48\%&9,32\% \\ \hline
\textbf{8}&0,12\%&\color{green}\textbf{0}\%&1,09\%&6,05\% \\ \hline
\textbf{9}&0,48\%&0,24\%&0,97\%&6,54\% \\ \hline
\textbf{10}&0,24\%&\color{green}\textbf{0}\%&1,09\%&9,20\% \\ \hline
\textbf{11}&0,12\%&\color{green}\textbf{0}\%&0,85\%&6,05\% \\ \hline
\textbf{12}&0,24\%&\color{green}\textbf{0}\%&0,12\%&7,14\% \\ \hline
\textbf{13}&\color{green}\textbf{0}\%&\color{green}\textbf{0}\%&0,85\%&7,99\% \\ \hline
\textbf{14}&0,48\%&0,12\%&1,69\%&8,84\% \\ \hline
\textbf{15}&\color{green}\textbf{0}\%&\color{green}\textbf{0}\%&0,73\%&8,23\% \\ \hline
\textbf{16}&0,12\%&0,24\%&0,12\%&7,75\% \\ \hline
\textbf{17}&0,48\%&\color{green}\textbf{0}\%&0,61\%&3,63\% \\ \hline
\textbf{18}&0,48\%&\color{green}\textbf{0}\%&1,21\%&7,51\% \\ \hline
\textbf{19}&0,48\%&\color{green}\textbf{0}\%&1,21\%&10,90\% \\ \hline
\textbf{20}&\color{green}\textbf{0}\%&0,48\%&0,85\%&6,42\% \\ \hline
\textbf{ŚREDNIA}&0,26\%&0,06\%&0,89\%&7,68\% \\ \hline
\textbf{ODCHYLENIE}&0,21\%&0,12\%&0,37\%&1,75\% \\ \hline
\end{tabular}}
\label{srodkowa50}
\end{center}
\end{table}

Na~podstawie tabeli \ref{srodkowa50} zauważalne staje się~wyjątkowo korzystne działanie metody krzyżowania OX, jako że~średnia wartość błędu względnego jest~równa 0,06 \%. W~15 na~20 prób wartość błędu względnego jest~równa 0\%, w~pozostałych 5 próbach nie~przekracza ona wartości 0,48 \%. Również w~kontekście odchylenia standardowego uzyskana wartość jest~niewielka i~wynosi 0,12\%, co świadczy o~dużym skupieniu wyników wokół wartości średniej.

\begin{table}[h!]
\begin{center}
\caption{Ranking metod krzyżowania na podstawie średniej wartości błędu względnego funkcji celu, reprodukcja losowa,  mutacja  DE/current to best/$n_{v}+1$+$\leftthreetimes$, 50 000 iteracji.}
\scalebox{0.8}{
\begin{tabular}{|c|c|c|c|}
\hline
\textbf{Miejsce}  &\textbf{Metoda}  & \textbf{Średni błąd względny funkcji celu} & \textbf{Liczba osiągnietych wartośći 0\%}\\\hline
 \textbf{1}&OX&0,06\%& 15\\\hline
 \textbf{2}&Dwumianowe&0,26\%& 5\\\hline
 \textbf{3}&CX&0,89\%& 0\\\hline
 \textbf{4}&PMX&7,68\%&0\\\hline
\end{tabular}}
\label{rankingcross4}
\end{center}
\end{table}

Działanie metody krzyżowania dwumianowego oraz~krzyżowania CX również okazało się~przynosić lepsze efekty przy zwiększeniu liczby iteracji do~50 000. Natomiast metoda krzyżowania PMX utrzymuje wyniki na~tym samym poziomie. Oznacza to,~że~algorytm z~zastosowaniem tej metody jako metody krzyżowania na~poziomie 20 000 iteracji zachowuje już stan stabilny, a~więc funkcja celu nie~ulega już dalszemu zmniejszaniu.

\end{itemize}

\subsubsection{Reprodukcja rankingowa, DE/current to best/$n_{v}+1$+$\leftthreetimes$}

W~przeprowadzonych testach dobrymi wynikami wyróżniała się~koncepcja algorytmu, w~której metodą reprodukcji była metoda rankingowa, a~metodą mutacji startegia DE/current to~best/$n_{v}+1$+$\leftthreetimes$. Wyniki tej kombinacji były porównywalne z~wynikami koncepcji opisanej w~poprzednim podrozdziale, a~mianowicie połączenia metody losowej z~metodą DE/current to~best/$n_{v}+1$+$\leftthreetimes$. Z~tego względu w~poniższym podrozdziale zostaną przeprowadzone testy z~metodą rankingową jako metodą reprodukcji.

\begin{itemize}
\item \textbf{20 000 iteracji}

\begin{table}[h]
\begin{center}
\caption{Wartości błędu względnego funkcji celu dla poszczególnych metod krzyżowania, reprodukcja rankingowa, mutacja DE/current to best/$n_{v}+1$+$\leftthreetimes$, 20 000 iteracji.}
\scalebox{0.55}{
\begin{tabular}{|c|c|c|c|c|}
\hline
\textbf{Iteracja}  &\textbf{Dwumianowe}  & \textbf{OX} & \textbf{CX} & \textbf{PMX}\\
\hline
\textbf{1}&0,48\%&0,12\%&1,33\%&7,75\% \\ \hline
\textbf{2}&\color{green}\textbf{0}\%&\color{green}\textbf{0}\%&1,45\%&7,75\% \\ \hline
\textbf{3}&0,24\%&0,61\%&0,73\%&5,45\% \\ \hline
\textbf{4}&0,48\%&0,12\%&1,45\%&9,32\% \\ \hline
\textbf{5}&0,12\%&0,12\%&1,21\%&9,81\% \\ \hline
\textbf{6}&0,12\%&0,48\%&0,85\%&6,42\% \\ \hline
\textbf{7}&1,09\%&0,73\%&\color{green}\textbf{0}\%&8,96\% \\ \hline
\textbf{8}&0,12\%&0,48\%&1,45\%&6,30\% \\ \hline
\textbf{9}&0,24\%&\color{green}\textbf{0}\%&0,48\%&7,14\% \\ \hline
\textbf{10}&\color{green}\textbf{0}\%&\color{green}\textbf{0}\%&1,69\%&10,29\% \\ \hline
\textbf{11}&0,12\%&0,48\%&1,45\%&7,99\% \\ \hline
\textbf{12}&0,12\%&\color{green}\textbf{0}\%&1,09\%&6,17\% \\ \hline
\textbf{13}&0,48\%&0,12\%&0,24\%&9,69\% \\ \hline
\textbf{14}&0,85\%&\color{green}\textbf{0}\%&0,48\%&7,14\% \\ \hline
\textbf{15}&0,24\%&\color{green}\textbf{0}\%&1,09\%&7,63\% \\ \hline
\textbf{16}&0,24\%&0,24\%&0,73\%&6,30\% \\ \hline
\textbf{17}&0,24\%&\color{green}\textbf{0}\%&1,21\%&6,17\% \\ \hline
\textbf{18}&0,24\%&\color{green}\textbf{0}\%&1,09\%&5,57\% \\ \hline
\textbf{19}&0,61\%&0,12\%&0,97\%&7,63\% \\ \hline
\textbf{20}&0,48\%&\color{green}\textbf{0}\%&1,33\%&8,47\% \\ \hline
\textbf{ŚREDNIA}&0,33\%&0,18\%&1,02\%&7,60\% \\ \hline
\textbf{ODCHYLENIE}&0,27\%&0,23\%&0,44\%&1,42\% \\ \hline
\end{tabular}}
\label{rankingowacross20}
\end{center}
\end{table}

Uzyskane i~zestawione w~tabeli \ref{rankingowacross20} wartości błędu względnego przyjmują wartości bliskie wartościom uzyskiwanym przez~strategie, w~której to~metodą reprodukcji jest~metoda losowa. Największa różnica w~wartości błędu pomiędzy tymi kompozycjami metod wynosi 0,14 \% i~występuje ona w~przypadku krzyżowania PMX. Krzyżowanie to~odznacza się~stosunkowo dużym odchyleniem standardowym o~wartości 1,5 \%. W~przypadku innych metod różnica w~działaniu jest~znacznie mniejsza.

\begin{table}[h]
\begin{center}
\caption{Ranking metod krzyżowania na podstawie średniej wartości błędu względnego funkcji celu, reprodukcja rankingowa,  mutacja  DE/current to best/$n_{v}+1$+$\leftthreetimes$, 20 000 iteracji.}
\scalebox{0.8}{
\begin{tabular}{|c|c|c|c|}
\hline
\textbf{Miejsce}  &\textbf{Metoda}  & \textbf{Średni błąd względny funkcji celu} & \textbf{Liczba osiągnietych wartośći 0\%}\\\hline
 \textbf{1}&OX&0,18\%& 9\\\hline
 \textbf{2}&Dwumianowe&0,33\%& 2\\\hline
 \textbf{3}&CX&1,02\%& 1\\\hline
 \textbf{4}&PMX&7,6\%&0\\\hline
\end{tabular}}
\label{rankingcross5}
\end{center}
\end{table}

\item \textbf{50 000 iteracji}

\begin{table}[!]
\begin{center}
\caption{Wartości błędu względnego funkcji celu dla poszczególnych metod krzyżowania, reprodukcja rankingowa,  mutacja  DE/current to best/$n_{v}+1$+$\leftthreetimes$, 50 000 iteracji.}
\scalebox{0.55}{
\begin{tabular}{|c|c|c|c|c|}
\hline
\textbf{Iteracja}  &\textbf{Dwumianowe}  & \textbf{OX} & \textbf{CX} & \textbf{PMX}\\
\hline
\textbf{1}&0,48\%&\color{green}\textbf{0}\%&1,09\%&6,17\% \\ \hline
\textbf{2}&1,45\%&\color{green}\textbf{0}\%&1,09\%&9,20\% \\ \hline
\textbf{3}&\color{green}\textbf{0}\%&0,48\%&1,33\%&10,05\% \\ \hline
\textbf{4}&0,24\%&\color{green}\textbf{0}\%&1,45\%&8,60\% \\ \hline
\textbf{5}&0,48\%&\color{green}\textbf{0}\%&1,57\%&7,75\% \\ \hline
\textbf{6}&0,48\%&\color{green}\textbf{0}\%&0,73\%&8,84\% \\ \hline
\textbf{7}&0,48\%&\color{green}\textbf{0}\%&0,61\%&10,29\% \\ \hline
\textbf{8}&0,61\%&0,12\%&0,48\%&5,33\% \\ \hline
\textbf{9}&0,61\%&\color{green}\textbf{0}\%&0,85\%&4,96\% \\ \hline
\textbf{10}&0,12\%&\color{green}\textbf{0}\%&0,85\%&7,02\% \\ \hline
\textbf{11}&0,48\%&\color{green}\textbf{0}\%&0,85\%&7,51\% \\ \hline
\textbf{12}&\color{green}\textbf{0}\%&0,12\%&0,85\%&7,99\% \\ \hline
\textbf{13}&0,24\%&\color{green}\textbf{0}\%&0,97\%&7,14\% \\ \hline
\textbf{14}&0,24\%&\color{green}\textbf{0}\%&0,61\%&7,87\% \\ \hline
\textbf{15}&0,85\%&\color{green}\textbf{0}\%&1,45\%&8,84\% \\ \hline
\textbf{16}&0,85\%&0,12\%&0,61\%&9,93\% \\ \hline
\textbf{17}&\color{green}\textbf{0}\%&\color{green}\textbf{0}\%&1,45\%&6,54\% \\ \hline
\textbf{18}&\color{green}\textbf{0}\%&0,48\%&0,73\%&10,17\% \\ \hline
\textbf{19}&\color{green}\textbf{0}\%&0,61\%&\color{green}\textbf{0}\%&8,23\% \\ \hline
\textbf{20}&0,61\%&\color{green}\textbf{0}\%&0,12\%&7,63\% \\ \hline
\textbf{ŚREDNIA}&0,41\%&0,10\%&0,88\%&8,00\% \\ \hline
\textbf{ODCHYLENIE}&0,36\%&0,19\%&0,42\%&1,50\% \\ \hline
\end{tabular}}
\label{srodkowa50}
\end{center}
\end{table}

W~przypadku gdy liczba iteracji jest~równa 50 000 uzyskiwane wyniki dla~strategii algorytmu, w~której uwzględnimy metodę rankingową, są~nieco gorsze niż te~zestawione w~tabeli \ref{srodkowa50}. Różnice są~jednak niewielkie, a~średnia wartość błędu względnego uzyskana przy użyciu metody OX jest~równa 0,1 \% co jest~drugą najlepszą wartością błędu względnego uzyskaną we wszystkich przeprowadzanych testach. Podłożem różnic występujących pomiędzy dwoma strategiami, opisanymi odpowiednio w~podrozdziale 6.1.3.2 oraz~6.1.3.3 może być~działanie czynnika losowego.

\begin{table}[h]
\begin{center}
\caption{Ranking metod krzyżowania na podstawie średniej wartości błędu względnego funkcji celu, reprodukcja rankingowa,  mutacja  DE/current to best/$n_{v}+1$+$\leftthreetimes$, 50 000 iteracji.}
\scalebox{0.8}{
\begin{tabular}{|c|c|c|c|}
\hline
\textbf{Miejsce}  &\textbf{Metoda}  & \textbf{Średni błąd względny funkcji celu} & \textbf{Liczba osiągnietych wartośći 0\%}\\\hline
 \textbf{1}&OX&0,1\%& 14\\\hline
 \textbf{2}&Dwumianowe&0,41\%& 5\\\hline
 \textbf{3}&CX&0,88\%& 1\\\hline
 \textbf{4}&PMX&8\%&0\\\hline
\end{tabular}}
\label{rankingcross4}
\end{center}
\end{table}

\end{itemize}

%--------------------------------------------------------------------------------------------------------------------------------------------

\subsection{Modyfikacje parametrów operatorów genetycznych}

Testy współczynników $F$ oraz~$C_{r}$ zostały przeprowadzone z~wykorzystaniem kilku wariantów zmodyfikowanego algorytmu ewolucji różnicowej, dostosowanego do~rozwiązywania problemu kwadratowego zagadnienia przydziału. Wybrane zostały te~strategie, które~na~podstawie uprzednio przeprowadzonych testów, opisanych szczegółowo w~rozdziałach \ref{reprodukcja}, \ref{mutacja} oraz~\ref{crossover}, uzyskały najlepsze wyniki.\\
Testy zostały wykonane dla~50 000 iteracji algorytmu, a~następnie powtórzone 20 razy. W~wyniku tego działania możliwe było policzenie średniego błędu względnego z~20 iteracji. Wartości te~zostały umieszczone w~odpowiednich tabelach. Dodatkowo tabele, w~celu konfrontacji wyników, zawierają informacje o~średniej wartości błędu względnego dla~wariantu algorytmu, w~którym współczynnik $F$ jest~ustalony na~stałym poziomie i~wynosi $F = 0,8$.

\subsubsection{Modyfikacje parametru $F$}\label{modF}

Najmniejsza średnia wartość błędu względnego uzyskiwana była dla~algorytmów wykorzystujących w~swoim działaniu następujące metody:\\
\begin{itemize}
\item metoda reprodukcji losowej, metoda mutacji DE/current to best/$n_{v}+1$+$\leftthreetimes$, metoda krzyżowania OX,\\
\item metoda reprodukcji rankingowej, metoda mutacji DE/current to best/$n_{v}+1$+$\leftthreetimes$, metoda krzyżowania OX,\\
\end{itemize}
Dynamiczna zmiana czynnika skalującego zostanie przetestowana w~kontekście działania wszystkich metod mutacji, by~zobaczyć wpływ działania tego współczynnika na~każdą z~nich. Jako że~w~rozdziale \ref{rozklad} zaprezentowane zostały dwie metody dynamicznej zmiany parametru $F$, poniższe testy również zostały podzielone w~taki sposób.
\begin{itemize}
\item \textbf{Modyfikacja parametru $F$ zgodnie z funkcją określoną w podrozdziale \ref{funkcja}}\\

\begin{table}[h!]
\begin{center}
\caption{Wartości średniego błędu względnego funkcji celu dla poszczególnych metod mutacji ze zmiennym parametrem F zgodnie z funkcją \ref{funkcja}.}
\scalebox{0.45}{
\begin{tabular}{|c|c|c|c|c|c|c|c|c|}
\hline
\textbf{Wersja algorytmu}  &\textbf{DE/rand/1}  & \textbf{DE/rand/1+$\leftthreetimes$} & \textbf{DE/best/1} & \textbf{DE/best/1+$\leftthreetimes$} & \textbf{DE/rand/$n_{v}$} & \textbf{DE/rand/$n_{v}$+$\leftthreetimes$} & \textbf{ DE/current to best/$n_{v}+1$} & \textbf{ DE/current to best/$n_{v}+1$+$\leftthreetimes$}\\
\hline
\textbf{Reprodukcja losowa,krzyżowanie OX, F = 0,8}&0,77\%&0,33\%&1,10\%&0,49\%&0,62\%&0,43\%&0,19\%&0,10\% \\ \hline
\textbf{Reprodukcja losowa,krzyżowanie OX, F zgodne z \ref{funkcja}}&0,44\%&0,39\%&2,05\%&0,81\%&0,48\%&0,33\%&0,44\%&0,09\% \\ \hline
\textbf{Reprodukcja rankingowa,krzyżowanie OX, F = 0,8}&0,44\%&0,24\%&0,96\%&0,29\%&0,49\%&0,21\%&0,34\%&0,10\% \\ \hline
\textbf{Reprodukcja rankingowa,krzyżowanie OX, F zgodne z \ref{funkcja}}&1,25\%&0,38\%&2,52\%&0,64\%&0,61\%&0,51\%&0,42\%&0,09\% \\ \hline
\end{tabular}}
\label{parametrFfunkcja}
\end{center}
\end{table}

Wersja algorytmu, stosująca reprodukcje losową, krzyżowanie OX oraz~dynamiczną zmianę parametru, dla~metod takich jak~DE/rand/1, DE/rand/$n_{v}$, DE/rand/$n_{v}$+$\leftthreetimes$ oraz~DE/current to~best/$n_{v}$+$\leftthreetimes$ okazała się~przynosić lepsze efekty. Szczególną uwagę należy zwrócić na~zastosowanie mechanizmu dostrajania parametru mutacji w~kontekście metody DE/current to~best/$n_{v}$+$\leftthreetimes$, ponieważ~kombinacja ta pozwala dodatkowo na~osiągnięcie najlepszego wyniku spośród wszystkich testowanych przypadków zamieszonych w~tabeli \ref{parametrFfunkcja}. Biorąc pod~uwagę algorytm, w~którym metodą reprodukcji była metoda rankingowa, wartości funkcji celu dla~każdego rodzaju mutacji były wyższe od~wersji algorytmu z~niezmiennym współczynnikiem $F$ ustawionym na~stałą wartość $F =0,8$. Modyfikacja w~tym przypadku nie~wpływa zatem korzystnie na~działanie algorytmu.\\


\item \textbf{Modyfikacja parametru $F$ w zakresie (0,1)}

W~celu zbadania, dla~jakiej wartości współczynnika $F$ algorytm uzyska najlepsze wyniki, przeprowadzono analizę działania algorytmu dla~różnych wartości parametru $F$. Wartość współczynnika zwiększana była z~każdym wywołaniem algorytmu o~0,05, w~zakresie (0,1). W~ten sposób przetestowanych zostało 20 różnych wartości czynnika skalującego. W~poniższej tabeli \ref{parametrFzmiennny} zaprezentowano średnie wartości błędu względnego obliczone na~podstawie 20 iteracji algorytmu.

\begin{table}[h!]
\begin{center}
\caption{Wartości średniego błędu względnego funkcji celu dla algorytmu ze zmiennym parametrem F w zakresie (0,1).}
\scalebox{0.55}{
\begin{tabular}{|c|c|c|c|}
\hline
\textbf{Wartość współczynnika $F$}  &\textbf{Średni błąd względny funkcji celu}  & \textbf{Wartość współczynnika $F$} & \textbf{Średni błąd względny funkcji celu} \\
\hline
\textbf{0,05}&0,15\%&\textbf{0,55}&0,12\% \\ \hline
\textbf{0,1}&0,07\%&\textbf{0,6 }&0,16\% \\ \hline
\textbf{0,15}&0,08\%&\textbf{0,65 }&0,07\% \\ \hline
\textbf{0,2}&0,13\%&\textbf{0,7 }&0,16\% \\ \hline
\textbf{0,25}&0,04\%&\textbf{0,75 }&0,09\% \\ \hline
\textbf{0,3}&0,18\%&\textbf{0,8}&0,10\% \\ \hline
\textbf{0,35}&0,17\%&\textbf{0,85 }&0,09\% \\ \hline
\textbf{0,4}&0,13\%&\textbf{0,9 }&0,21\% \\ \hline
\textbf{0,45}&0,13\%&\textbf{0,95 }&0,16\% \\ \hline
\textbf{0,5}&0,21\%&\textbf{1}&0,18\% \\ \hline
\end{tabular}}
\label{parametrFzmiennny}
\end{center}
\end{table}

Zauważalne staje się~osiąganie najniższych średnich wartości funkcji celu (0,04\%) przez~wersję algorytmu, w~której współczynnik mutacji został ustalony na~poziomie $F= 0,25$. Jest~to~jednocześnie najmniejsza wartość tej miary we wszystkich przeprowadzonych testach. Wyniki osiągane przez~pozostałe koncepcje znajdują się~w~zakresie (0, 0,21\%) co świadczy o~skuteczności działania algorytmu, w~którym metodą reprodukcji jest~metoda losowa, metodą mutacji strategia DE/current to~best/$n_{v}+1$+$\leftthreetimes$, natomiast metodę krzyżowania stanowi krzyżowanie OX, x każdą z~wartości przyjętą dla~parametru $F$. W~celu uzyskania pełniejszego obrazu zachodzącego zjawiska, wartości zawarte w~tabeli \ref{parametrFzmiennny} zestawione zostały w~postaci wykresu \ref{wykresF}.

\begin{figure}[h!]
\begin{center}
		\includegraphics[scale=0.6]{../../../../Screeny/parametrFf.png}
		\caption{Zależność współczynnika mutacji $F$ od~uzyskiwanych wartości funkcji celu.}
		\label{wykresF}	
\end{center}		
\end{figure}
\par
\item \textbf{Modyfikacja parametru $F$ zgodnie z rozkładem normalnym, podrozdział \ref{gauss1}}

Wśród uzyskiwanych wyników, zestawionych w~powyżej, nie~występuje trend pozwalający ustalić, na~jakiej zasadzie czynnik skalujący wpływa na~działanie algorytmu. Dostrajanie parametru $F$ zgodnie z~rozkładem normalnym nie~znajduje więc~w~tym przypadku zastosowania.

\end{itemize}

\subsubsection{Modyfikacje parametrów $C_{r}$}

Niewiele gorsze wyniki od~wspomnianych wyżej dwóch strategii uzyskiwały następujące warianty algorytmu:\\
\begin{itemize}
\item metoda reprodukcji losowej, metoda mutacji  DE/current to best/$n_{v}+1$+$\leftthreetimes$, metoda krzyżowania dwumianowego,\\
\item metoda reprodukcji rankingowej, metoda mutacji  DE/current to best/$n_{v}+1$+$\leftthreetimes$, metoda krzyżowania dwumianowego,\\
\end{itemize}
Testy zostaną przeprowadzone z~podziałem w~zależności od~sposobu aktualizacji współczynnika krzyżowania.

\begin{itemize}
\item  \textbf{Modyfikacja parametru $C_{r}$ zgodnie z funkcją określoną w podrozdziale \ref{funkcjacr}}


\begin{table}[h!]
\begin{center}
\caption{Wartości średniego błędu względnego funkcji celu dla metody krzyżowania dwumianowego ze zmiennym parametrem $C_{r}$ zgodnie z funkcją \ref{funkcjacr}.}
\scalebox{0.55}{
\begin{tabular}{|c|c|c|c|c|c|c|c|c|}
\hline
\textbf{Wersja algorytmu}  &\textbf{Dwumianowe} \\
\hline
\textbf{Reprodukcja losowa,mutacja DE/current to best/$n_{v}+1$+$\leftthreetimes$, $C_{r}$ = rand(0,1)}&0,26\% \\ \hline
\textbf{Reprodukcja losowa,mutacja DE/current to best/$n_{v}+1$+$\leftthreetimes$, $C_{r}$ zgodne z \ref{funkcjacr}}&0,1\% \\ \hline
\textbf{Reprodukcja rankingowa,mutacja DE/current to best/$n_{v}+1$+$\leftthreetimes$, $C_{r}$ = rand(0,1)}&0,4\% \\ \hline
\textbf{Reprodukcja rankingowa,mutacja DE/current to best/$n_{v}+1$+$\leftthreetimes$, $C_{r}$ zgodne z \ref{funkcjacr}}&0,11\% \\ \hline
\end{tabular}}
\label{parametrCRfunkcja}
\end{center}
\end{table}

Zgodnie z~tabelą \ref{parametrCRfunkcja} wprowadzenie dynamicznej zmiany parametru $C_{r}$ wpływa korzystnie zarówno na~wersję algorytmu, w~której metodą reprodukcji jest~metoda losowa, jak~i~rankingowa. W~przypadku metody rankingowej modyfikacja parametru zgodnie z~funkcją \ref{funkcja} przynosi około czterokrotnie lepsze efekty. Niemniej jednak w~wyniku testów udało się~uzyskać wersje algorytmu odznaczające się~mniejszą średnią wartością błędu względnego funkcji celu. 
\item \textbf{Modyfikacja parametru $C_{r}$ w zakresie (0,1)}

W~celu zbadania dla~jakiej wartości parametru $C_{r}$ algorytm uzyskuje najmniejszą wartość funkcji celu, przeprowadzono testy dla~20 różnych jego wartości. Początkowo parametr miał wartość $C_{r}$=0,3 a~następnie był sukcesywnie zwiększany o~0,05, aż do~osiągnięcia ostatecznej wartości równej 1. Testy zostały przeprowadzone z~użyciem algorytmu działającego w~oparciu o~metodę reprodukcji losowej, mutacji DE/current to~best/$n_{v}+1$+$\leftthreetimes$ oraz~krzyżowania dwumianowego. Poniżej znajduje się~tabela podsumowująca uzyskane wyniki, przedstawione są~one w~postaci średnich wartości błędu względnego funkcji celu.
\begin{table}[h!]
\begin{center}
\caption{Wartości średniego błędu względnego funkcji celu dla algorytmu ze zmiennym parametrem $C_{r}$ w zakresie (0,1).}
\scalebox{0.55}{
\begin{tabular}{|c|c|c|c|}
\hline
\textbf{Wartość współczynnika $C_{r}$}  &\textbf{Średni błąd względny funkcji celu}  & \textbf{Wartość współczynnika $C_{r}$} & \textbf{Średni błąd względny funkcji celu} \\
\hline
\textbf{0,05}&0,32\%&\textbf{0,55}&0,28\% \\ \hline
\textbf{0,1}&0,31\%&\textbf{0,6 }&0,32\% \\ \hline
\textbf{0,15}&0,08\%&\textbf{0,65 }&0,54\% \\ \hline
\textbf{0,2}&0,1\%&\textbf{0,7 }&0,52\% \\ \hline
\textbf{0,25}&0,05\%&\textbf{0,75 }&0,94\% \\ \hline
\textbf{0,3}&0,17\%&\textbf{0,8}&1,01\% \\ \hline
\textbf{0,35}&0,1\%&\textbf{0,85 }&1,47\% \\ \hline
\textbf{0,4}&0,16\%&\textbf{0,9 }&1,67\% \\ \hline
\textbf{0,45}&0,18\%&\textbf{0,95 }&2,26\% \\ \hline
\textbf{0,5}&0,18\%&\textbf{1}&3,06\% \\ \hline
\end{tabular}}
\label{parametrCrzmiennny}
\end{center}
\end{table}

Zauważalne staje się~osiąganie przez~algorytm najmniejszej średniej wartości funkcji celu (0,05 \%) przy ustaleniu współczynnika na~poziomie $C_{r}$ = 0,25. Uzyskany wynik jest~niewątpliwie lepszy w~porównaniu do~wersji algorytmu, w~której wartość $C_{r}$ była ustalana losowo, gdyż wartość ta wówczas była równa 0,26\%. Podejście to~jest odmienne i~wprowadza modyfikacje w~stosunku do~klasycznej wersji krzyżowania dwumianowego, gdyż wartość ta na~cały czas trwania algorytmu ustalona jest~na~stałym poziomie. Wśród otrzymanych wyników istnieje również pewna tendencja ukazująca charakter pracy algorytmu w~zależności od~wartości współczynnika $C_{r}$, co można zaobserwować analizując wykres \ref{wykresCr}.

\begin{figure}[h!]
\begin{center}
		\includegraphics[scale=0.6]{../../../../Screeny/wykresCr2.png}
		\caption{Zależność współczynnika krzyżowania $C_{r}$ od~uzyskiwanych wartości funkcji celu.}
		\label{wykresCr}	
\end{center}		
\end{figure}

\item \textbf{Modyfikacja parametru $C_{r}$ zgodnie z rozkładem normalnym, podrozdział \ref{gauss2}}

Widoczny na~wykresie \ref{wykresCr} trend kształtuje się~w~sposób przybliżenie eksponencjalny. Istnieje zatem sens badania wpływu na~algorytm dynamicznej zmiany parametru $C_{r}$ zgodnie z~rozkładem normalnym. Dla~wartości współczynnika $C_{r}$ większych niż 0,65 średnie wartości funkcji celu zaczynają przybierać znacznie większe wartości. Na~tej podstawie można ustalić wartości współczynników $\mu$ oraz~$\sigma$. Za~wartość średnią $\mu$ została uznana $\mu$=0,25, natomiast odchylenie standardowe $\sigma$ było równe 0,25.

\begin{table}[h!]
\begin{center}
\caption{Wartości średniego błędu względnego funkcji celu dla metody krzyżowania dwumianowego ze zmiennym parametrem $C_{r}$ zgodnie z rozkładem normalnym.}
\scalebox{0.55}{
\begin{tabular}{|c|c|c|c|c|c|c|c|c|}
\hline
\textbf{Wersja algorytmu}  &\textbf{Dwumianowe} \\
\hline
\textbf{Reprodukcja losowa,mutacja DE/current to best/$n_{v}+1$+$\leftthreetimes$, $C_{r}$ = rand(0,1)}&0,26\% \\ \hline
\textbf{Reprodukcja losowa,mutacja DE/current to best/$n_{v}+1$+$\leftthreetimes$, $C_{r}$ zgodne z \ref{gauss2}}&0,14\% \\ \hline
\textbf{Reprodukcja rankingowa,mutacja DE/current to best/$n_{v}+1$+$\leftthreetimes$, $C_{r}$ = rand(0,1)}&0,4\% \\ \hline
\textbf{Reprodukcja rankingowa,mutacja DE/current to best/$n_{v}+1$+$\leftthreetimes$, $C_{r}$ zgodne z \ref{gauss2}}&0,14\% \\ \hline
\end{tabular}}
\label{parametrFgauss}
\end{center}
\end{table}

Zastosowanie rozkładu normalnego w~celu wyznaczania wartości współczynnika $C_{r}$ wpływa korzystnie na~działanie algorytmu. Niemniej jednak w~przeprowadzonych uprzednio testach istniały wersje algorytmu odznaczające się~skuteczniejszymi wynikami.
\end{itemize}
 
\section{Dane wejściowe instancja II}

Algorytm zostanie przetestowany z~użyciem danych wejściowych, których rozmiar jest~większy od~poprzedniej instancji testowej, wynosi 18. Zarówno macierz przepływu, jak~i~macierz odległości jest~symetryczna. Dane testowe zostały zaczerpnięte z~biblioteki \cite{qaplib}, gdzie również znajdowała się~informacja o~najmniejszej wartości funkcji celu, jaką udało się~uzyskać dla~danej instancji testowej. W~tym przypadku wartość ta jest~równa 5358. Wartość ta traktowana jest~jako wartość dokładna i~stanowi podstawę do~obliczania wartości błędu względnego poszczególnych pomiarów. Poniżej znajdują się~odpowiednio macierz przepływu oraz~macierz odległości.

\par
$$
\mathbf{Macierz\_trasy} =
\left( \begin{array}{cccccccccccccccccc}
0& 1& 2& 2& 3& 4& 4& 5& 3& 5& 6& 7& 8& 9& 7& 8& 4& 5\\
1& 0& 1& 1& 2& 3& 3& 4& 2& 4& 5& 6& 7& 8& 6& 7& 3& 4\\
2& 1& 0& 2& 1& 2& 2& 3& 1& 3& 4& 5& 6& 7& 5& 6& 2& 3\\
2& 1& 2& 0& 1& 2& 2& 3& 3& 3& 4& 5& 6& 7& 5& 6& 4& 5\\
3& 2& 1& 1& 0& 1& 1& 2& 2& 2& 3& 4& 5& 6& 4& 5& 3& 4\\
4& 3& 2& 2& 1& 0& 2& 3& 3& 1& 2& 3& 4& 5& 3& 4& 4& 5\\
4& 3& 2& 2& 1& 2& 0& 1& 3& 1& 2& 3& 4& 5& 3& 4& 4& 5\\
5& 4& 3& 3& 2& 3& 1& 0& 4& 2& 1& 2& 3& 4& 2& 3& 5& 6\\
3& 2& 1& 3& 2& 3& 3& 4& 0& 4& 5& 6& 7& 8& 6& 7& 1& 2\\
5& 4& 3& 3& 2& 1& 1& 2& 4& 0& 1& 2& 3& 4& 2& 3& 5& 6\\
6& 5& 4& 4& 3& 2& 2& 1& 5& 1& 0& 1& 2& 3& 1& 2& 6& 7\\
7& 6& 5& 5& 4& 3& 3& 2& 6& 2& 1& 0& 1& 2& 2& 3& 7& 8\\
8& 7& 6& 6& 5& 4& 4& 3& 7& 3& 2& 1& 0& 1& 1& 2& 8& 9\\
9& 8& 7& 7& 6& 5& 5& 4& 8& 4& 3& 2& 1& 0& 2& 1& 9& 10\\
7& 6& 5& 5& 4& 3& 3& 2& 6& 2& 1& 2& 1& 2& 0& 1& 7& 8\\
8& 7& 6& 6& 5& 4& 4& 3& 7& 3& 2& 3& 2& 1& 1& 0& 8& 9\\
4& 3& 2& 4& 3& 4& 4& 5& 1& 5& 6& 7& 8& 9& 7& 8& 0& 1\\
5& 4& 3& 5& 4& 5& 5& 6& 2& 6& 7& 8& 9& 10& 8& 9& 1& 0\\
\end{array} \right)
$$

\par
$$
\mathbf{Macierz\_kosztu} =
\left( \begin{array}{cccccccccccccccccc}
   0& 3& 4& 6& 8& 5& 6& 6& 5& 1& 4& 6& 1& 5& 4& 5& 6& 8\\
    3& 0& 6& 3& 7& 9& 9& 2& 2& 7& 4& 7& 9& 6& 3& 2& 6& 6\\
    4& 6& 0& 2& 6& 4& 4& 4& 2& 6& 3& 6& 5& 6& 2& 6& 5& 7\\
    6& 3& 2& 0& 5& 5& 3& 3& 9& 4& 3& 6& 3& 4& 7& 8& 3& 2\\
    8& 7& 6& 5& 0& 4& 3& 4& 5& 7& 6& 7& 7& 3& 3& 3& 4& 4\\
    5& 9& 4& 5& 4& 0& 8& 5& 5& 5& 7& 5& 1& 8& 5& 4& 3& 3\\
    6& 9& 4& 3& 3& 8& 0& 6& 8& 4& 6& 7& 1& 8& 5& 6& 7& 6\\
    6& 2& 4& 3& 4& 5& 6& 0& 1& 5& 5& 3& 7& 5& 9& 4& 4& 4\\
    5& 2& 2& 9& 5& 5& 8& 1& 0& 4& 5& 2& 4& 5& 4& 5& 4& 7\\
    1& 7& 6& 4& 7& 5& 4& 5& 4& 0& 7& 7& 5& 6& 5& 5& 6& 10\\
    4& 4& 3& 3& 6& 7& 6& 5& 5& 7& 0& 9& 6& 5& 1& 8& 5& 3\\
    6& 7& 6& 6& 7& 5& 7& 3& 2& 7& 9& 0& 6& 5& 4& 5& 4& 6\\
    1& 9& 5& 3& 7& 1& 1& 7& 4& 5& 6& 6& 0& 5& 7& 4& 5& 2\\
    5& 6& 6& 4& 3& 8& 8& 5& 5& 6& 5& 5& 5& 0& 5& 3& 2& 4\\
    4& 3& 2& 7& 3& 5& 5& 9& 4& 5& 1& 4& 7& 5& 0& 8& 5& 6\\
    5& 2& 6& 8& 3& 4& 6& 4& 5& 5& 8& 5& 4& 3& 8& 0& 6& 8\\
    6& 6& 5& 3& 4& 3& 7& 4& 4& 6& 5& 4& 5& 2& 5& 6& 0& 3\\
    8& 6& 7& 2& 4& 3& 6& 4& 7& 10& 3& 6& 2& 4& 6& 8& 3& 0\\
\end{array} \right)
$$

Liczba iteracji ze~względu na~wielkość danych testowych została zwiększona do~150000. Jako że~nie istniała jednoznacznie określona najlepiej działająca wersja algorytmu, testy zostały przeprowadzone z~wykorzystaniem kilku wersji algorytmu, które~w~uprzednio przeprowadzonych testach okazały się~działać najkorzystniej. Były to~strategie zakładające odpowiednio:\\
\begin{enumerate}
\item reprodukcja losowa, mutacja DE/current to best/$n_{v}+1$+$\leftthreetimes$, krzyżowanie OX, F = 0,8,
\item reprodukcja losowa, mutacja DE/current to best/$n_{v}+1$+$\leftthreetimes$, krzyżowanie OX, F = 0,25,
\item reprodukcja losowa, mutacja DE/current to best/$n_{v}+1$+$\leftthreetimes$, krzyżowanie OX, F zgodnie z funkcją \ref{funkcja},
\item reprodukcja rankingowa, mutacja DE/current to best/$n_{v}+1$+$\leftthreetimes$, krzyżowanie OX, F zgodnie z funkcją \ref{funkcja},
\item reprodukcja losowa, mutacja DE/current to best/$n_{v}+1$+$\leftthreetimes$, krzyżowanie dwumianowe, $C_{r}$ = 0,25, F = 0,8
\end{enumerate}

W~tabeli \ref{duzezestawienie} zestawione zostały wartości błędu względnego osiągane przez~poszczególne strategie. W~tabeli przyjęto numeracje kolumn zgodnie w~wyżej umieszczoną numeracją strategii.

\begin{table}[h!]
\begin{center}
\caption{Wartości błędu względnego funkcji celu dla poszczególnych wersji algorytmu, dane wejściowe II.}
\scalebox{0.55}{
\begin{tabular}{|c|c|c|c|c|c|}
\hline
\textbf{Iteracja}  &\textbf{1} &\textbf{2} &\textbf{3} &\textbf{4} &\textbf{5}\\ \hline
\textbf{1}&\color{green}\textbf{0}\%&0,49\%&0,15\%&\color{green}\textbf{0}\%&\color{green}\textbf{0}\% \\ \hline
\textbf{2}&0,04\%&0,30\%&0,34\%&0,22\%&\color{green}\textbf{0}\% \\ \hline
\textbf{3}&0,60\%&0,15\%&0,67\%&0,04\%&0,04\% \\ \hline
\textbf{4}&0,82\%&0,34\%&0,19\%&0,26\%&\color{green}\textbf{0}\% \\ \hline
\textbf{5}&0,07\%&0,22\%&0,60\%&\color{green}\textbf{0}\%&\color{green}\textbf{0}\% \\ \hline
\textbf{6}&0,63\%&0,15\%&0,19\%&\color{green}\textbf{0}\%&\color{green}\textbf{0}\% \\ \hline
\textbf{7}&0,60\%&0,04\%&0,34\%&\color{green}\textbf{0}\%&\color{green}\textbf{0}\% \\ \hline
\textbf{8}&\color{green}\textbf{0}\%&0,34\%&0,26\%&\color{green}\textbf{0}\%&\color{green}\textbf{0}\% \\ \hline
\textbf{9}&0,60\%&0,56\%&0,71\%&0,11\%&\color{green}\textbf{0}\% \\ \hline
\textbf{10}&0,15\%&0,15\%&0,56\%&0,15\%&0,15\% \\ \hline
\textbf{11}&0,22\%&0,15\%&0,49\%&0,11\%&\color{green}\textbf{0}\% \\ \hline
\textbf{12}&0,26\%&0,34\%&0,30\%&\color{green}\textbf{0}\%&\color{green}\textbf{0}\% \\ \hline
\textbf{13}&0,34\%&\color{green}\textbf{0}\%&0,22\%&0,15\%&\color{green}\textbf{0}\% \\ \hline
\textbf{14}&0,86\%&0,37\%&0,49\%&\color{green}\textbf{0}\%&\color{green}\textbf{0}\% \\ \hline
\textbf{15}&0,15\%&0,22\%&\color{green}\textbf{0}\%&0,15\%&0,04\% \\ \hline
\textbf{16}&0,60\%&0,19\%&0,41\%&0,04\%&\color{green}\textbf{0}\% \\ \hline
\textbf{17}&0,60\%&0,19\%&0,37\%&0,15\%&0,15\% \\ \hline
\textbf{18}&0,49\%&0,22\%&0,60\%&0,04\%&0,22\% \\ \hline
\textbf{19}&0,22\%&0,34\%&0,26\%&\color{green}\textbf{0}\%&\color{green}\textbf{0}\% \\ \hline
\textbf{20}&0,19\%&0,15\%&0,67\%&0,04\%&0,07\% \\ \hline
\textbf{ŚREDNIA}&0,37\%&0,24\%&0,39\%&0,07\%&0,03\% \\ \hline
\textbf{ODCHYLENIE}&0,27\%&0,14\%&0,20\%&0,08\%&0,06\% \\ \hline
\end{tabular}}
\label{duzezestawienie}
\end{center}
\end{table}

Na~podstawie tabeli zauważyć można najkorzystniejsze działanie strategii 5, gdyż średnia wartość błędu względnego jest~najmniejsza spośród wszystkich metod i~wynosi jedynie 0,03\%. Odchylenie standardowe jest~również niewielkie ( 0,06\% ) co oznacza, iż~algorytm ten charakteryzuje się~dużym skupieniem wyników wokół średniej wartości. Strategia zakładająca zastosowanie reprodukcji losowej, mutacji DE/current to~best/$n_{v}+1$+$\leftthreetimes$ oraz~krzyżowania dwumianowego z~wartością współczynnika ustaloną na~stałym poziomie równym $C_{r} = 0,25$, aż w~14 na~20 wykonanych iteracji algorytmu osiągała wartość błędu względnego równą 0\%. Satysfakcjonującym wynikiem odznacza się~także strategia 4 osiągająca wartość średniego błędu względnego równą 0,07\%. Od~strategii 3 odróżnia ją jedynie zastosowanie metody rankingowej jako metody reprodukcji, niemniej jednak zabieg ten pozwala na~osiągnięcie ponad czterokrotnie lepszych wyników. Zależności pomiędzy metodami zestawione zostały w~formie poniższego rankingu. 


\begin{table}[h]
\begin{center}
\caption{Ranking zastosowanych strategii dla danych wejściowych II.}
\scalebox{0.8}{
\begin{tabular}{|c|c|c|c|}
\hline
\textbf{Miejsce}  &\textbf{Strategia}  & \textbf{Średni błąd względny funkcji celu} & \textbf{Liczba osiągnietych wartośći 0\%}\\\hline
 \textbf{1}&5&0,03\%& 14\\\hline
 \textbf{2}&4&0,07\%& 8\\\hline
 \textbf{3}&2&0,24\%& 1\\\hline
 \textbf{4}&1&0,37\%&2\\\hline
 \textbf{4}&3&0,39\%&0\\\hline
\end{tabular}}
\label{rankingduzedane}
\end{center}
\end{table}

Strategia 1, zakładająca przypisanie do~współczynnika mutacji $F$ wartości równej 0,8, oraz~3 używająca w~tym celu funkcji określonej w~podrozdziale \ref{funkcja}, znacząco odbiegają od~wyników uzyskiwanych przez~strategie 5 oraz~4. 

\section{Pozostałe testy danych wejściowych}

W~rozdziale tym zostały przeprowadzone testy z~zastosowaniem wersji algorytmu, która~okazała się~działać najskuteczniej w~testach opisanych w~powyższych rozdziałach. W~testach dla~instancji II wersja ta osiągnęła pierwsze miejsce w~rankingu, natomiast dla~instancji I~drugie miejsce. Za~wersję tą został uznany algorytm wykorzystujący w~swoim działaniu metodę reprodukcji losowej, metodę mutacji  DE/current to~best/$n_{v}+1$+$\leftthreetimes$ oraz~metodę krzyżowania dwumianowego z~parametrem $C_{r}$ równym 0,25. Instancje danych wejściowych wraz~z~odpowiednią dla~nich nomenklaturą zostały zaczerpnięte z~biblioteki QAPLIB \cite{qaplib}. Różnią się~one między sobą przede wszystkim rozmiarem oraz~najmniejszą wartością funkcji celu jaką udało się~uzyskać dla~konkretnej instancji danych. Liczba iteracji dla~której~wykonano testy była równa 150 000. W~poniższej tabeli zostały zestawione uzyskane wyniki. Dla~każdej instancji testowej została umieszczona wartość funkcji celu uznawana przez~biblioteke \cite{qaplib} za~najmniejszą. Względem tej wartości obliczany był średni błąd względny funkcji celu uzyskany na~podstawie 20 powtórzeń algorytmu dla~każdej z~instancji danych wejściowych.

\begin{table}[h!]
\begin{center}
\caption{Zestawienie działania algorytmu dla poszczególnych instancji danych wejściowych.}
\scalebox{0.55}{
\begin{tabular}{|c|c|c|c|}
\hline
\textbf{Instancja testowa}  &\textbf{QAPLIB} &\textbf{Średni błąd względny funkcji celu} &\textbf{Odchylenie standardowe błędu względnego funkcji celu} \\ \hline
\textbf{Els19}&17212548&0,3\%&0,41\% \\ \hline
\textbf{Had14}&2724&0,007\%&0,03\% \\ \hline
\textbf{Had20}&6922&0,07\%&0,08\% \\ \hline
\textbf{Chr15a}&9896&0,29\%&0,32\% \\ \hline
\textbf{Chr18b}&1534&0,32\%&0,37\% \\ \hline
\textbf{Chr22a}&6156&5,13\%&1,41\% \\ \hline
\textbf{Esc16a}&68&0\%&0\% \\ \hline
\textbf{Esc16h}&996&0\%&0\% \\ \hline
\textbf{Esc16j}&8&0\%&0\% \\ \hline
\textbf{Esc32d}&200&2,18\%&1,41\% \\ \hline
\textbf{Nug21}&2438&0,99\%&0,46\% \\ \hline
\textbf{Nug25}&3744&2,21\%&0,62\% \\ \hline
\textbf{Bur26a}&5426670&Osiągnięto wartości mniejsze niż QAPLIB &Osiągnięto wartości mniejsze niż QAPLIB\\ \hline
\textbf{Bur26f}&3782044&Osiągnięto wartości mniejsze niż QAPLIB &Osiągnięto wartości mniejsze niż QAPLIB\\ \hline
\textbf{Tai50a}&4938796&5,25\%&4,17\% \\ \hline
\end{tabular}}
\label{malezestawienie}
\end{center}
\end{table}

Na~podstawie tabeli \ref{malezestawienie} można zauważyć istnienie relacji zachądzących w~zależności od~konkretnych instancji danych wejściowych. Macierze opisujące ten sam rodzaj rzeczywistego problemu osiągają podobne do~siebie wyniki, tj. np Esc161, Esc16h oraz~Esc16j. Macierze o~większym rozmiarze charakteryzują się~wiekszą wartością błędu względnego oraz~odchylenia standardowego, co można zauważyć np. na~podstawie instancji Had14 oraz~Had20 czy~też Chr15a, Chr18b oraz~Chr22a. Analizując działanie przypadków testowych o~rozmiarze większym od~20 można było dostrzec trudności w~dochodzeniu algorytmu do~rozwiązania najmniejszego zgodnie z~biblioteką QAPLIB. Interesującym wydaje się~być fakt, że~wyniki osiągnięte dla~instancji Bur26a oraz Bur26f, okazały się~być lepsze od~wartości znajdującej się~w~bibliotece QAPLIB. Były on lepsze o około 2,76 \%, co może świadczyć o~zbieganiu testowanego algorytmu do~innego minimum lokalnego niż algorytm na~podstawie którego~obliczono wartość zawartą w~bibliotece QAPLIB.





























