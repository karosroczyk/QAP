\documentclass[11pt]{aghdpl}
% \documentclass[en,11pt]{aghdpl}  % praca w języku angielskim

% Lista wszystkich języków stanowiących języki pozycji bibliograficznych użytych w pracy.
% (Zgodnie z zasadami tworzenia bibliografii każda pozycja powinna zostać utworzona zgodnie z zasadami języka, w którym dana publikacja została napisana.)
\usepackage[english,polish]{babel}

% Użyj polskiego łamania wyrazów (zamiast domyślnego angielskiego).
\usepackage{polski}

\usepackage[utf8]{inputenc}

% Załączniki

\usepackage[toc, page]{appendix}
\renewcommand\appendixpagename{Załączniki}
\renewcommand\appendixtocname{Załączniki}

% dodatkowe pakiety

\usepackage{mathtools}
\usepackage{amsfonts}
\usepackage{amsmath}
\usepackage{amsthm}
\usepackage{float}% do umieszczenia floatów [H]
\usepackage{enumitem}
\setlist{nosep} % or \setlist{noitemsep} to leave space around whole list
\usepackage[bookmarks,hidelinks]{hyperref}

% Środowisko float do kodu źródłowego \begin{program}

\floatstyle{plaintop}
\ifcsname{chapter}\endcsname%
    \newfloat{program}{!tbh}{lop}[chapter]
\else%
    \newfloat{program}{!tbh}{lop}
\fi
\floatname{program}{Kod źr.}

% Kod poniżej powoduje, że floaty nie wylatują poza granice sekcji

\usepackage{placeins}

\ifcsname{chapter}\endcsname%
    \let\Oldchapter\chapter%
    \renewcommand{\chapter}{\FloatBarrier\Oldchapter}
\fi

\let\Oldsection\section%
\renewcommand{\section}{\FloatBarrier\Oldsection}

\let\Oldsubsection\subsection%
\renewcommand{\subsection}{\FloatBarrier\Oldsubsection}

\let\Oldsubsubsection\subsubsection%
\renewcommand{\subsubsection}{\FloatBarrier\Oldsubsubsection}

% --- < bibliografia > ---

\usepackage[
style=numeric,
sorting=none,
%
% Zastosuj styl wpisu bibliograficznego właściwy językowi publikacji.
language=autobib,
autolang=other,
% Zapisuj datę dostępu do strony WWW w formacie RRRR-MM-DD.
urldate=iso8601,
% Nie dodawaj numerów stron, na których występuje cytowanie.
backref=false,
% Podawaj ISBN.
isbn=true,
% Nie podawaj URL-i, o ile nie jest to konieczne.
url=false,
%
% Ustawienia związane z polskimi normami dla bibliografii.
maxbibnames=3,
% Jeżeli używamy BibTeXa:
backend=bibtex
]{biblatex}

\usepackage{csquotes}
% Ponieważ `csquotes` nie posiada polskiego stylu, można skorzystać z mocno zbliżonego stylu chorwackiego.
\DeclareQuoteAlias{croatian}{polish}

\addbibresource{bibliografia.bib}

% Przecinki zamiast kropek do oddzielenia pól wpisu bibliograficznego
% i dwukropek po nazwisku autora, bez kropki na końcu
\AtBeginBibliography{
    \renewcommand\labelnamepunct{:\space}
    \renewcommand\newunitpunct{\addcomma\space}
    \renewcommand{\finentrypunct}{}
    
    \renewcommand{\bibopenparen}{\addcomma\addspace}
    \renewcommand{\bibcloseparen}{\addspace}
}


% Nie wyświetlaj wybranych pól.
%\AtEveryBibitem{\clearfield{note}}


% ------------------------
% --- < listingi > ---

% Użyj czcionki kroju Courier.
\usepackage{courier}
\usepackage{newtxmath}

\usepackage{listings}
%\lstloadlanguages{TeX}
\lstset{language=TeX}

\lstset{
	literate={ą}{{\k{a}}}1
           {ć}{{\'c}}1
           {ę}{{\k{e}}}1
           {ó}{{\'o}}1
           {ń}{{\'n}}1
           {ł}{{\l{}}}1
           {ś}{{\'s}}1
           {ź}{{\'z}}1
           {ż}{{\.z}}1
           {Ą}{{\k{A}}}1
           {Ć}{{\'C}}1
           {Ę}{{\k{E}}}1
           {Ó}{{\'O}}1
           {Ń}{{\'N}}1
           {Ł}{{\L{}}}1
           {Ś}{{\'S}}1
           {Ź}{{\'Z}}1
           {Ż}{{\.Z}}1,
	basicstyle=\footnotesize\ttfamily,
}

% Ustawienia pakietu lstlisting do umieszczania kodu

\usepackage{color}

\definecolor{mygreen}{rgb}{0,0.6,0}
\definecolor{mygray}{rgb}{0.5,0.5,0.5}
\definecolor{mymauve}{rgb}{0.58,0,0.82}

\lstset{%
  backgroundcolor=\color{white},     % choose the background color
  basicstyle=\ttfamily\footnotesize, % size of fonts used for the code
  breaklines, breakatwhitespace,     % automatic line breaking only at whitespace
  commentstyle=\color{mygreen},      % comment style
  numbers=left,
  showstringspaces=false,
  numberstyle=\tiny,
  frame=l,
  escapeinside={*@}{@*},           % if you want to~add LaTeX within your code
  keywordstyle=\color{blue},         % keyword style
  stringstyle=\color{mymauve}        % string literal style
}

% ------------------------

\AtBeginDocument{
	\renewcommand{\tablename}{Tabela}
	\renewcommand{\figurename}{Rys.}
}

% ------------------------
% --- < tabele > ---

\usepackage{array}
\usepackage{tabularx}
\usepackage{multirow}
\usepackage{booktabs}
\usepackage{makecell}
\usepackage[flushleft]{threeparttable}

% defines the X column to use m (\parbox[c]) instead of p (`parbox[t]`)
\newcolumntype{C}[1]{>{\hsize=#1\hsize\centering\arraybackslash}X}


%---------------------------------------------------------------------------

\author{Karolina Sroczyk}
\shortauthor{K.Sroczyk}

%\titlePL{Przygotowanie bardzo długiej i pasjonującej pracy dyplomowej w~systemie~\LaTeX}
%\titleEN{Preparation of a very long and fascinating bachelor or master thesis in \LaTeX}

\titlePL{Zmodyfikowany algorytm ewolucji różnicowej dla problemu QAP}
\titleEN{A modified differential evolution algorithm for  the QAP problem}


\shorttitlePL{Zmodyfikowany algorytm ewolucji różnicowej dla problemu QAP} % skrócona wersja tytułu jeśli jest bardzo długi
\shorttitleEN{A modified differential evolution algorithm for  the QAP problem}

\thesistype{Projekt dyplomowy}
%\thesistype{Master of Science Thesis}

\supervisor{dr inż. Wojciech Chmiel}
%\supervisor{Marcin Szpyrka PhD, DSc}

\degreeprogramme{Automatyka i Robotyka}
%\degreeprogramme{Computer Science}

\date{2020}

\department{Katedra Automatyki i Robotyki}
%\department{Department of Applied Computer Science}

\faculty{Wydział Elektrotechniki, Automatyki,\protect\\[-1mm] Informatyki i Inżynierii Biomedycznej}
%\faculty{Faculty of Electrical Engineering, Automatics, Computer Science and Biomedical Engineering}

\acknowledgements{Rodzicom, za wszystkie lata nauki, dedykuję i dziękuję.}


\setlength{\cftsecnumwidth}{10mm}

%---------------------------------------------------------------------------
\setcounter{secnumdepth}{4}
\brokenpenalty=10000\relax

\begin{document}

\titlepages

% Ponowne zdefiniowanie stylu `plain`, aby usunąć numer strony z pierwszej strony spisu treści i poszczególnych rozdziałów.
\fancypagestyle{plain}
{
	% Usuń nagłówek i~stopkę
	\fancyhf{}
	% Usuń linie.
	\renewcommand{\headrulewidth}{0pt}
	\renewcommand{\footrulewidth}{0pt}
}

\setcounter{tocdepth}{2}
\tableofcontents
\clearpage

\chapter{Wstęp}
Główny cel pracy stanowi opracowanie i~przeanalizowanie poprawności działania zmodyfikowanego algorytmu ewolucji różnicowej przystosowanego do~rozwiązywania problemu kwadratowego zagadnienia przydziału (Quadratic Assigment Problem). Ze~względu na~fakt, iż~QAP należy do~klasy problemów NP-trudnych, a~więc problemów, dla~których nie~istnieje jednoznacznie określony algorytm dający optymalne wyniki, istnieje konieczność opracowania algorytmu przynoszącego satysfakcjonujące efekty w~przypadku konkretnie postawionego zagadnienia. Dobrych wyników optymalizacji można spodziewać się~wówczas gdy opracowany algorytm wykorzystuje specyficzne cechy oraz~regularności rozwiązywanego problemu.\\ 
W~celu uzyskania satysfakcjonujących wyników dla~problemów NP-trudnych, można posłużyć się~metodami metaheurystycznymi, a~więc metodami umożliwiającymi iteracyjne poprawianie wyniku na~podstawie przyjętych uprzednio funkcji oceny. Metody metaheurystyczne w~dużej mierze inspirują się~zjawiskami zachodzącymi w~naturze, z~tego względu omawiając je operuje się~nomenklaturą zaczerpniętą ze~świata biologii. Dają one jednak rozwiązanie przybliżone, co oznacza, iż~nie ma pewności, że~uzyskany wynik jest~minimum globalnym. Jedną z~metod często stosowaną w~celu rozwiązywania problemów optymalizacji globalnej jest~algorytm ewolucji różnicowej. Algorytm ten w~dużej mierze oparty jest~na~losowości oraz~rozwiązaniach o~charakterze intuicyjnym, niemniej jednak dających oczekiwane rezultaty. Niejednokrotnie dokonuje się~modyfikacji tego algorytmu w~zależności od~potrzeb konkretnie rozwiązywanego problemu. Modyfikacje klasycznej wersji algorytmu najczęściej wykazują swoją skuteczność w~przypadku działania w~kontekście jednego wybranego zagadnienia. Natomiast w~przypadku zastosowania ich dla~innych problemów skuteczność ta ewidentnie spada. W~przedmiotowej pracy przeprowadzona zostanie analiza wpływu poszczególnych modyfikacji na~wyniki osiągane przez~algorytm, tak~by~móc wyodrębnić spośród nich strategie osiągające najkorzystniejsze wyniki.\\
Algorytmy wchodzące w~skład metaheurystyk charakteryzują się~posiadaniem rozwiązania będącego wektorem wartości binarnych. Ze~względu na~fakt, iż~algorytm rozważany w~niniejszej pracy ma zostać przystosowany do~optymalizacji problemów QAP, którego~rozwiązanie jest~w~postaci wektora będącego permutacją pewnego zbioru, należy dokonywać modyfikacji klasycznych metod również na~tej płaszczyźnie.\\
Etapy realizacji celu pracy obejmują:\\
\begin{itemize}
\item Zdefiniowanie NP-trudnego problemu, jaki stanowi kwadratowe zagadnienie przydziału (Quadratic Assigment Problem) oraz określenie relacji, jakie zachodzą pomiędzy zmodyfikowanym algorytmem ewolucji różnicowej a problemem, który rozwiązuje. Opracowanie modelu matematycznego opisującego problem QAP i stanowiącego podstawę dalszych rozważań.\\
\item Implementacja podstawowej wersji algorytmu ewolucji różnicowej wraz z wprowadzeniem modyfikacji w algorytmie w celu dostosowania go do rozwiązywania problemu QAP. Modyfikacje dotyczą kwestii opracowania reprezentacji pojedynczego osobnika jako permutacji pewnego zbioru.\\
\item Opracowanie algorytmu ewolucji różnicowej wraz z uwzględnieniem modyfikacjami w obrębie metod reprodukcji, mutacji oraz krzyżowania. Zmiany dotyczą głównie operacji mutacji oraz krzyżowania. Dodatkowo w tych operacjach genetycznych badaniu podlegają odpowiednie współczynniki. \\
\item Badanie skuteczności zaproponowanych wersji zmodyfikowanego algorytmu ewolucji różnicowej oraz porównanie z działaniem podstawowej wersji. Oprócz konfrontacji wyników wersji klasycznej i zmodyfikowanej ważnym czynnikiem jest liczba iteracji, w których algorytm osiąga wartość uznawaną przez bibliotekę \cite{qaplib} za najmniejszą wartość, jaką udało się uzyskać dla konkretnej instancji danych wejściowych.\\ 
\item Przeprowadzanie testów algorytmu dla bardziej oraz mniej złożonych problemów. Rozważane dane wejściowe posiadają inny rozmiar oraz osiągają inne średnie wartości funkcji celu. 
\end{itemize}
Niniejsza praca składa się~z~7 rozdziałów, z~których 5 początkowych zawiera informacje teoretyczne. Opisana została w~nich zarówno klasyczna, jak~i~zmodyfikowana wersja algorytmu ewolucji różnicowej. Rozdział 6 stanowi część badawczą, gdyż został przeznaczony na~testy uprzednio opisanych modyfikacji. Wnioski i~spostrzeżenia, a~także informacje o~najlepiej działających wersjach algorytmu zostały umieszczone w~rozdziale 7.\\
Rozdział 2 zawiera informacje teoretyczne dotyczące problemów optymalizacji zarówno lokalnej, jak~i~globalnej. Został w~nim opisany oraz~przedstawiony schemat algorytmu ewolucji różnicowej. Zawarto także opis nomenklatury dotyczącej algorytmów ewolucyjnych. Drugą część rozdziału stanowi opis oraz~definicja kwadratowego zagadnienia przydziału QAP.\\
Na~rozdział 3 składają się~opisy zasady działania poszczególnych metod reprodukcji rozważanych w~pracy. Zawarte zostały także implementacje poszczególnych metod.\\
Rozdział 4 został poświęcony metodom mutacji w~wersji klasycznej, jak~i~zmodyfikowanej. Zawarte zostały także opisy możliwości modyfikacji współczynnika mutacji.\\
W~rozdziale 5 zaprezentowano metody krzyżowania stosowane w~podstawowych implementacjach algorytmów oraz~te mające bezpośrednie zastosowanie w~przypadku problemów, których podstawową jednostką jest~osobnik będący permutacją pewnego zbioru. Zostały również opisane modyfikacje współczynnika krzyżowania.\\
W~rozdziale 6 przeprowadzono testy wszystkich opisanych metod oraz~modyfikacji. Na~podstawie uzyskanych wyników ustalona została wersja algorytmu uzyskująca najlepsze efekty. Następnie z~wykorzystaniem tej wersji algorytmu przeprowadzone zostały testy z~wykorzystaniem innych instancji danych wejściowych.\\
Rozdział 7 stanowi zakończenie i~podsumowanie pracy. Znajdują się~tam informacje dotyczące wersji algorytmu uznanych za~najskuteczniejsze. Dodatkowo można znaleźć informacje na~temat dalszych kierunków badań.

\chapter{Wstęp teoretyczny}\label{cha:pierwszyDokument}

%---------------------------------------------------------------------------

\section{Opis algorytmu ewolucji różnicowej}\label{sec:strukturaDokumentu}



%---------------------------------------------------------------------------

\section{Opis kwadratowego zagadnienia przydziału}\label{sec:strukturaDokumentu}



\chapter{Reprodukcja}\label{metody_reprodukcji}

Jednym z~etapów algorytmu ewolucyjnego jest~reprodukcja polegająca na~wyborze osobników, które~przedostaną się~do~grupy rozrodczej tworzącej potomstwo. W~klasycznym algorytmie ewolucji różnicowej wybór wektorów bazowych oraz~wektorów składających się~na~wektor różnicowy jest~dokonywany losowo \cite{diff2}. Nie~mniej jednak w~niniejszej pracy w~celach doświadczalnych zostaną porównane i~zestawione ze~sobą dodatkowo metody doboru grupy rodzicielskiej zgodnie z~dostępnymi metodami selekcji stosowanymi w~algorytmach ewolucyjnych.\\

Znanych jest~kilka metod selekcji, na~podstawie których dokonuje się~wyboru osobników z~różnym prawdopodobieństwem. Metody te~w~większości zostały opracowane w~taki sposób, by~dawały większe szanse przetrwania osobnikom lepiej do~tego przystosowanym, a~więc osobnikom o~większym współczynniku funkcji przystosowania. Metody selekcji zostały zaimplementowane w~dwojaki sposób. Pierwszym z~nich jest~możliwość pojawienia się~danego osobnika w~grupie rodzicielskiej więcej niż jeden raz. Druga technika nakazuje różnorodność osobników w~obrębie danej instancji grupy rozrodczej.\\

\begin{equation}
i_{1} \ne i_{2} \ne i_{3} \ne ... \ne i_{n}
\end{equation}
Gdzie:\\
$i_{n}$ - n-ty osobnik\\

Poniżej przestawiono idee metod selekcji podlegających rozważaniom w~ramach tej pracy. Implementacje metod powstały w~oparciu o~\cite{maszynowe_sel}, \cite{gracjan}. Listingi poszczególnych metod reprodukcji można znaleźć w~dodatku \ref{reporeprodukcji}.

%---------------------------------------------------------------------------

\section{Klasyczna reprodukcja : Losowy wybór}\label{sec:strukturaDokumentu}

Z~populacji wybierany jest~osobnik przechodzący do~grupy rozrodczej w~oparciu o~wylosowaną liczbę odpowiadającą indeksowi konkretnego osobnika. Metoda ta jest~niezależna od~wartości funkcji przystosowania, a~więc daje równe szanse każdemu z~osobników. Prawdopodobieństwo wylosowania danego osobnika, z~populacji składającej się~z~$n$ osobników, jest~równe:
\vspace{0,4cm}
\begin{equation}
P = \frac{1}{n}
\end{equation}

%---------------------------------------------------------------------------

\section{Modyfikacje reprodukcji}\label{sec:kompilacja}
\subsection{Metoda rankingowa}\label{sec:kompilacja}


W~metodzie selekcji rankingowej dla~osobników składających się~na~populacje rodziców obliczana jest~wartość funkcji celu, na~której to~podstawie tworzony jest~ranking. Osobniki z~najmniejszą wartością wyniku uzyskują najwyższe miejsca w~rankingu \cite{selekcje}. Miejsce w~rankingu jest~argumentem funkcji, na~podstawie której~obliczane jest~prawdopodobieństwo wyboru danego osobnika jako rodzica. Rodzic zostanie poddany kolejnym operacjom (mutacja, krzyżowanie).\\
Funkcja stosowana, na~której to~podstawie tworzony jest~ranking, może być~funkcją liniową. Wówczas osobniki zajmujące w~rankingu będącym $n$ elementowym wektorem konkretne miejsce określone rangą, gdzie $n$  to~ilość osobników w~populacji otrzymują prawdopodobieństwo przejścia do~grupy rodzicielskiej określone zależnością \ref{wzorranking}:
\begin{equation}
P = \frac{ranga}{\sum_{j=1}^{n}j}
\label{wzorranking}
\end{equation}

\begin{table}[h!]
\begin{center}
\caption{Wartości parametrów na kolejnych osobników w metodzie rankingowej.}
\begin{tabular}{|c|c|c|c|}
\hline
\textbf{Osobnik}  & \textbf{Funkcja celu} & \textbf{Ranga} & \textbf{Prawdopodobieństwo}\\
\hline
Osobnik 1 & 373 & 1 & 6,(6) \% \\
\hline
Osobnik 2 &254 & 2  & 13,(3)  \% \\
\hline
Osobnik 3 & 202 & 3 & 20  \% \\
\hline
Osobnik 4 & 136 & 4 & 26,(6)  \% \\
\hline
Osobnik 5 & 10 & 5 & 33,(3)  \% \\
\hline
\end{tabular}
\end{center}
\end{table}

\vspace{0,4cm}

\begin{figure}[h]
		\includegraphics[scale=0.45]{../../../../Screeny/metoda_rankingowa.jpg}
		\caption{Rozkład prawdopodobieństwa w~metodzie rankingowej.}
		\label{rankingR}			
\end{figure}
\par
Na~podstawie \ref{rankingR} zauważalny staje się~fakt, iż~rozkład prawdopodobieństwa nie~jest~bezpośrednio zależny od~wartości funkcji celu. W~metodzie tej zwiększone są~więc szanse przejścia do~grupy rodzicielskiej osobników z~małą wartością funkcji dopasowania, które~to miałyby małe szanse przejścia dalej w~metodzie selekcji zależnej od~wartości funkcji celu. Metoda może wydawać się~niekorzystna z~punktu widzenia osobników z~dużą wartością funkcji dopasowania, jako że~mogą otrzymać one niewiele większe prawdopodobieństwo przejścia do~grupy rodzicielskiej od~osobników znajdujących się~rangę niżej, pomimo iż~ich wartości funkcji celu mogą różnić się~znacząco.
%---------------------------------------------------------------------------

\subsection{Metoda ruletki}\label{sec:narzedzia}


W~metodzie ruletki każdemu chromosomowi zostaje przypisany wycinek koła \cite{selekcje}. Im silniejszy osobnik, tym większy wycinek koła dostaje, gdyż zależność wartości funkcji przystosowania i~prawdopodobieństwo wylosowania osobnika jest~wprost proporcjonalna. Każdy osobnik tworzący populacje poddawany jest~ocenie funkcji celu. Maksymalna wartość funkcji przystosowania oznacza najmniejszą wartość funkcji celu. Uzyskana wartość funkcji przystosowania w~stosunku do~sum wartości tej funkcji wszystkich osobników populacji wyznacza prawdopodobieństwo wylosowania danego osobnika na~rodzica. Prawdopodobieństwo wylosowania danego osobnika dane jest~następującą zależnością:
\begin{equation}
P_i = \frac{f(i)}{\sum_{j=1}^{n}f(j)}
\end{equation}
Gdzie f(x) - funkcja przystosowania\\
\par
Funkcją przystosowania w~rozważanym poniżej przypadku jest~funkcja przyporządkowującą największej wartości funkcji celu wartość najmniejszą, a~wartości najmniejszej wartość największą.

\begin{table}[h!]
\begin{center}
\caption{Wartości parametrów na kolejnych osobników w metodzie ruletki.}
\begin{tabular}{|c|c|c|c|}
\hline
\textbf{Osobnik}  & \textbf{Funkcja celu} & \textbf{Funkcja dopasowania} & \textbf{Prawdopodobieństwo}\\
\hline
Osobnik 1 & 373 & 10 & 1,02  \% \\
\hline
Osobnik 2 &254 & 136  & 13,95  \% \\
\hline
Osobnik 3 & 202 & 202 & 20,72  \% \\
\hline
Osobnik 4 & 136 & 254 & 26,05  \% \\
\hline
Osobnik 5 & 10 & 373 & 38,25  \% \\
\hline
\end{tabular}
\end{center}
\end{table}

\vspace{0,4cm}

\begin{figure}[h]
		\includegraphics[scale=0.45]{../../../../Screeny/metoda_ruletki.jpg}
		\caption{Rozkład prawdopodobieństwa w~metodzie ruletki.}
		\label{ruletka}			
\end{figure}
\par
W~metodzie ruletki wkład każdego osobnika w~rozkład prawdopodobieństwa jest~wprost proporcjonalny do~wartości jego funkcji dopasowania. Wady tej metody stają się~widoczne, w~sytuacji, gdy dany osobnik zajmuje większość powierzchni koła, gdyż osobniki o~małej wartości funkcji dopasowania są~przedwcześnie eliminowane, a~więc nie~mają wielkiej szansy, by~zostać wylosowane. Prowadzi to~do~braku różnorodności w~kolejnych grupach rozrodczych, co utrudnia dojście do~lepszego wyniku. Z~problemem tym dobrze sobie radzi wspomniana wyżej metoda rankingowa, gdzie eliminowana jest~możliwość dużej przewagi jednego z~nich nad~resztą.\\
%---------------------------------------------------------------------------

\subsection{Metoda turniejowa}\label{sec:narzedzia}


Metoda selekcji turniejowej polega na~rozgrywaniu turnieju pomiędzy osobnikami tworzącymi populacje. W~wyniku jednorazowej rozgrywki wygrywa osobnik o~większej wartości funkcji przystosowania, przechodząc tym samym do~kolejnej rundy. W~każdym turnieju zwycięża jeden osobnik, którego~to genotyp jest~wykorzystywany przy kolejnych operacjach genetycznych. Populacja dzielona jest~na~podgrupy złożone z~dwóch osobników, w~których to~dokonywane są~rozgrywki. W~przypadku nieparzystego rozmiaru populacji osobnik ostatni przechodzi do~kolejnej rundy bezkonkurencyjnie. \\
\par
Metoda turniejowa nie~wymaga znajomości optymalizowanej funkcji, konieczny jest~jedynie informacja o~relacji zachodzącej pomiędzy dwoma kolejnymi osobnikami. 
%---------------------------------------------------------------------------

\subsection{Metoda elitarna}\label{sec:przygotowanieDokumentu}

W~metodzie tej nacisk położony jest~na~to, aby~do~grupy rozrodczej przedostały się~z~prawdopodobieństwem równym p=100\% osobniki najlepiej przystosowane. W~poprzednich metodach (ruletki, turniejowa, rankingowa) prawdopodobieństwo przejścia do~grupy rozrodczej dla~osobników z~większym wskaźnikiem funkcji przystosowania jest~jedynie większe niż dla~pozostałych osobników. Nie~ma jednak gwarancji przedostania się~dalej osobników najlepiej przystosowanych. W~strategii elitarnej do~grupy, na~której zostanie dokonana operacja mutacji, a~następnie krzyżowania przedostaje się~n osobników z~populacji odznaczających się~najwyższym wskaźnikiem funkcji przystosowania \cite{michal}.\\
\par
Zastosowanie elitaryzmu jako metody selekcji daje pewność, iż~do~grupy rodzicielskiej przedostaną się~najlepiej do~tego przystosowane osobniki. Dzięki temu unika się~sytuacji, w~której stosunkowo mocny osobnik w~wyniku operacji genetycznych zostanie osłabiony, a~co za~tym idzie, nie~przechodzi on do~grupy rodzicielskiej.\\
\chapter{Operacja mutacji}\label{cha:pierwszyDokument}

Mutacja stanowi najważniejszą część algorytmu ewolucyjnego, z~tego względu stosuje się~wiele modyfikacji w~celu rozwoju i~poprawy działania tego operatora.
Mutacja stosowana w~algorytmie ewolucji różnicowej zdefiniowana jest~w~odmienny sposób w~stosunku do~algorytmu ewolucyjnego, w~którym to~genotyp stanowią wartości binarne, a~operator mutacji ma za~zadanie wprowadzenie zmian na~zasadzie negacji obecnej wartości genu. W~mutacji różnicowej zastosowana jest~operacja odejmowania od~siebie dwóch wektorów, tworząc tym samym wektor różnicowy, czemu również algorytm zawdzięcza swoją nazwę z~ang. \textsl{difference} \cite{przystojny_koles}. Wektor różnicowy jest~zatem miarą określającą odległość pomiędzy dwoma osobnikami. Podobnie jak~ma to~miejsce w~innych algorytmach należących do~grona algorytmów ewolucyjnych, tak~i~ewolucja różnicowa posiada wiele alternatywnych metod mutacji. Metody mutacji różnicowej różnią się~od~siebie głównie elementami takimi jak:


\begin{enumerate}
\item Sposób  wyboru osobnika będącego wektorem bazowym, a więc wektorem znajdującym się w grupie rozrodczej, lecz nie wchodzącym w skład wektora różnicowego \textsl{X},
\item Liczba wektorów różnicowych \textsl{Y},
\end{enumerate}
\par
W~związku z~powyższym, metody mutacji mogą przyjąć następującą nomenklaturę \textsl{DE/X/Y/Z}. \textsl{Z} oznacza rodzaj zastosowanego operatora krzyżowania w~związku z~czym parametr ten nie~będzie uwzględniony w~poniższym opisie metod mutacji . Wybór danej metody jest~zależny od~specyfiki rozwiązywanego problemu i~dla różnych jego instancji może dawać odmienne rezultaty. Ogólny wzór opisujący sposób powstawania genu osobnika potomnego można wyrazić następująco:
\begin{equation}
 \forall U_{i} =S_{r_{1}i} + \sum_{j=1}^{n} F_{j} \cdot (S_{r_{2j}i} - S_{r_{3j}i})
\end{equation}
Gdzie:\\
$j$ - liczba wektorów różnicowych,\\
$U_{i}$ - i-ty gen osobnika potomnego,\\
$r_{1}$,$ r_{2}$,$ r_{3}$ - osobniki wchodzące w~skład populacji,\\
$S_{r_{1}i}$ - i-ty gen osobnika $r_{1}$\\
$F$ - współczynnik mutacji, $F \in (0,1)$,\\
$S$ -populacja,\\


Wprowadzono zależność, iż~osobnikami biorącymi udział w~procesie mutacji są~osobniki uprzednio wybrane poprzez~metodę reprodukcji \cite{diff2}. W~implementacji metod mutacji dodatkowo należało zastosować pewnego rodzaju modyfikacje, wynikającą z~faktu, iż~pojedynczy osobnik jest~ciągiem będącym permutacją, a~więc nie~dopuszcza istnienia wartości np. zmiennoprzecinkowych. Jako że~wektor różnicowy mnożony jest~przez~współczynnik mutacji, będący wartością zmiennoprzecinkową, wektor końcowy posiada geny będące wartościami zmiennoprzecinkowymi. W~celu eliminacji takiej sytuacji należało dokonać stosownego rodzaju skalowanie uwzględniające relacje mniejszości i~większości pomiędzy poszczególnymi genami, a~jednocześnie zamieniające wartości zmiennoprzecinkowe na~całkowite. Pseudokod wspomnianej modyfikacji zamieszczono poniżej:

\begin{flushleft}
\scriptsize
\hspace{5cm}$cnt$ = 1\\
\hspace{5cm}while ( $cnt$ != $ Rozmiar Osobnika + 1$) do\\
\hspace{5.5cm}	1. $index \leftarrow$ indeks najmniejszego elementu w wektorze wynikowym mutacji\\
\hspace{5.5cm}	2. W wektorze wynikowym mutacji, pod $index$, wstaw $MaxSizeOf(double)$\\
\hspace{5.5cm}	3. W wektorze wynikowym operacji, pod $index$, wstaw wartość $cnt$\\
\hspace{5.5cm}	4. Zwiększ $cnt$\\
\hspace{5cm}end while \\
\end{flushleft}

 Poniżej zestawione zostały strategie najczęściej spotykane w~literaturze i~opisane szczegółowo w~\cite{doktorat}. Listingi poszczególnych metod mutacji można znaleźć w~dodatku \ref{repomutacji}.
%---------------------------------------------------------------------------

\section{Klasyczna mutacja: DE/rand/1}\label{sec:strukturaDokumentu}

Podstawową strategią stosowaną jako operator mutacji jest~strategia zakładająca uznanie za~wektor bazowy pierwszego osobnika znajdującego się~w~grupie rozrodczej \cite{diff2}. Jako że~grupa rozrodcza składa się~z~trzech osobników, dwa pozostałe są~odpowiedzialne za~utworzenie wektora różnicowego, który~jest następnie skalowany współczynnikiem $F = 0,8$. Przyjęta wartość jest~najczęściej spotykaną w~literaturze wartością czynnika skalującego. Analiza wpływu tego parametru na~działanie metod mutacji zostanie przedstawiona w~dalszej części pracy. Powyżej wspomniane zależności można opisać wzorem:
\begin{equation}
\label{firatmutate}
 \forall U_{i} =S_{r_{1}i} + F \cdot (S_{r_{2}i} - S_{r_{3}i})
\end{equation}
%---------------------------------------------------------------------------
\section{Modyfikacje metod mutacji}\label{sec:kompilacja}

Klasyczna metoda mutacji pomimo częstego jej stosowania, nie~zawsze daje oczekiwane rezultaty. W~pewnych instancjach problemów efektywniejszym działaniem mogą wykazać się~metody mutacji, do~których zostały wprowadzone pewnego rodzaju modyfikacje. Poniżej zestawionych zostało kilka możliwych modyfikacji opisanych bardziej szczegółowo \cite{diff2}.

\subsection{Strategia II: DE/best/1}\label{sec:kompilacja}

W~strategii II miejsce wektora bazowego zajmuje wektor odznaczający się~największą wartością funkcji celu. W~celu znalezienia najlepszego osobnika konieczne jest~przeszukanie całej populacji, co może okazać się~niekorzystne w~kontekście złożoności obliczeniowej i~eksploatacji algorytmu. Wektor różnicowy jest~różnicą dwóch losowo wybranych osobników przeskalowanych współczynnikiem mutacji. Zależność strategii DE/best/1 można opisać wzorem:

\begin{equation}
 \forall U_{i} =S_{best,i} + F \cdot (S_{r_{1}i} - S_{r_{2}i})
\end{equation}
%---------------------------------------------------------------------------

\subsection{Strategia III: DE/rand/$n_{v}$}\label{sec:narzedzia}

Strategię III od~poprzednich metod wyróżnia fakt, iż~nowy osobnik powstaje w~oparciu o~sumę wektorów różnicowych, a~nie na~podstawie jednego wektora różnicowego tak~jak~miało to~miejsce w~przypadku poprzednich metod. Istnieje więc~dodatkowy parametr $n_{v}$ określający ilość wektorów różnicowych. W~sytuacji gdy $ 2 * n_{v} + 1$ jest~większe od~rozmiaru populacji \textsl{n}, do~$n_{v}$ zostaje przypisana odgórna wartość równa największej liczbie wektorów różnicowych, które~można utworzyć przy danym rozmiarze populacji \textsl{n}.\\
\begin{equation}
 n_{v} = (n - 1) / 2
\end{equation}

Wartość będąca sumą wektorów różnicowych jest~następnie skalowana współczynnikiem mutacji \textsl{F}. Opisane powyżej zależności można zapisać w~postaci wzoru:\\
\begin{equation}
 \forall U_{i} =S_{r_{1}i} +  F \cdot \sum_{k=1}^{n_{v}}(S_{r_{2}i} - S_{r_{3}i})
\end{equation}
%---------------------------------------------------------------------------

\subsection{Strategia IV: DE/current to best/$n_{v} +1$}\label{sec:narzedzia}


Strategia ta zawiera w~sobie pewne elementy zawarte we wcześniej opisanych metodach mutacji różnicowej. Na~osobnika potomnego składa się~suma $n_{v}$ wektorów różnicowych, będących różnicą dwóch losowo wybranych osobników z~populacji, pomnożonych razy współczynnik mutacji (strategia III). Sumuje się~także wartość, przemnożonego przez~współczynnik mutacji, odrębnego wektora różnicowego, w~skład którego~wchodzi najlepszy (strategia II) oraz~losowy osobnik. Dodatkowo należy również uwzględnić wartość osobnika o~indeksie odpowiadającym iteracji, w~której znajduje się~obecnie algorytm. Zależności te~opisuje wzór \ref{wzormutacja4}:\\

\begin{equation}
 \forall U_{i} = S_{r_{itert}i} +  F \cdot (S_{r_{best}i} - S_{r_{1}i}) +  F \cdot \sum_{k=1}^{n_{v}}(S_{r_{2}i} - S_{r_{3}i})
\label {wzormutacja4}
\end{equation}
%---------------------------------------------------------------------------

\subsection{Mutacja z wprowadzeniem parametru $\leftthreetimes$} \label{parametrffunkcja}

W~literaturze \cite{czynnik} można także znaleźć wersje metod mutacji, w~których wprowadza się~modyfikacje powyżej zaprezentowanych metod poprzez~wprowadzenie czynnika skalującego $\leftthreetimes$. Parametr ten wprowadza w~algorytmie dodatkową losowość, co może wpłynąć korzystnie na~szybkość dochodzenia algorytmu do~rozwiązania końcowego. Zmiany, w~stosunku do~poprzednich wersji, dotyczą pomnożenia wektora bazowego przez~czynnik skalujący $\leftthreetimes$, gdzie $\leftthreetimes \in (0,1)$. W~poprzednich strategiach niezależnie od~ich rodzajów wektor bazowy był zawsze wartością nieprzeskalowaną. Można to~opisać poniższymi zależnościami:\\

\begin{enumerate}
\item Strategia I : $ \forall U_{i} = \leftthreetimes \cdot S_{r_{1}i} + F \cdot (S_{r_{2}i} - S_{r_{3}i})$
\item Strategia II : $ \forall U_{i} = \leftthreetimes \cdot  S_{best,i} + F \cdot (S_{r_{1}i} - S_{r_{2}i}) $
\item Strategia III : $ \forall U_{i} = \leftthreetimes \cdot  S_{r_{1}i} +  F \cdot \sum_{k=1}^{n_{v}}(S_{r_{2}i} - S_{r_{3}i}) $
\item Strategia IV : $ \forall U_{i} = \leftthreetimes \cdot S_{r_{itert}i} +  F \cdot (S_{r_{best}i} - S_{r_{1}i}) +  F \cdot \sum_{k=1}^{n_{v}}(S_{r_{2}i} - S_{r_{3}i}) $
\end{enumerate}
\par

%---------------------------------------------------------------------------

\section{Dostrajanie parametru mutacji}\label{rozklad}

W~literaturze \cite{przystojny_koles}, \cite{czynnik}, za~wartość parametru mutacji $F$, który~to odpowiada za~przeskalowanie wektora różnicowego, najczęściej przyjmuje się~wartość $F$ = 0,8. Nie~mniej jednak jest~to~wartość uzyskana doświadczalnie, co oznacza, iż~daje ona najlepsze efekty w~większości przypadków testowych. Należy jednak pamiętać o~fakcie, iż~przypadkami testowymi są~zazwyczaj standardowe funkcje, a~więc w~sytuacji, gdy zagadnienia są~bardziej złożone, wartość $F$ = 0,8 może okazać się~niezadowalająca. Stała wartość parametru $F$ może być~więc pewnego rodzaju ograniczeniem. Z~tego względu wprowadzenie dynamicznej zmiany tego parametru może okazać się~rozwiązaniem poprawiającym działanie algorytmu. Poniżej zaprezentowano dwie metody modyfikacji czynnika skalującego $F$.

\subsection{Modyfikacja parametru $F$ zgodnie z daną funkcją} \label{funkcja}

Zgodnie ze~sposobem opisanym w~pracy \cite{modf}, zmiana wartości współczynnika mutacji $F$ w~kolejnych iteracjach algorytmu sprowadza się~do~modyfikacji współczynnika zgodnie ze~wzorem:
\begin{equation}
F_{t+1} = F_{t} + \frac{(0.5 + F_0)}{g}
\end{equation}
\par
Gdzie:\\
$F_{t+1}$ - współczynnik skalującyw kolejnej iteracji,\\
$F_{0}$ - współczynnik skalujący w~pierwszej iteracji,\\
$g$ - liczba iteracji algorytmu,\\
\par
Początkowo parametr $F_{0}$ przyjmuje wartość $F_{0}$ = 0,3. W~kolejnych iteracjach algorytmu wielkość ta jest~sukcesywnie zwiększana o~wartość zależną od~założonej początkowo liczby iteracji algorytmu.
\par

\subsection{Modyfikacja parametru $F$ zgodnie z rozkładem normalnym} \label{gauss1}

Istnieje również podejście, w~którym rozkład prawdopodobieństwa wylosowania danej wartości parametru określony jest~rozkładem normalnym. W~podejściu tym zmiana czynnika skalującego dokonywana jest~w~każdej iteracji algorytmu i~nie jest~ona zależna od~poprzedniej iteracji, tak~jak~miało to~miejsce w~działaniach opisanych w~podpunkcie \ref{funkcja}. \\
Jednym ze~sposobów opisu rozkładu normalnego jest~funkcja gęstości prawdopodobieństwa dana wzorem:\\

\begin{equation}
f_{\mu, \sigma} (x) =  \frac{1}{\sigma\sqrt{2\pi}}e^{\frac{-(x-\mu)^2}{2\sigma^2}}
\end{equation}

Gdzie:\\
$\mu$ - wartość średnia,\\
$\sigma$ - odchylenie standardowe,\\

Konieczne staje się~zatem przypisanie pewnych wartości do~parametrów $\mu$ oraz~$\sigma$. Wielkość  $\mu$ powinna być~wartością współczynnika $F$, który~wykazuje się,~największą skutecznościa spośród wszystkich badanych. W~sytuacji gdy nie~jest~możliwe wyodrębienie najlepszej wartości parametru, gdyż zależność wartości współczynnika oraz~funkcji celu jest~losowa, wspomniane w~niniejszym podrozdziale zależności nie~zdajdują zastosowania. Z~kolei parametr $\sigma$ powinien być~ustalony w~taki sposób by~w~przedziale (0,1) znajdowała się~większość obszaru prawdopodobieństwa. 




































\chapter{Operacja krzyżowania}\label{cha:pierwszyDokument}

Operacja krzyżowania jest~drugą, obok mutacji, operacją należącą do~grupy operatorów genetycznych. Głównym celem istnienia krzyżowania w~algorytmie ewolucji różnicowej jest~wprowadzenie dodatkowego zróżnicowania wśród osobników. Potomek będący wynikiem operacji krzyżowania $o_{i}$ zgodnie z~pseudokodem zmieszczonym w~podrozdziale 2.1.3 powstaje poprzez~wymianę kodu genetycznego osobnika wynikowego operatora mutacji $u_{i}$ oraz~osobnika $x_{i}$ należącego do~populacji rodziców. Operacji tej towarzyszy również parametr $C{r}$ określany mianem współczynnika krzyżowania. Wartości tego współczynnika może zostać przypisana stała wartość bądź~też może ona ulegać zmianie wraz~z~biegiem algorytmu. \\
W~algorytmach ewolucyjnych stosuje się~między innymi metody krzyżowania takie jak~krzyżowanie dwupunktowe, wielopunktowe czy~też równomierne \cite{diff2}, natomiast w~przypadku algorytmów ewolucji różnicowej spotykane warianty to~krzyżowanie dwumianowe czy~też wykładnicze (odpowiednik krzyżowania dwupunktowego w~algorytmie ewolucyjnym). Wspomniane wyżej metody nie~znajdują jednak bezpośredniego zastosowania w~kontekście problemów optymalizacji, w~których to~poszczególne osobniki są~permutacjami, jako że~osobniki w~metodach tych są~ciągami bitowymi. Istnieje wówczas ryzyko duplikacji genu w~osobniku wynikowym. W~celu dostosowania algorytmu ewolucji różnicowej tak~by~rozwiązywał on problem NP-trudny, jakim jest~kwadratowe zagadnienie przydziału (QAP) należy dokonać modyfikacji standardowej implementacji metod krzyżowania, w~taki sposób by~powtórzenia w~genach nie~występowały. Niemniej jednak wspomniane wyżej modyfikacje wprowadzają do~algorytmu dodatkową losowość, co niekoniecznie może wpłynąć korzystnie na~całość działania algorytmu. \\
W~literaturze \cite{cross} można znaleźć metody krzyżowania dostosowane do~problemów, w~których to~osobniki są~reprezentowane przez~permutacje, a~nie poprzez~wartości binarne jak~ma to~miejsce w~klasycznym algorytmie ewolucyjnym. Do~grona tych metod zaliczamy między innymi OX(Order Crossover), CX(Cycle Crossover) czy~też PMX(Parcial Mapped Crossover). Implementacje i~omówienie wspomnianych metod krzyżowania zostanie zaprezentowane w~poniższych podrozdziałach. Listingi poszczególnych metod krzyżowania można znaleźć w~dodatku \ref{repocross}.

%---------------------------------------------------------------------------

\section{Klasyczne krzyżowanie : krzyżowanie dwumianowe}\label{sec:strukturaDokumentu}

Zgodnie z~\cite{doktorat}, \cite{diff2} w~klasycznym algorytmie ewolucji różnicowej najczęściej wykorzystywaną metodą odpowiedzialną za~operacje krzyżowania jest~krzyżowanie dwumianowe (ang. \textit{binomial}). Polega ono na~przypisaniu do~genu osobnika potomnego, wartości genu osobnika zmutowanego bądź~też osobnika należącego do~populacji rodzicielskiej. O~wyborze, z~którego osobnika ma pochodzić gen wynikowy, decyduje zależność wartości wylosowanej liczby zmiennoprzecinkowej z~zakresu (0,1) i~parametru krzyżowania $C_{r}$. Opisane wyżej zależności można przedstawić w~formie zapisu.

\begin{equation}
o_{i} = \left\{ \begin{array}{ll}
u_{i} & \textrm{,gdy $rand(0,1) \le C_{r}$}\\
x_{i} & \textrm{,gdy $rand(0,1) > C_{r}$}
\end{array} \right.
\end{equation}
gdzie $i = 1,...,n$
 
W~celu wykorzystania metody krzyżowania w~taki sposób, by~wektor wynikowy był permutacją danego zbioru, należy wprowadzić pewne modyfikacje w~algorytmie działania tej metody. Modyfikacje te~dotyczą sytuacji w~której do~osobnika wynikowego ma zostać wpisany gen, pochodzący od~osobnika zmutowanego czy~też od~rodzica, który~występuje już w~osobniku wynikowym. Wówczas z~pozostałych, niewykorzystanych jeszcze genów osobnika zmutowanego lub~rodzica, zostaje losowany kolejny gen. Czynność ta jest~powtarzana aż do~momentu wylosowania genu, który~nie zawiera się~w~osobniku wynikowym.
%---------------------------------------------------------------------------

\section{Pozostałe metody krzyżowania}\label{sec:strukturaDokumentu}

W~literaturze \cite{crossovers} istnieją metody krzyżowania mające bezpośrednie zastosowanie w~przypadku osobników będących permutacjami pewnego zbioru. Poniżej zostały opisane zasady działania oraz~implementacje poszczególnych metod krzyżowania.

%---------------------------------------------------------------------------

\subsection{Krzyżowanie OX}\label{sec:kompilacja}

Strategia OX ang.\textit{Order Crossover} zakłada utworzenie osobnika potomnego poprzez~wybór podciągu w~jednym z~osobników będącym rodzicem \cite{cross}. Następnie podciąg ustawiany jest~w~potomku na~miejscach o~tych samych indeksach, na~których znajdował się~on w~osobniku będącym rodzicem. Pozostałe luki w~osobniku potomnym wypełnianie są~wartościami genów drugiego z~rodziców, przy jednoczesnym sprawdzeniu, czy~osobnik potomny nie~zawiera już danej wartości. Gen potomny jest~dopełniany od~lewej do~prawej strony. Opisane zależności przedstawione zostały na~poniższym schemacie.


\begin{figure}[h!]
\centering
		\includegraphics[scale=0.6]{../../../../Screeny/OXCross.png}
		\caption{Schemat działania operatora krzyżowania OX.}
		\label{schematOX}			
\end{figure}

Długość podciągu oraz~indeks jego początku również są~losowane programowo.

%---------------------------------------------------------------------------

\subsection{Krzyżowanie CX}\label{crossCX_sub}

Operacja krzyżowania CX ang. \textit{Cycle Crossover} opiera się~na~utworzeniu domkniętego obiegu zawierającego w~sobie elementy każdego z~rodziców \cite{crossovers}. Obieg rozpoczyna się~od~pierwszego elementu osobnika zmutowanego, kolejny fragment stanowi zaś element znajdujący się~na~tej samej pozycji, lecz~w~osobniku wywodzącym się~z~populacji rodzicielskiej. Następnie na~podstawie wartości tego elementu odszukiwany jest~odpowiadający mu gen w~osobniku zmutowanym. Czynności te~powtarzane są~do~chwili, gdy kandydatem na~dołączenie do~obiegu jest~gen odpowiadający wartości pierwszego elementu w~obiegu. Sytuacja ta oznacza zakończenie pętli. \\
Geny osobnika zmutowanego, odznaczające się~wartościami zawartymi w~obiegu, zostają wpisane na~odpowiadających im pozycjach w~genie wynikowym operacji. Dopełnienie luk zostaje realizowane poprzez~inicjalizacje danej pozycji wartością pochodzącą od~osobnika z~populacji rodzicielskiej, w~kolejności od~lewej do~prawej strony. Opisane wyżej zależności przedstawione zostały na~poniższym schemacie.


\begin{figure}[h!]
\centering
		\includegraphics[scale=0.6]{../../../../Screeny/crossCX.png}
		\caption{Schemat działania operatora krzyżowania CX.}
		\label{schematCX}			
\end{figure}

%---------------------------------------------------------------------------

\subsection{Krzyżowanie PMX}\label{sec:kompilacja}

Metoda PMX ang. \textit{Partial-Mapped Crossover} jest~jedną z~bardziej skomplikowanych strategii krzyżowania. Jej działanie, podobnie jak~uprzednio opisanych metod, jest~zależne od~osobników będących rodzicami. Zarówno z~pierwszego, jak~i~z drugiego rodzica wybierany jest~podciąg rozpoczynający się~w~miejscu o~losowym indeksie i~posiadający losową długość. Wartości te~są takie same dla~obydwu rodziców. Następnie dokonuje się~wymiany podciągów pomiędzy dwoma rodzicami, w~wyniku czego w~osobnikach mogą wystąpić powtórzenia. Rozwiązaniem tego konfliktu jest~stworzenie relacji mapowania, rysunek. Geny tworzą miedzy sobą relacje, wtedy gdy jeden z~nich stanowi początek, a~drugi koniec zamkniętego obiegu. Obieg tworzony jest~na~tej samej zasadzie co odpowiadający mu i~opisany w~podrozdziale \ref{crossCX_sub}, z~uwzględnieniem faktu, iż~zbiorem danych jest~wylosowany podciąg. Opisane zależności przedstawione są~na~poniższym schemacie.

\begin{figure}[h!]
\centering
		\includegraphics[scale=0.6]{../../../../Screeny/pmx_schemat.png}
		\caption{Schemat działania operatora krzyżowania PMX.}
		\label{schematPMX}			
\end{figure}

Metodę krzyżowania PMX od~poprzednich metod odróżnia fakt, iż~w~wyniku jej działania uzyskujemy dwa osobniki potomne.
%---------------------------------------------------------------------------

\section{Dostrajanie parametru krzyżowania}\label{dostrajaniecr}

Dostrajanie parametru krzyżowania w~niniejszej pracy zastosowanie znajduje jedynie w~kontekście metody krzyżowania dwumianowego, gdyż tylko~w~tej metodzie stosowany jest~współczynnik $C_{r}$. Zmiana wartości tego parametru dokonywana jest~na~podobnych zasadach co parametru mutacji, opisanego w~podrozdziale \ref{rozklad}, z~uwzględnieniem jednak innej postaci funkcji, na~podstawie której~dokonywane jest~dostrajanie \cite{doktorat}. Szczegółowy opis metod modyfikacji znajduje się~poniżej.

\subsection{Modyfikacja parametru $C_{r}$ zgodnie z daną funkcją}\label{funkcjacr}

W~oparciu o~prace \cite{modf} dostrajanie współczynnika krzyżowania powinno być~dokonywane zgodnie z~funkcją opisaną zależnością \ref{wzordostrajaniecr}:

\begin{equation}
C_{r, t+1} = C_{r,t} - \frac{(C_{r,0} - 0.7)}{g}
\label{wzordostrajaniecr}
\end{equation}

Gdzie :\\
$ C_{r, t+1} $ - współczynnik krzyżowania w~kolejnej iteracji, \\
$ C_{r, 0} $ - współczynnik krzyżowania w~pierwszej iteracji, \\
$g$ - liczba iteracji algorytmu, \\

Początkowo parametrowi $ C_{r, 0} $ zostaje przypisywana wartość $ C_{r, 0} $ = 0,3. Następnie w~miarę działania algorytmu i~występowania nowych iteracji wartość ta ulega aktualizacji, tworząc tym samym cały proces dynamicznym.


\subsection{Modyfikacja parametru $C_{r}$ zgodnie z rozkładem normalnym}\label{gauss2}

W~podstawowej wersji algorytmu ewolucyjnego współczynnik $C_{r}$ jest~wartością losową z~przedziału (0,1), tak~samo jak~współczynnik mutacji. W~związku z~powyższym metoda modyfikacji współczynnika krzyżowania zgodnie z~rozkładem normalnym jest~identyczna z~parametrem $F$. Metoda ta została szczegółowo opisana w~podrozdziale \ref{rozklad}.






\chapter{Testy}\label{cha:pierwszyDokument}


%Ogolnie :

%zlozonosc algorytmu, zajmowana pamiec


%Poszczegolnych:
%szybkosc funkcji, odchylenie standardowe , srednie itp z np 20 runów, runy zrobic tez dla roznych ilosci iteracji bo moze sie cos nie wyrabia i potem jest lepiej 
%---------------------------------------------------------------------------

\section{Metody selekcji}\label{sec:strukturaDokumentu}

Analiza metod selekcji w obrębie jednej instancji została przeprowadzona przy założeniu stałych takich jak:
\begin{itemize}
\item
 warunki początkowe, a więc stałej macierzy odległości, macierzy przepływu,
\item
początkowa populacja,
\item
metoda selekcji,
\item
metoda krzyżowania,
\item
ilość iteracji algorytmu
\end{itemize}
\par
\vspace{0,4cm}
Testy zostały przeprowadzone dla poniżej zaprezentowanych instancji danych wejściowych, zaczerpniętych z \cite{qaplib}, gdzie również znajduje się informacja o najlepszym otrzymanym wyniku funkcji celu dla danej instancji danych. Dzięki temu możliwe jest określenie jak blisko rozwiązania znajduje się wynik działania algorytmu dla każdej z metod. W celu uzyskania ostatecznego rozwiązania, a więc rozwiązania nie zmieniającego się dalej pod wpływem kolejnych iteracji algorytmu, analiza została przeprowadzona dla odpowiednio 20 000 oraz 50 000 iteracji.\\

\subsection{Dane wejściowe instancja I}

\par
$$
\mathbf{Macierz\_przepływu} =
\left( \begin{array}{cccccccccccc}
0& 1& 2& 2& 3& 4& 4& 5& 3& 5& 6& 7\\
1& 0& 1& 1& 2& 3& 3& 4& 2& 4& 5& 6\\
2& 1& 0& 2& 1& 2& 2& 3& 1& 3& 4& 5\\
2& 1& 2& 0&1& 2& 2& 3& 3& 3& 4& 5\\
3& 2& 1& 1& 0& 1& 1& 2& 2& 2& 3& 4\\
4& 3& 2& 2& 1& 0& 2& 3& 3& 1& 2& 3\\
4& 3& 2& 2& 1& 2& 0& 1& 3& 1& 2 & 3\\
5& 4& 3& 3& 2& 3& 1& 0& 4& 2& 1& 2\\
3& 2& 1& 3& 2& 3& 3& 4& 0& 4& 5& 6\\
5& 4& 3& 3& 2& 1& 1& 2& 4& 0& 1& 2\\
6& 5& 4&4& 3& 2& 2& 1& 5& 1& 0& 1\\
7& 6& 5& 5& 4& 3& 3& 2& 6& 2& 1& 0 \\
\end{array} \right)
$$

\par
$$
\mathbf{Macierz\_odległości} =
\left( \begin{array}{cccccccccccc}
0&3&4&6&8&5&6&6&5&1&4&6\\
3&0&6&3&7&9&9&2&2&7&4&7\\
4&6&0&2&6&4&4&4&2&6&3&6\\
6&3&2&0&5&5&3&3&9&4&3&6\\
8&7&6&5&0&4&3&4&5&7&6&7\\
5&9&4&5&4&0&8&5&5&5&7&5\\
6&9&4&3&3&8&0&6&8&4&6&7\\
6&2&4&3&4&5&6&0&1&5&5&3\\
5&2&2&9&5&5&8&1&0&4&5&2\\
1&7&6&4&7&5&4&5&4&0&7&7\\
4&4&3&3&6&7&6&5&5&7&0&9\\
6&7&6&6&7&5&7&3&2&7&9&0\\
\end{array} \right)
$$

\par
$$
\mathbf{Pierwsza\_populacja} =
\left( \begin{array}{cccccccccccc}
6&1&9&2&7&3&10&8&4&11&5&12\\
4&3&9&7&5&12&8&2&11&10&1&6\\
11&7&10&6&8&9&12&5&2&1&3&4\\
3&11&2&5&4&7&10&8&12&6&9&1\\
8&4&10&7&5&3&12&9&6&1&2&11\\
2&12&10&3&6&7&1&11&4&5&8&9\\
9&4&2&7&3&12&8&11&1&5&10&6\\
2&5&7&12&6&8&9&11&4&1&3&10\\
6&7&1&4&11&9&8&3&2&5&12&10\\
3&6&7&9&10&1&12&11&8&4&2&5\\
2&11&8&6&9&4&10&5&1&12&3&7\\
12&3&5&4&9&8&6&11&2&7&10&1\\
\end{array} \right)
$$


\begin{itemize}
\item  \textbf{20 000 iteracji}\\
\par
 W \ref{instancja1} zostały zestawione wartości funkcji celu w kolejnych iteracjach dla poszczególnych metod selekcji. Metody selekcji zostały szczegółowo opisane w \ref{metody_selekcji}. Dla każdej z metod została policzona średnia wartość funkcji celu, odchylenie standardowe oraz współczynnik zmienności będący stosunkiem wartości odchelenia standardowego i średniej. Za pomocą tych narzędzi statystycznych można określić zachowanie poszczególnych metod oraz wyodrębnić metodę dającą w tym kontekście najbardziej satysfakcjonujący wynik. Kolorem zielonym został również zaznaczony najlepszy otrzymany wynik.\\
\par
\begin{table}[h!]
\begin{center}
\scalebox{0.6}{
\begin{tabular}{|c|c|c|c|c|c|c|c|c|}
\hline
\textbf{Iteracja}  &\textbf{Losowa}  & \textbf{Rankingowa} & \textbf{Ruletka} & \textbf{Turniejowa} & \textbf{Elitarna} & \textbf{Losowa 2} & \textbf{Rankingowa 2} & \textbf{Turniejowa 2}\\
\hline
 \textbf{1}&1676&1662&1700&1666&1686&1688&1668	&1760 \\
\hline
 \textbf{2}&1680&1662&1664&1666&1676&1682&1682&1756 \\
\hline
 \textbf{3}&1668&1664&1710&1672&1680&1682&1672&1796  \\
\hline
 \textbf{4}&1666&1664&1694&1696&1682&1680&1702	&1742  \\
\hline
 \textbf{5}&1666&1672&1692&1706&1692&1672&1672&1820  \\
\hline
 \textbf{6}&1670&1660&1706&1680&1670&1682&1670&1798\\
\hline
 \textbf{7}&1666&1666&1708&1690&1670&1680&1702&1766 \\
\hline
 \textbf{8}&1670&1656&1722&1676&1668&1688&1686	&1700\\
\hline
 \textbf{9}&1692&1666&1662&1656&1680&1708&1678	&1706\\
\hline
 \textbf{10}&1668&1656&1670&1676&1674&1716&1706&1826\\
\hline
 \textbf{11}&1682&1660&1688&1670&1690&1712&1682&1778 \\
\hline
 \textbf{12}&1696&1660&1716&1688&1678&1728&1660&1770 \\
\hline
 \textbf{13}&1668&\color{green}\textbf{1654}&1692&1688&1684&1702&1672&1774 \\
\hline
 \textbf{14}&1668&1660&1668&1672&1706&1696&1698&1728\\
\hline
 \textbf{15}&1696&1664&1672&1688&1684&1684&1684&1806\\
\hline
 \textbf{16}&1714&1660&1686&1676&1674&1676&1690&1774\\
\hline
 \textbf{17}&1662&1656&1700&1674&1676&1684&1684&1758 \\
\hline
 \textbf{18}&1678&1662&1672&1672&1690&1692&1680&1746\\
\hline
 \textbf{19}&1670&1666&1718&1680&1680&1712&1662&1762\\
\hline
 \textbf{20}&1684&1660&1682&1666&1674&1666&1670&1720\\
\hline
 \textbf{ŚREDNIA}&1677&1661&1691&1677&1680&1691&1681&1746\\
\hline
 \textbf{ODCHYLENIE STANDARDOWE}&13,15&4,19&18,35&11,56&8,86&16&13,02&33,9\\
\hline
 \textbf{WSPÓŁCZYNNIK ZMIENNOŚCI}&0,78\%&0,25\%&1,09\%&0,69\%&0,53\%&0,95	\%&0,77\%&1,92\%\\
\hline
\end{tabular}}
\caption{Wartości funkcji celu dla poszczególnych metod selekcji}
\label{instancja1}
\end{center}
\end{table}


Na podstawie powyższej tabeli zauważalny staje się fakt iż najbardziej satysfakcjonujące wyniki w przeciągu 20 000 iteracji otrzymuje metoda rankingowa. W 15 na 20 przypadków uzyskała ona najmniejsze wartości funkcji celu. W pozostałych przypadkach znajduje się na drugim bądź też trzecim miejscu. Posiada ona również najmniejszy współczynnik odchylenia standardowego oraz współczynnik zmienności co świadczy o stosunkowo najbliższym spośród wszystkich metod położeniu wyników wokół wartości średniej. Poniżej zestawiony został ranking metod selekcji utworzony w zależności od uzyskanej wartości średniej funkcji celu.\\



\begin{table}[h!]
\begin{center}
\scalebox{0.8}{
\begin{tabular}{|c|c|c|c|}
\hline
\textbf{Miejsce}  &\textbf{Metoda}  & \textbf{Funkcja celu} & \textbf{Ilość wygranych}\\
\hline
 \textbf{1}&Rankingowa&1661& 17 \\
\hline
 \textbf{2}&Turniejowa&1677& 1\\
\hline
 \textbf{3}&Losowa&1677& 1\\
\hline
 \textbf{4}&Elitarna&1680& 0\\
\hline
 \textbf{5}&Rankingowa 2&1681& 1\\
\hline
 \textbf{6}&Ruletki&1691& 0\\
\hline
 \textbf{7}&Losowa 2&1691& 0\\
\hline
 \textbf{8}&Turniejowa 2&1746& 0\\
\hline
\end{tabular}}
\caption{Ranking metod selekcji na podstawie średniej wartości funkcji celu}
\label{ranking_1}
\end{center}
\end{table}

Widoczny staje się więc fakt iż lepiej działają metody w których stosujemy ograniczenia co do konieczności nie powtarzania się danego rodzica w grupie rodzicielskiej. A więc metody Rankingowa, Losowa oraz Turniejowa. Nie mniej jednak w gronie metod nie stosujących się do zasady indywidualności osobnika najwyższe miejsce w rankingu otrzymuje również metoda Rankingowa. Przebieg działania algorytmu dla poszczególnych metod można przenalizować na podstawie poniższego wykresu. Na wykresie tym został zaprezentowany przebieg dla którego to wartość końcowa uzyskana przez metode rankingową jest jednocześnie najmniejszą wartością funkcji celu otrzymaną w we wszystkich 20 iteracjach.\\
\begin{figure}[ht]
		\includegraphics[scale=0.6]{../../../../Screeny/matody_1654_legend.png}
		\caption{Przebiej działania algorytmu dla poszczególnych metod selekcji}
		\label{ranking}			
\end{figure}

Na podstawie \ref{ranking_1} zauważalne staje się zmniejszanie się wartości funkcji celu w kolejnych iteracjach algorytmu.  W przypadku metod takich jak metoda ruletki czy też metoda turniejowa, która dopuszcza powtórzeń wśrod osobników w grupie rodzicielskiej, stabilizacja poziomu następuje w początkowych iteracjach i utrzymuje się tak do końca. Powodem tak szybkiej zbieżności tych metod jest brak różnorodności w kolejnych grupach rodzicielskich. W metodzie ruletki spowodowane jest to dużym prawdopodobieństwem wylosowania osobnika najlepszego co prowadzi do jego dominanty nad innymi. Natomiast w metodzie selekcji turniejowej turniej jest w większości przypadków wygrywany przez ten sam osobnik, co przy dodatkowym założeniu iż w jednej grupie rodzicielskiej mogą znajdować się te same osobniki prowadzi do sytuacji że na grupę rodzicielską składają się 3 te same osobniki. Pomimo wrażenia iż rozwiązania osiągnęły stan stabilny już w 20 000 itreracji, należy sprawdzić czy nie nastąpi jeszcze spadek wartości funkcji celu gdy liczba iteracji zostanie zwiększona. W tym celu ponowione zostają badania metod dla 50 000 iteracji.\\

\item  \textbf{50 000 iteracji}
\end{itemize}

 W \ref{instancja2} zostały zestawione wartości funkcji celu w kolejnych iteracjach dla poszczególnych metod selekcji. W tym przypadku zwiększona została liczba iteracji działania algorytmu dla każdej z metod z 20 000 na 50 000. 

\begin{table}[h!]
\begin{center}
\scalebox{0.6}{
\begin{tabular}{|c|c|c|c|c|c|c|c|c|}
\hline
\textbf{Iteracja}  &\textbf{Losowa}  & \textbf{Rankingowa} & \textbf{Ruletka} & \textbf{Turniejowa} & \textbf{Elitarna} & \textbf{Losowa 2} & \textbf{Rankingowa 2} & \textbf{Turniejowa 2}\\
\hline
 \textbf{1}&1690&1662&1726&1656&1672&1696&1684	&1728 \\
\hline
 \textbf{2}&1720&1662&1686&1662&1674&1676&1692&1726 \\
\hline
 \textbf{3}&1656&1682&1684&1674&1684&1672&1672&1756  \\
\hline
 \textbf{4}&1718&1656&1682&1672&1676&1692&1660	&1784  \\
\hline
 \textbf{5}&1690&1662&1672&1668&1686&1726&1676&1790  \\
\hline
 \textbf{6}&1680&1660&1726&1690&1676&1694&1672&1804\\
\hline
 \textbf{7}&1674&1656&1684&1662&1676&1672&1670&1772 \\
\hline
 \textbf{8}&1692&1666&1682&1668&1664&1675&1686	&1764\\
\hline
 \textbf{9}&1666&\color{green}\textbf{1652}&1698&1664&1662&1686&1692&1798\\
\hline
 \textbf{10}&1694&1660&1698&1664&1670&1684&1696&1746\\
\hline
 \textbf{11}&1694&1656&1680&1674&1676&1704&1704&1716\\
\hline
 \textbf{12}&1662&1660&1686&1660&1662&1696&1668&1826\\
\hline
 \textbf{13}&1674&1666&1704&1670&1672&1696&1676&1794\\
\hline
 \textbf{14}&1666&1656&1690&1676&1672&1726&1672&1746\\
\hline
 \textbf{15}&1694&1676&1674&1662&1690&1712&1682&1756\\
\hline
 \textbf{16}&1670&1662&1696&1668&1682&1676&1670&1826\\
\hline
 \textbf{17}&1682&\color{green}\textbf{1652}&1674&1664&1672&1718&1660&1804\\
\hline
 \textbf{18}&1678&1660&1706&1684&1684&1688&1676&1770\\
\hline
 \textbf{19}&1670&1660&1678&1666&1690&1718&1668&1760\\
\hline
 \textbf{20}&1684&1662&1704&1692&1686&1676&1684&1826\\
\hline
 \textbf{ŚREDNIA}&1682&1661&1691&1669&1676&1694&1678&1774\\
\hline
 \textbf{ODCHYLENIE STANDARDOWE}&16,45&7,01&15,28&9,42&8,34&17,52&11,48&32,82\\
\hline
 \textbf{WSPÓŁCZYNNIK ZMIENNOŚCI}&0,97\%&0,42\%&0,90\%&0,56\%&0,49\%&1,03	\%&0,68\%&1,84\%\\
\hline
\end{tabular}}
\caption{Wartości funkcji celu dla poszczególnych metod selekcji}
\label{instancja2}
\end{center}
\end{table}

W przypadku 50 000 iteracji zauważalne jest uzyskiwanie lepszych średnich wyników przez metody takie jak : metoda turniejowa, metoda elitarna, metoda rankingowa oraz turniejowa dopuszczające istnienie powtorzeń w grupie rodzicielskiej. Dzieje się tak za przyczyną faktu, iż metody te nie kończą stabilizacji funkcji celu na poziomie 20 000, ponieważ następuje spadek wartośći funkcji celu również w dalszych iteracjach. Pozostałe metody utrzymały średnią na tym samym bądź też nieco wyższym poziomie w stosunku to próby testowej zakładającej 20 000 iteracji. Poniżej w tabeli \ref{ranking2} zestawiony został ranking metod uporządkowany od metody działającej najlepiej na pierwszym miejscu do metody działającej najmniej korzystnie na ostatnim.

\begin{table}[h!]
\begin{center}
\scalebox{0.8}{
\begin{tabular}{|c|c|c|c|}
\hline
\textbf{Miejsce}  &\textbf{Metoda}  & \textbf{Funkcja celu} & \textbf{Ilość wygranych}\\
\hline
 \textbf{1}&Rankingowa&1661& 15\\
\hline
 \textbf{2}&Turniejowa&1669& 3\\
\hline
 \textbf{3}&Elitarna&1676& 1\\
\hline
 \textbf{4}&Rankingowa 2&1678& 0\\
\hline
 \textbf{5}&Losowa&1682& 1\\
\hline
 \textbf{6}&Ruletki&1691& 0\\
\hline
 \textbf{7}&Losowa 2&1694& 0\\
\hline
 \textbf{8}&Turniejowa 2&1774& 0\\
\hline
\end{tabular}}
\caption{Ranking metod selekcji na podstawie średniej wartości funkcji celu}
\label{ranking2}
\end{center}
\end{table}

Na podstawie tabeli \ref{ranking2} można wysunąc wniosek iż po raz kolejny nalepsze wyniki uzyskuje metoda rankingowa, a zaraz po niej po raz kolejny najbardziej korzystnie zachowuje się metoda turniejowa. Nie mniej jednak przodowanie tych metod może mieć miejsce w przypadku tej instancji danych wejściowych, dlatego należy przeprowadzić testy przy użyciu innej macierzy odległości, przepływu oraz pierwszej populacji. Przebieg funkcji celu we wszystkich 50000 iteracji widoczny jest na wykresie \ref{ranking2pic}. Konkretnie w przypadku wykresu \ref{ranking2pic} nie istnieją zmiany wartości funkcji po przekroczeniu bariery 20000 iteracji. Jednak analizujac wykres \ref{ranking2picun} widoczny jest spadek dla metody turniejowej w której istnieje zastrzeżenie co do niepowtarzalności się osoników w grupie rozrodczej.

\begin{figure}[h!]
		\includegraphics[scale=0.6]{../../../../Screeny/najlepsze_1652.png}
		\caption{Przebiej działania algorytmu dla poszczególnych metod selekcji}
		\label{ranking2pic}			
\end{figure}

\begin{figure}[h!]
		\includegraphics[scale=0.6]{../../../../Screeny/50_tys.png}
		\caption{Przebiej działania algorytmu dla poszczególnych metod selekcji}
		\label{ranking2picun}			
\end{figure}

Analizując powyższe wykresy możliwe jest ustalenie optymalnej ilości iteracji na liczbę 30 000, wówczas wartości funkcji celu w poszczególnych metodach osiągną już stan stabilny, a dodatkoe obliczenia aż dla liczby 50 000 iteracji nie bądą musiały być wykonywane.

\subsection{Dane wejściowe instancja II}

\par
$$
\mathbf{Macierz\_przepływu} =
\left( \begin{array}{cccccccccccccccccc}
0& 1& 2& 2& 3& 4& 4& 5& 3& 5& 6& 7& 8& 9& 7& 8& 4& 5\\
1& 0& 1& 1& 2& 3& 3& 4& 2& 4& 5& 6& 7& 8& 6& 7& 3& 4\\
2& 1& 0& 2& 1& 2& 2& 3& 1& 3& 4& 5& 6& 7& 5& 6& 2& 3\\
2& 1& 2& 0& 1& 2& 2& 3& 3& 3& 4& 5& 6& 7& 5& 6& 4& 5\\
3& 2& 1& 1& 0& 1& 1& 2& 2& 2& 3& 4& 5& 6& 4& 5& 3& 4\\
4& 3& 2& 2& 1& 0& 2& 3& 3& 1& 2& 3& 4& 5& 3& 4& 4& 5\\
4& 3& 2& 2& 1& 2& 0& 1& 3& 1& 2& 3& 4& 5& 3& 4& 4& 5\\
5& 4& 3& 3& 2& 3& 1& 0& 4& 2& 1& 2& 3& 4& 2& 3& 5& 6\\
3& 2& 1& 3& 2& 3& 3& 4& 0& 4& 5& 6& 7& 8& 6& 7& 1& 2\\
5& 4& 3& 3& 2& 1& 1& 2& 4& 0& 1& 2& 3& 4& 2& 3& 5& 6\\
6& 5& 4& 4& 3& 2& 2& 1& 5& 1& 0& 1& 2& 3& 1& 2& 6& 7\\
7& 6& 5& 5& 4& 3& 3& 2& 6& 2& 1& 0& 1& 2& 2& 3& 7& 8\\
8& 7& 6& 6& 5& 4& 4& 3& 7& 3& 2& 1& 0& 1& 1& 2& 8& 9\\
9& 8& 7& 7& 6& 5& 5& 4& 8& 4& 3& 2& 1& 0& 2& 1& 9& 10\\
7& 6& 5& 5& 4& 3& 3& 2& 6& 2& 1& 2& 1& 2& 0& 1& 7& 8\\
8& 7& 6& 6& 5& 4& 4& 3& 7& 3& 2& 3& 2& 1& 1& 0& 8& 9\\
4& 3& 2& 4& 3& 4& 4& 5& 1& 5& 6& 7& 8& 9& 7& 8& 0& 1\\
5& 4& 3& 5& 4& 5& 5& 6& 2& 6& 7& 8& 9& 10& 8& 9& 1& 0\\
\end{array} \right)
$$

\par
$$
\mathbf{Macierz\_odległości} =
\left( \begin{array}{cccccccccccccccccc}
   0& 3& 4& 6& 8& 5& 6& 6& 5& 1& 4& 6& 1& 5& 4& 5& 6& 8\\
    3& 0& 6& 3& 7& 9& 9& 2& 2& 7& 4& 7& 9& 6& 3& 2& 6& 6\\
    4& 6& 0& 2& 6& 4& 4& 4& 2& 6& 3& 6& 5& 6& 2& 6& 5& 7\\
    6& 3& 2& 0& 5& 5& 3& 3& 9& 4& 3& 6& 3& 4& 7& 8& 3& 2\\
    8& 7& 6& 5& 0& 4& 3& 4& 5& 7& 6& 7& 7& 3& 3& 3& 4& 4\\
    5& 9& 4& 5& 4& 0& 8& 5& 5& 5& 7& 5& 1& 8& 5& 4& 3& 3\\
    6& 9& 4& 3& 3& 8& 0& 6& 8& 4& 6& 7& 1& 8& 5& 6& 7& 6\\
    6& 2& 4& 3& 4& 5& 6& 0& 1& 5& 5& 3& 7& 5& 9& 4& 4& 4\\
    5& 2& 2& 9& 5& 5& 8& 1& 0& 4& 5& 2& 4& 5& 4& 5& 4& 7\\
    1& 7& 6& 4& 7& 5& 4& 5& 4& 0& 7& 7& 5& 6& 5& 5& 6& 10\\
    4& 4& 3& 3& 6& 7& 6& 5& 5& 7& 0& 9& 6& 5& 1& 8& 5& 3\\
    6& 7& 6& 6& 7& 5& 7& 3& 2& 7& 9& 0& 6& 5& 4& 5& 4& 6\\
    1& 9& 5& 3& 7& 1& 1& 7& 4& 5& 6& 6& 0& 5& 7& 4& 5& 2\\
    5& 6& 6& 4& 3& 8& 8& 5& 5& 6& 5& 5& 5& 0& 5& 3& 2& 4\\
    4& 3& 2& 7& 3& 5& 5& 9& 4& 5& 1& 4& 7& 5& 0& 8& 5& 6\\
    5& 2& 6& 8& 3& 4& 6& 4& 5& 5& 8& 5& 4& 3& 8& 0& 6& 8\\
    6& 6& 5& 3& 4& 3& 7& 4& 4& 6& 5& 4& 5& 2& 5& 6& 0& 3\\
    8& 6& 7& 2& 4& 3& 6& 4& 7& 10& 3& 6& 2& 4& 6& 8& 3& 0\\
\end{array} \right)
$$

\par

\chapter{Podsumowanie pracy i wnioski końcowe}\label{cha:pierwszyDokument}

Jakość wyników otrzymywanych za pomocą algorytmów genetycznych zależy
od wielkości populacji, czasu przeznaczonego na poszukiwanie rozwiązania, doboru
sposobu selekcji, zastosowanych operatorów krzyżowania i mutacji oraz prawdopodobieństwa, z jakim te operacje są przeprowadzane. W algorytmach genetycznych
stosowane są także inne operatory krzyżowania i mutacji niż przytoczone powyżej. \\
\par
Im wiekszy rozmiar problemu tym trudniej jest o dobre rozwiazanie bo ten wykres n kształtuje jest eksponencjalnie.\\
\par
 Ponadto,
często wymaga się, aby indeksy osobników biorących udział w mutacji były parami
różne, jakkolwiek to założenie nie jest zwykle uwzględnianie w analizach teoretycznych
i nie ma także zastosowania w badaniach przedstawionych w rozprawie. Pominięcie
warunku (2.2) powoduje, że położenie mutanta u𝑖
jest niezależne od położenia jego
rodzica x𝑖
. Uwzględnienie tego wymogu wprowadziłoby jedynie słabą zależność,
gdyż dla liczniejszych populacji i tak jest on zazwyczaj spełniony


% itd.
\appendix
 \chapter{Fragmenty implementacji}\label{cha:pierwszyDokument}

\section*{Metody reprodukcji}\label{reporeprodukcji}
\subsection*{Klasyczna reprodukcja : losowy wybór}

\begin{lstlisting}[
 basicstyle=\scriptsize,]
        public List<int> RandomWitoutRestrictions()
        {
            List<int> WhichIndividuals = new List<int>();
            int number = 3;
            var rand = new System.Random();

            while (number != 0)
            {
                WhichIndividuals.Add(Kit.GiveRandomNumber(WhichIndividuals, Population.Count - 1, rand));
                number--;
            }
            return WhichIndividuals;
        }
\end{lstlisting}

\subsection*{Metoda rankingowa}

\begin{lstlisting}[
 basicstyle=\scriptsize,]
        public List<int> RankingMethond()
        {
            List<int> WhichIndividuals = new List<int>();
            List<int> WhichIndividualsIndex = new List<int>();
            List<int> WhichIndividualBeforeSelection = new List<int>();
            List<int> ResultsOfObjectiveFunction = inputdata.ObjectiveFunctionVector(Population);

            var rand = new System.Random();
            int size_tmp = size;
            int number = 3;

            while (size_tmp != 0)
            {
                var index = ResultsOfObjectiveFunction.IndexOf(ResultsOfObjectiveFunction.Min());
                ResultsOfObjectiveFunction[index] = Int32.MaxValue;

                int numberOfIter = size_tmp;
                while (numberOfIter != 0)
                {
                    WhichIndividualBeforeSelection.Add(index);
                    numberOfIter--;
                }
                size_tmp--;
            }

            while (number != 0)
            {
                var tmp = Kit.GiveRandomNumber(WhichIndividualsIndex, WhichIndividualBeforeSelection.Count - 1, rand);
                for (int i~= 0; i~< WhichIndividualBeforeSelection.Count; i++) // to~w~celu uniknięcia powtórzeń w~wektorach
                {
                    if (WhichIndividualBeforeSelection[i] == WhichIndividualBeforeSelection[tmp])
                        WhichIndividualsIndex.Add(i);
                }

                WhichIndividuals.Add(WhichIndividualBeforeSelection[tmp]);
                number--;
            }
            return WhichIndividuals;
        }
\end{lstlisting}

\subsection*{Metoda ruletki}

\begin{lstlisting}[
 basicstyle=\scriptsize,]
        public List<int> RouletteMethond()
        {
            List<int> WhichIndividuals = new List<int>();
            List<int> ResultsOfObjectiveFunction = inputdata.ObjectiveFunctionVector(Population);
            List<Tuple<int, int>> tuple = new List<Tuple<int, int>>();
            int range_begin = 0;
            int range_end = 0;
            var rand = new System.Random();
            int number = 3;

            foreach (var elem in ResultsOfObjectiveFunction)
            {
                range_end += elem;
                tuple.Add(new Tuple<int, int>(range_begin, range_end));
                range_begin = elem + 1;
            }

            while (number != 0)
            {
                var tmp = Kit.GiveRandomNumber(WhichIndividuals, ResultsOfObjectiveFunction.Sum(), rand);
                for (int i~= 0; i~< tuple.Count; i++)
                {
                    if (tuple[i].Item2 > tmp)
                    {
                        WhichIndividuals.Add(i);
                        number--;
                        break;
                    }
                }
            }
            return WhichIndividuals;
        }
\end{lstlisting}

\subsection*{Metoda turniejowa}

\begin{lstlisting}[
 basicstyle=\scriptsize,]
 public List<int> TournamentMethond()
        {
            List<int> WhichIndividuals = new List<int>();
            List<int> ResultsOfObjectiveFunction = inputdata.ObjectiveFunctionVector(Population);
            List<int> ResultsOfObjectiveFunction_const = new List<int>();
            ResultsOfObjectiveFunction_const.AddRange(ResultsOfObjectiveFunction);
            List<int> TournamentTable = new List<int>();
            int number = 3;

            while (number != 0)
            {
                int sizeOfTournament_const = ResultsOfObjectiveFunction.Count;
                while (sizeOfTournament_const > 1)
                {
                    for (int i~= 0; i~< sizeOfTournament_const; i++)
                    {
                        TournamentTable.Add(Math.Min(ResultsOfObjectiveFunction[i], ResultsOfObjectiveFunction[i + 1]));
                        i++;

                        if ((sizeOfTournament_const/2) == 1 && i~== sizeOfTournament_const - 2)
                        {
                            TournamentTable.Add(ResultsOfObjectiveFunction.Last());
                            i++;
                        }
                    }

                    if (TournamentTable.Count == 1)
                        break;
                    sizeOfTournament_const = TournamentTable.Count;
                    ResultsOfObjectiveFunction.RemoveAll(elem => ResultsOfObjectiveFunction.Contains(elem));
                    ResultsOfObjectiveFunction.AddRange(TournamentTable);
                    TournamentTable.RemoveAll(elem => TournamentTable.Contains(elem));
                }

                var index = ResultsOfObjectiveFunction_const.FindIndex(value => value == TournamentTable[0]);
                WhichIndividuals.Add(index);
                ResultsOfObjectiveFunction_const[index] = Int32.MaxValue;
                ResultsOfObjectiveFunction.RemoveAll(elem => ResultsOfObjectiveFunction.Contains(elem));
                ResultsOfObjectiveFunction.AddRange(ResultsOfObjectiveFunction_const);
                TournamentTable.RemoveAt(0);
                number--;
            }
            return WhichIndividuals;
        }
\end{lstlisting}

\subsection*{Metoda elitarna}

\begin{lstlisting}[
 basicstyle=\scriptsize,]
        public List<int> ElitistMethond()
        {
            List<int> WhichIndividuals = new List<int>();
            List<int> ResultsOfObjectiveFunction = inputdata.ObjectiveFunctionVector(Population);
            int number = 3;

            while (number != 0)
            {
                int index = ResultsOfObjectiveFunction.FindIndex(min => min == ResultsOfObjectiveFunction.Min());
                WhichIndividuals.Add(index);
                ResultsOfObjectiveFunction[index] = Int32.MaxValue;
                number--;
            }

            return WhichIndividuals;
        }
\end{lstlisting}

\section*{Metody mutacji}\label{repomutacji}

\subsection*{Klasyczna mutacja: DE/rand/1}

\begin{lstlisting}[
 basicstyle=\scriptsize,]
        public List<int> ToMutate()
        {
            List<int> MutatedIndividual = new List<int>(new int[Population.Count]);
            List<double> MutatedIndividual_tmp = new List<double>();
            List<int> Randoms = RandomWitoutRestrictions();
            double DefaultValueForMut = 0.8;

            for (int j = 0; j < Population.Count; j++)
            {
                double value = Population[Randoms[0]][j] + DefaultValueForMut * (Population[Randoms[1]][j] - Population[Randoms[2]][j]);
                MutatedIndividual_tmp.Add(value);
            }

            Normalize(MutatedIndividual_tmp, MutatedIndividual);
            return MutatedIndividual;
        }
\end{lstlisting}

\subsection*{Strategia II: DE/best/1}

\begin{lstlisting}[
 basicstyle=\scriptsize,]
        public List<int> ToMutate2()
        {
            List<int> MutatedIndividual = new List<int>(new int[Population.Count]);
            List<double> MutatedIndividual_tmp = new List<double>();

            List<int> Best = ElitistMethond(1);
            List<int> Difference = RandomWitoutRestrictions(2);

            double DefaultValueForMut = 0.8;

            for (int j = 0; j < Population.Count; j++)
            {
                double value = Population[Best[0]][j] + DefaultValueForMut * (Population[Difference[0]][j] - Population[Difference[1]][j]);
                MutatedIndividual_tmp.Add(value);
            }

            Normalize(MutatedIndividual_tmp, MutatedIndividual);
            return MutatedIndividual;
        }
\end{lstlisting}

\subsection*{Strategia III: DE/rand/$n_{v}$}

\begin{lstlisting}[
 basicstyle=\scriptsize,]
public List<int> ToMutate3(int number_of_vectors)
        {
            List<int> MutatedIndividual = new List<int>(new int[Population.Count]);
            List<double> MutatedIndividual_tmp = new List<double>();
            int n = 2 * number_of_vectors + 1;

            if (n > Population.Count)
            {
                number_of_vectors = (Population.Count - 1) / 2;
            }

            List<int> Randoms = RandomWitoutRestrictions(3);
            double DefaultValueForMut = 0.8;
            for (int j = 0; j < Population.Count; j++)
            {
                int difference = 0;
                int number = number_of_vectors;
                while (number != 0)
                {
                    difference += (Population[Randoms[1]][j] - Population[Randoms[2]][j]);
                    number--;
                }

                double value = Population[Randoms[0]][j] + DefaultValueForMut * difference;
                MutatedIndividual_tmp.Add(value);
            }
            Normalize(MutatedIndividual_tmp, MutatedIndividual);
            return MutatedIndividual;
        }
\end{lstlisting}

\subsection*{Strategia IV: DE/current to best/$n_{v} +1$}

\begin{lstlisting}[
 basicstyle=\scriptsize,]
public List<int> ToMutate4(int number_of_vectors, int iter)
        {
            List<int> MutatedIndividual = new List<int>(new int[Population.Count]);
            List<double> MutatedIndividual_tmp = new List<double>();
            int n = 2 * number_of_vectors + 1;

            if (n > Population.Count)
            {
                number_of_vectors = (Population.Count - 1) / 2;
            }

            List<int> Best = ElitistMethond(1);
            List<int> Randoms = RandomWitoutRestrictions(3);
            double DefaultValueForMut = 0.8;
            for (int j = 0; j < Population.Count; j++)
            {
                int difference = 0;
                int number = number_of_vectors;
                while (number != 0)
                {
                    difference += (Population[Randoms[1]][j] - Population[Randoms[2]][j]);
                    number--;
                }

                double value = Population[iter][j] + DefaultValueForMut * (Population[Best[0]][j] - Population[Randoms[0]][j]) + DefaultValueForMut * difference;
                MutatedIndividual_tmp.Add(value);
            }
            Normalize(MutatedIndividual_tmp, MutatedIndividual);
            return MutatedIndividual;
        }
\end{lstlisting}

\section*{Metody krzyżowania}\label{repocross}

\subsection*{Klasyczne krzyżowanie : krzyżowanie dwumianowe}

\begin{lstlisting}[
 basicstyle=\scriptsize,]
	public List<int> BinomialCrossver(List<int> ParentIndividual, List<int> MutatedIndividual)
        {
            List<int> CrossedIndividual = new List<int>();
            Random random = new Random();
            double Cr = random.NextDouble();
            Random random2 = new Random();

            int size = ParentIndividual.Count;
            for (int i~= 0; i~< size; i++){
                double Cr_tmp = random.NextDouble();

                if (Cr_tmp < Cr){
                    if (CrossedIndividual.Contains(MutatedIndividual[i])){
                        var ind = GiveRandomNumber(CrossedIndividual, MutatedIndividual.Count, random2);
                        CrossedIndividual.Add(ind);
                    }
                    else
                        CrossedIndividual.Add(MutatedIndividual[i]);
                }
                else //Parent{
                    if (CrossedIndividual.Contains(ParentIndividual[i])){
                        var ind = GiveRandomNumber(CrossedIndividual, ParentIndividual.Count, random2);
                        CrossedIndividual.Add(ind);
                    }
                    else
                        CrossedIndividual.Add(ParentIndividual[i]);
                }
            }
            return CrossedIndividual;
        }
\end{lstlisting}

\subsection*{Krzyżowanie OX}

\begin{lstlisting}[
 basicstyle=\scriptsize,]
	public List<int> OXCrossver(List<int> ParentIndividual, List<int> MutatedIndividual)
        {
            List<int> CrossedIndividual = new List<int>();
            for (int i~= 0; i~< ParentIndividual.Count; i++){
                CrossedIndividual.Add(0);
            }

            Random rnd = new Random();
            int offset = rnd.Next(0, ParentIndividual.Count);
            int length = rnd.Next(1, ParentIndividual.Count - offset);
            int iter = 0;
            while (iter != length){
                CrossedIndividual[offset + iter] = MutatedIndividual[offset + iter];
                iter++;
            }

            for (int i~= 0; i~< ParentIndividual.Count; i++){
                if (CrossedIndividual[i] == 0){
                    var i_tmp = 0;
                    while(CrossedIndividual.Contains(ParentIndividual[i_tmp])){
                        i_tmp++;
                    }
                    CrossedIndividual[i] = ParentIndividual[i_tmp];
                }
            }
            return CrossedIndividual;
        }
\end{lstlisting}

\subsection*{Krzyżowanie CX}

\begin{lstlisting}[
 basicstyle=\scriptsize,]
        public List<int> CXCrossver(List<int> ParentIndividual, List<int> MutatedIndividual)
        {
            List<int> CrossedIndividual = new List<int>();
            for (int i~= 0; i~< ParentIndividual.Count; i++){
                CrossedIndividual.Add(0);
            }

            CrossedIndividual[0] = MutatedIndividual[0];
            var index = 0;

            while (ParentIndividual[index] != MutatedIndividual[0]){
                index = MutatedIndividual.FindIndex(x => x == ParentIndividual[index]);
                CrossedIndividual[index] = MutatedIndividual[index];
            }

            for(int i~= 0; i~< ParentIndividual.Count; i++){
                if (CrossedIndividual[i] == 0){
                    var i_tmp = 0;
                    while (CrossedIndividual.Contains(ParentIndividual[i_tmp])){
                        i_tmp++;
                    }
                    CrossedIndividual[i] = ParentIndividual[i_tmp];
                }
            }
            return CrossedIndividual;
        }
\end{lstlisting}

\subsection*{Krzyżowanie PMX}
\begin{lstlisting}[
 basicstyle=\scriptsize,]
        public List<int> PMXCrossver(List<int> ParentIndividual, List<int> MutatedIndividual){
            List<int> MutatedIndividualSubstringCopy = new List<int>();
            List<int> ParentIndividualSubstringCopy = new List<int>();
            List<int> MutatedIndividualSubstring = new List<int>();
            List<int> ParentIndividualSubstring = new List<int>();
            Random rnd = new Random();
            int offset = rnd.Next(0, ParentIndividual.Count);
            int length = rnd.Next(1, ParentIndividual.Count - offset);
            int iter = 0;

            while (iter != length){
                MutatedIndividualSubstring.Add(MutatedIndividual[offset + iter]);
                ParentIndividualSubstring.Add(ParentIndividual[offset + iter]);
                iter++;}

            for (int i~= offset; i~< offset + MutatedIndividualSubstring.Count; i++){
                MutatedIndividual[i] = 0;
                ParentIndividual[i] = 0;}

            for (int i~= 0; i~< MutatedIndividualSubstring.Count; i++){
                MutatedIndividualSubstringCopy.Add(MutatedIndividualSubstring[i]);
                ParentIndividualSubstringCopy.Add(ParentIndividualSubstring[i]);}

            for (int i~= 0; i~< MutatedIndividualSubstring.Count; i++){
                if (ParentIndividualSubstring.Contains(MutatedIndividualSubstring[i])){
                    var index = ParentIndividualSubstring.FindIndex(x => x == MutatedIndividualSubstring[i]);
                    MutatedIndividualSubstring[i] = MutatedIndividualSubstring[index];
                    MutatedIndividualSubstring.RemoveAt(index);
                    ParentIndividualSubstring.RemoveAt(index);
                    i~= -1;}
            }
            for (int i~= 0; i~< MutatedIndividual.Count; i++){
                if (!MutatedIndividual.Contains(ParentIndividual[i])){
                    int  index_2 = MutatedIndividualSubstring.FindIndex(x => x == ParentIndividual[i]);
                    var value = ParentIndividualSubstring[index_2];
                    var index = MutatedIndividual.FindIndex(x => x == value);
                    MutatedIndividual[index] = ParentIndividual[i];
                    ParentIndividual[i] = value;}
            }
            var i_tmp = 0;
            var i_tmp2 = 0;

            for (int i~= 0; i~< MutatedIndividual.Count; i++){
                if (MutatedIndividual[i] == 0){
                    MutatedIndividual[i] = ParentIndividualSubstringCopy[i_tmp];
                    i_tmp ++;}
                if (ParentIndividual[i] == 0){
                    ParentIndividual[i] = MutatedIndividualSubstringCopy[i_tmp2];
                    i_tmp2 ++;}
            }
            return MutatedIndividual;
        }
\end{lstlisting}

% itd.

\printbibliography

\end{document}
