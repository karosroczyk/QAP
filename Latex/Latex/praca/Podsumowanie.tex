\chapter{Podsumowanie pracy i wnioski końcowe}\label{cha:pierwszyDokument}
Algorytm ewolucji różnicowej należący do~grona algorytmów metaheurystycznych pozwala na~znalezienie pewnego rodzaju rozwiązania dla~problemów NP-trudnych. Nierzadko jednak lepszą skutecznością wykazują się~jego zmodyfikowane wersje, uwzględniające pewne charakterystyczne cechy rozwiązywanego problemu, a~więc mogące się~dostosować do~wymaganych potrzeb. W~ten sposób tworzone są~wersję algorytmów dostosowane do~rozwiązywania konkretnie zdefiniowanego problemu, lecz~niekoniecznie wykazujące dużą skuteczność w~przypadku innego rodzaju rozważanego zagadnienia. Proces doboru odpowiednich parametrów czy~też wersji poszczególnych etapów algorytmu jest~zagadnieniem bardzo złożonym. Z~tego względu dużą rolę odgrywają czynniki losowe czy~też decyzje o~charakterze intuicyjnym.\\

Zagadnieniem będącym problemem NP-trudnym w~kontekście którego~możliwe staje się~zastosowanie oraz~przetestowanie skuteczności działania algorytmu ewolucji różnicowej jest~miedzy innymi kwadratowe zagadnienie przydziału (QAP). To~w~kontekście tego zagadnienia w~niniejszej pracy został opracowany zmodyfikowany algorytm ewolucji różnicowej wykazujący się~dużą skutecznością działania. W~wyniku działania algorytmu uzyskiwane są~informacje o~całkowitym koszcie działania systemu oraz~informacje na~temat rozlokowania poszczególnych budynków w~konkretnych lokalizacjach.\\

W~pracy udało się~zrealizować założone cele, a~więc znaleźć wersję zmodyfikowanego algorytmu dostosowanego do~rozwiązywania problemu kwadratowego zagadnienia przydziału. O~skuteczności zmodyfikowanego algorytmu świadczą jego wyniki testów w~konfrontacji z~wynikami osiąganymi przez~klasyczną postać algorytmu zakładającą metodę reprodukcji losowej, mutacje DE/rand/1 oraz~krzyżowanie dwumianowe. Klasyczna wersja uzyskuje średnią wartość błędu względnego funkcji celu równą 1,14\% (tabela 6.15), z~kolei najmniejszą wartością uzyskiwaną przez~zmodyfikowany algorytm, wykorzystujący w~swoim działaniu reprodukcję losową, mutacje DE/current to~best/$n_{v}+1$+$\leftthreetimes$ oraz~krzyżowanie dwumianowe, dla~którego~parametr $C_{r}$ jest~równy 0,25, jest~wartość rzędu 0,05\% (tabela 6.27). Różnica jest~zatem znacząca i~wynosi aż 1,09\%.\\

Badania rozpoczęto od~metod reprodukcji, opisanych szczegółowo w~rozdziale 3, w~celu wyłonienia spośród nich metody działającej najkorzystniej. Na~tym etapie badań największą skutecznością wykazała się~metoda reprodukcji rankingowej, a~średnia wartość błędu względnego wynosiła 0,57\%. Pomimo przewagi działania tej metody nad~podstawową wersją stosowaną w~algorytmach ewolucji różnicowej, a~więc reprodukcja losowa, uzyskany wyniki nie~był w~pełni satysfakcjonujący. Zauważalna stała się~także przewaga działania algorytmów z~użyciem metod reprodukcji, w~których istniał warunek co do~niepowtarzalności się~osobników wchodzących w~skład grupy rodzicielskiej. Pominięcie tego warunku powoduje, że~położenie osobnika jest~niezależne od~położenia jego
rodzica \cite{diff2}. Metody które~nie spełniały postawionego założenia osiągały o~minimum 1\% gorszy wynik średniej wartości błędu względnego. Założenie to~nie jest~jednoznacznie określone w~analizach teoretycznych, zastosowanie go wprowadza, więc~pewnego rodzaju modyfikacje w~stosunku do~klasycznej wersji algorytmu. Metoda ruletki oraz~turniejowa odznaczały się~przedwczesną zbieżnością algorytmu, co oznacza, że~poprawa wartości funkcji celu następowała jedynie w~początkowych iteracjach działania algorytmu i~na stałym poziomie utrzymywała się~już do~końca. Takie zachowanie algorytmu pod~wpływem użycia tych metod spowodowane było tym, że~najlepsze, ale~jeszcze nie~optymalne osobniki dominowały w~populacji, co wykluczało możliwość powstawania w~kolejnych iteracjach różniących się~od~siebie osobników potomnych.\\

Kolejną część testów stanowiły eksperymenty wykonywane na~metodach mutacji. Części tej została poświęcona szczególna uwaga, jako że~operacja mutacji jest~najistotniejszym i~najbardziej wpływowym elementem algorytmu. Dodatkowo metody zostały przetestowane pod~kątem wpływu na~nie współczynnika $\leftthreetimes$. Testy wykazały ewidentną przewagę metod uwzględniających w~swoim działaniu właśnie ten dodatkowy czynnik skalujący $\leftthreetimes$ nad~metodami nie~zakładającymi zastosowania tej modyfikacji (tabela 6.8 oraz~6.12). Badania zostały podzielone na~dwie zasadnicze części. Pierwsza z~nich miała na~celu dokonanie badań podczas, gdy metodą reprodukcji była metoda losowa, druga część zaś zakładała użycie metody rankingowej. Zarówno w~pierwszej, jak~i~w drugiej części testowej najskuteczniejszym działaniem odznacza się~metoda mutacji DE/current to~best/$n_{v}+1$+$\leftthreetimes$, gdzie liczba wektorów różnicowych $n_{v}$ jest~równa 3. Z~przewagą 0,8\% na~tym etapie prowadzenia testów lepsze wyniki uzyskała wersja algorytmu zakładająca użycie metody losowej.\\

Ostatnią testowaną operacją genetyczną była operacja krzyżowania. Na~etapie prowadzenia testów krzyżowania istniały już pewne wersje algorytmu, które~odznaczały się~lepszym działaniem od~pozostałych. Z~tego względu oprócz przeprowadzenia badań klasycznej wersji algorytmu, zostały przeprowadzone również badania z~uwzględnieniem modyfikacji metod reprodukcji oraz~mutacji. Najskuteczniejszą wersją okazał się~być algorytm stosujący metodę reprodukcji losowej, mutacji DE/current to~best/$n_{v}+1$+$\leftthreetimes$, a~więc z~uwzględnieniem działania mnożnika $\leftthreetimes$ oraz~krzyżowania OX. Strategia ta osiągnęła średnią wartość błędu względnego o~wartości 0.06 \% (tabela 6.20). Niemniej jednak strategia zakładająca użycie metody rankingowej również osiąga dobre wyniki, tj. wartość 0.1\%. Różnica pomiędzy uzyskanymi wynikami jest~równa 0,04 \%, co jest~wartością niewielką i~może wynikać jedynie z~działania czynnika losowego, który~w~przypadku algorytmów metaheurystycznych ma duże znaczenie.\\

Został poddany rozważaniom fakt, iż~metody mutacji zawierają współczynnik $F$ skalujący wektor różnicowy, który~w~klasycznej wersji algorytmu ewolucji różnicowej przyjmuje proponowaną w~literaturze wartości $F = 0.8$. Wartości współczynnika były zmieniane zgodnie z~wartością obliczaną na~podstawie funkcji opisanej szczegółowo w~podrozdziale 4.3.1, a~także na~podstawie wartości określanych w~każdej iteracji przez~rozkład normalny. Interesująca i~uzyskująca lepsze wyniki od~wersji, w~której wartość współczynnika $F$ była ustalona na~stałym poziomie $F=0.8$, okazała się~być strategia stosująca dynamiczną zmianę parametrów zgodnie z~funkcją określoną w~podrozdziale 4.3.1 (tabela 6.25). Dodatkowo w~pracy przeprowadzono testy w~celu znalezienia stałej wartości współczynnika $F$, która~dla rozważanego problemu będzie przynosiła bardziej satysfakcjonujące efekty od~algorytmu z~wartością tą ustaloną na~poziomie $F=0.8$. Testy wykazały, iż~dużą skutecznością, tj. średnią wartością błędu względnego równą 0,04\%, wykazała się~strategia zakładająca przypisanie do~czynnika skalującego $F$, wartości 0,25. W~przypadku współczynnika mutacji nie~została wykazana żadna zależność wartości tego współczynnika od~wartości funkcji celu. Na~tej podstawie można wysnuć wniosek, iż~wartość ta jest~zależna bezpośrednio od~rozwiązywanego problemu i~dla różnych instancji danych wejściowych może przynosić odmienne efekty. 
Podobne testy zostały przeprowadzone dla~metod krzyżowania, a~ściślej mówiąc dla~metody krzyżowania dwumianowego, gdyż tylko~ta metoda zawierała współczynnik $C_{r}$, który~w~klasycznej wersji jest~wartością zmiennoprzecinkową losowaną w~każdej iteracji algorytmu w~przedziale (0,1). Pomimo iż~dynamiczna zmiana wartości współczynnika zgodnie z~funkcją 5.3.1, czy~też 5.3.2 pozwoliła na~osiągnięcie lepszych wyników ( odpowiednio 0,1 \% oraz~0,14\%), wyniki te~nadal nie~były na~tyle dobre, by~móc dorównać strategiom algorytmu opisanym powyżej. Zaskakująco dobrym działaniem i~jednym z~lepszych wyników osiągniętych wśród wszystkich przeprowadzonych testów, wykazała się~strategia, która~zakładała użycie reprodukcji losowej, mutacji DE/current to~best/$n_{v}+1$+$\leftthreetimes$ oraz~krzyżowania dwumianowego ze~stałą wartością współczynnika $F$ ustaloną na~poziomie równym 0,25, gdyż wynik średniej wartości błędu względnego funkcji celu był równy 0,05\%. Dodatkowo zauważyć można było, iż~zależność współczynnika $C_{r}$ oraz~wartości funkcji celu była w~przybliżeniu eksponencjalna.
Wprowadzenie dynamicznej zmiany parametrów, zarówno w~operacji mutacji, jak~i~krzyżowania, pozwala uniknąć trudnego etapu strojenia algorytmu. W~literaturze \cite{doktorat}, \cite{diff2} podane są~optymalne wartości tych współczynników, należy jednak pamiętać, że~najczęściej zbiorem testowym jest~standardowy zestaw funkcji opisujących problemy optymalizacyjne. W~przypadku bardziej złożonych zagadnień najczęściej wymagane jest~ustalenie innych wartości parametrów.\\

Po~przeprowadzeniu testów w~obrębie metod reprodukcji, mutacji oraz~krzyżowania możliwe było wyodrębnienie kilku wersji algorytmu, które~wykazały się~największą skutecznością. Są~to~odpowiednio:\\
\begin{enumerate}
\item reprodukcja losowa, mutacja DE/current to best/$n_{v}+1$+$\leftthreetimes$, krzyżowanie dwumianowe, $C_{r}$ = 0.25, F = 0.8
\item reprodukcja rankingowa, mutacja DE/current to best/$n_{v}+1$+$\leftthreetimes$, krzyżowanie OX, F zgodnie z funkcją \ref{funkcja},
\item reprodukcja losowa, mutacja DE/current to best/$n_{v}+1$+$\leftthreetimes$, krzyżowanie OX, F = 0.25,
\item reprodukcja losowa, mutacja DE/current to best/$n_{v}+1$+$\leftthreetimes$, krzyżowanie OX, F = 0.8,
\item reprodukcja losowa, mutacja DE/current to best/$n_{v}+1$+$\leftthreetimes$, krzyżowanie OX, F zgodnie z funkcją \ref{funkcja},
\end{enumerate}
Dalsza część testów została przeprowadzona dla~macierzy o~różnych rozmiarach z~wykorzystaniem 1 wersji algorytmu. Na~podstawie uzyskanych wyników można było wysnuć wniosek, iż~im większy rozmiar macierzy, tym trudniej o~uzyskanie dobrego wyniku. Skuteczność działania algorytmu po~przekroczeniu rozmiaru macierzy równego 20 znacznie spada. Odpowiedni dobór liczby iteracji jest~również kwestią indywidualną każdej z~rozważanych instancji testowych, a~najlepszym ze~sposobów jej doboru są~metody doświadczalne. Wielkość funkcji celu do~której dąży algorytm jest~również czynnikiem w~pewnym stopniu wpływającym na~skuteczność działania algorytmu.

Uzyskanie różniących się~od~siebie wartości błędu względnego dla~macierzy o~tym samym rozmiarze wynika nie~tylko ze~specyfiki konkretnych danych wejściowych, ale~jest spowodowane również faktem, iż~algorytm ewolucji różnicowej jest~algorytmem opartym w~dużej mierze na~czynniku losowym, którego~działania nie~da się~zagwarantować czy~też przewidzieć. Podsumowując, można stwierdzić, iż~jakość wyników otrzymywanych za~pomocą zmodyfikowanego algorytmu ewolucji różnicowej zależy od~rozmiaru macierzy testowej, wartości funkcji celu, do~której dąży algorytm, wielkości populacji, czasu przeznaczonego na~poszukiwanie rozwiązania oraz~doboru sposobu reprodukcji, a~także zastosowanych operatorów mutacji i~krzyżowania.\\
Z~powodu mnogości dostępnych wersji metod i~możliwości modyfikacji algorytmu, w~niniejszej pracy badaniom poddano jedynie niektóre z~nich. Przeprowadzona analiza w~zakresie dostosowania algorytmu ewolucji różnicowej w~celu rozwiązania kwadratowego zagadnienia przydziału stanowią dobry punkt wyjścia do~dalszych rozważań w~tej dziedzinie. Dodatkowe badania w~kierunku adaptacji parametrów mogłyby skutkować ograniczeniem wpływu na~działanie algorytmu czynnika ludzkiego.
