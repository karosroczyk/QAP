\chapter{Wstęp}
Główny cel pracy stanowi opracowanie i~przeanalizowanie poprawności działania zmodyfikowanego algorytmu ewolucji różnicowej przystosowanego do~rozwiązywania problemu kwadratowego zagadnienia przydziału (Quadratic Assigment Problem). Ze~względu na~fakt, iż~QAP należy do~klasy problemów NP-trudnych, a~więc problemów, dla~których nie~istnieje jednoznacznie określony algorytm dający optymalne wyniki, istnieje konieczność opracowania algorytmu przynoszącego satysfakcjonujące efekty w~przypadku konkretnie postawionego zagadnienia. Dobrych wyników optymalizacji można spodziewać się~wówczas gdy opracowany algorytm wykorzystuje specyficzne cechy oraz~regularności rozwiązywanego problemu.\\ 
W~celu uzyskania satysfakcjonujących wyników dla~problemów NP-trudnych, można posłużyć się~metodami metaheurystycznymi, a~więc metodami umożliwiającymi iteracyjne poprawianie wyniku na~podstawie przyjętych uprzednio funkcji oceny. Metody metaheurystyczne w~dużej mierze inspirują się~zjawiskami zachodzącymi w~naturze, z~tego względu omawiając je operuje się~nomenklaturą zaczerpniętą ze~świata biologii. Dają one jednak rozwiązanie przybliżone, co oznacza, iż~nie ma pewności, że~uzyskany wynik jest~minimum globalnym. Jedną z~metod często stosowaną w~celu rozwiązywania problemów optymalizacji globalnej jest~algorytm ewolucji różnicowej. Algorytm ten w~dużej mierze oparty jest~na~losowości oraz~rozwiązaniach o~charakterze intuicyjnym, niemniej jednak dających oczekiwane rezultaty. Niejednokrotnie dokonuje się~modyfikacji tego algorytmu w~zależności od~potrzeb konkretnie rozwiązywanego problemu. Modyfikacje klasycznej wersji algorytmu najczęściej wykazują swoją skuteczność w~przypadku działania w~kontekście jednego wybranego zagadnienia. Natomiast w~przypadku zastosowania ich dla~innych problemów skuteczność ta ewidentnie spada. W~przedmiotowej pracy przeprowadzona zostanie analiza wpływu poszczególnych modyfikacji na~wyniki osiągane przez~algorytm, tak~by~móc wyodrębnić spośród nich strategie osiągające najkorzystniejsze wyniki.\\
Algorytmy wchodzące w~skład metaheurystyk charakteryzują się~posiadaniem rozwiązania będącego wektorem wartości binarnych. Ze~względu na~fakt, iż~algorytm rozważany w~niniejszej pracy ma zostać przystosowany do~optymalizacji problemów QAP, którego~rozwiązanie jest~w~postaci wektora będącego permutacją pewnego zbioru, należy dokonywać modyfikacji klasycznych metod również na~tej płaszczyźnie.\\
Etapy realizacji celu pracy obejmują:\\
\begin{itemize}
\item Zdefiniowanie NP-trudnego problemu, jaki stanowi kwadratowe zagadnienie przydziału (Quadratic Assigment Problem) oraz określenie relacji, jakie zachodzą pomiędzy zmodyfikowanym algorytmem ewolucji różnicowej a problemem, który rozwiązuje. Opracowanie modelu matematycznego opisującego problem QAP i stanowiącego podstawę dalszych rozważań.\\
\item Implementacja podstawowej wersji algorytmu ewolucji różnicowej wraz z wprowadzeniem modyfikacji w algorytmie w celu dostosowania go do rozwiązywania problemu QAP. Modyfikacje dotyczą kwestii opracowania reprezentacji pojedynczego osobnika jako permutacji pewnego zbioru.\\
\item Opracowanie algorytmu ewolucji różnicowej wraz z uwzględnieniem modyfikacjami w obrębie metod reprodukcji, mutacji oraz krzyżowania. Zmiany dotyczą głównie operacji mutacji oraz krzyżowania. Dodatkowo w tych operacjach genetycznych badaniu podlegają odpowiednie współczynniki. \\
\item Badanie skuteczności zaproponowanych wersji zmodyfikowanego algorytmu ewolucji różnicowej oraz porównanie z działaniem podstawowej wersji. Oprócz konfrontacji wyników wersji klasycznej i zmodyfikowanej ważnym czynnikiem jest liczba iteracji, w których algorytm osiąga wartość uznawaną przez bibliotekę \cite{qaplib} za najmniejszą wartość, jaką udało się uzyskać dla konkretnej instancji danych wejściowych.\\ 
\item Przeprowadzanie testów algorytmu dla bardziej oraz mniej złożonych problemów. Rozważane dane wejściowe posiadają inny rozmiar oraz osiągają inne średnie wartości funkcji celu. 
\end{itemize}
Niniejsza praca składa się~z~7 rozdziałów, z~których 5 początkowych zawiera informacje teoretyczne. Opisana została w~nich zarówno klasyczna, jak~i~zmodyfikowana wersja algorytmu ewolucji różnicowej. Rozdział 6 stanowi część badawczą, gdyż został przeznaczony na~testy uprzednio opisanych modyfikacji. Wnioski i~spostrzeżenia, a~także informacje o~najlepiej działających wersjach algorytmu zostały umieszczone w~rozdziale 7.\\
Rozdział 2 zawiera informacje teoretyczne dotyczące problemów optymalizacji zarówno lokalnej, jak~i~globalnej. Został w~nim opisany oraz~przedstawiony schemat algorytmu ewolucji różnicowej. Zawarto także opis nomenklatury dotyczącej algorytmów ewolucyjnych. Drugą część rozdziału stanowi opis oraz~definicja kwadratowego zagadnienia przydziału QAP.\\
Na~rozdział 3 składają się~opisy zasady działania poszczególnych metod reprodukcji rozważanych w~pracy. Zawarte zostały także implementacje poszczególnych metod.\\
Rozdział 4 został poświęcony metodom mutacji w~wersji klasycznej, jak~i~zmodyfikowanej. Zawarte zostały także opisy możliwości modyfikacji współczynnika mutacji.\\
W~rozdziale 5 zaprezentowano metody krzyżowania stosowane w~podstawowych implementacjach algorytmów oraz~te mające bezpośrednie zastosowanie w~przypadku problemów, których podstawową jednostką jest~osobnik będący permutacją pewnego zbioru. Zostały również opisane modyfikacje współczynnika krzyżowania.\\
W~rozdziale 6 przeprowadzono testy wszystkich opisanych metod oraz~modyfikacji. Na~podstawie uzyskanych wyników ustalona została wersja algorytmu uzyskująca najlepsze efekty. Następnie z~wykorzystaniem tej wersji algorytmu przeprowadzone zostały testy z~wykorzystaniem innych instancji danych wejściowych.\\
Rozdział 7 stanowi zakończenie i~podsumowanie pracy. Znajdują się~tam informacje dotyczące wersji algorytmu uznanych za~najskuteczniejsze. Dodatkowo można znaleźć informacje na~temat dalszych kierunków badań.
