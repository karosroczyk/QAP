\chapter{Wstęp teoretyczny}\label{cha:pierwszyDokument}

%---------------------------------------------------------------------------

\section{Opis algorytmu ewolucji różnicowej}\label{sec:strukturaDokumentu}


Algorytm ewolucji różnicowej jest~rodzajem heurystyki, a~więc rodzajem algorytmu, dla~którego~znalezione rozwiązanie nie~jest~objęte gwarancją bycia rozwiązaniem optymalnym. Metod heurystycznych używa się~w~sytuacji, gdy przestrzeń potencjalnych rozwiązań jest~na~tyle duża, iż~nie jest~możliwe w~dostatecznie krótkim czasie przebadanie wszystkich potencjalnych rozwiązań w~celu znalezienia tego najlepszego. Algorytmy ewolucyjne, których to~pewnego rodzaju odmianą są~algorytmy ewolucji różnicowej \cite{diff}, są~jednym z~narzędzi służących do~ukierunkowania procesu poszukiwań poprzez~zastosowanie technik probabilistycznych, a~co za~tym idzie służących do~usprawnienia rozwiązywania wspomnianej wyżej sytuacji. W~większości przypadków, przy odpowiednim doborze parametrów rozwiązanie będące wynikiem działania algorytmu ewolucyjnego jest~zarazem rozwiązaniem optymalnym, nie~mniej jednak zależność ta nie~występuje we wszystkich przypadkach.

Swoim działaniem algorytmy ewolucyjne przypominają zjawisko ewolucji biologicznej, a~kolejne etapy algorytmu stanowią operacje będące odpowiednikami zjawisk typowych dla~świata przyrody, a~więc zjawiska reprodukcji, mutacji oraz~krzyżowania. Inspiracja zasadą doboru naturalnego sprowadza się~do~ kierowania się~ideą, iż~tylko najlepiej przystosowane osobniki mają szanse na~przeżycie i~to one powinny zapoczątkowywać istnienie nowych co raz to~lepszych osobników potomnych, które~to z~kolei tworzą nowe mocniejsze populacje. Dzięki temu algorytm ewolucyjny tworzy stopniowo coraz lepsze rozwiązania i~może służyć on do~rozwiązywania problemów optymalizacyjnych. Więcej informacji na~ten temat można znaleźć w~\cite{zeszyty}.

Obecnie w~skład algorytmów ewolucyjnych wchodzą między innymi algorytmy genetyczne, programowanie genetyczne czy~też programowanie ewolucyjne. Jednak jeszcze w~latach sześćdziesiątych istniały i~rozwijały się~one niezależnie od~siebie jako oddzielne wersje wcielające w~życie darwinowską zasadę doboru naturalnego. Dopiero od~początku lat 20 XX wieku strategie ewolucyjne, algorytm genetyczny, programowanie ewolucyjne i~programowanie genetyczne zaczęły być~traktowane jako różne formy tej samej techniki określane wspólnym mianem algorytmów ewolucyjnych \cite{doktorat}.  Algorytm ewolucji różnicowej został zaproponowany w~1995 roku przez~Kene'a Price. Główna cecha odróżniająca algorytm ewolucji różnicowej od~klasycznego algorytmu ewolucyjnego dotyczy dokonywania operacji odejmowania od~siebie wektorów w~operacji mutacji, a~więc tworzenia wektorów różnicowych.

\subsection{Optymalizacja lokalna i globalna}

Zgodnie z~\cite{michal} problem optymalizacji może zostać w~ogólności przedstawiony jako problem mający na~celu znalezienie takiej wartości x w~danym zbiorze X, dla~której~funkcja f(x) będąca funkcją celu, a~więc funkcją mierzącą cel, który~ma zostać osiągnięty, przyjmuje najkorzystniejszą wartość. Wartość najkorzystniejsza może być~interpretowana w~zależności od~intencji jako minimum bądź~też maksimum danej funkcji f(x). Minima i~maksima są~zbiorowo nazywane ekstremami odpowiednio globalnymi bądź~lokalnymi.\\

Niech~R oznacza zbiór wszystkich rozwiązań obranego problemu. Punkt x* w~przestrzeni R jest~minimum globalnym, jeśli dla~każdego x należącego do~R zachodzi zależność taka, że:


\begin{equation}
 f(x*) \le f(x)
\end{equation}
Co można zapisać :
\begin{equation}
x = x*   \Leftrightarrow   \forall x \in R,    f(x*)\le f(x)
\end{equation}
W~takiej sytuacji x* jest~optimum globalnym określającym najmniejszą wartość w~całej dziedzinie. Nie~mniej jednak łatwiejszym do~znalezienia, rozwiązania i~częściej formułowanym problemem jest~zadanie określające minimum lokalne. Problem sprowadza się~do~znalezienia takiego $x_{1} $ dla~którego~
\begin{equation}
f(x_{1}) \le f(x_{1} + t), t >0
\end{equation}
gdzie $x_{1} + t $ są~punktami otaczającymi $x_{1}$.
Można to~zdefiniować w~następujący sposób:
\begin{equation}
x = x_{1} \Leftrightarrow f(x_{1}) \le f(x_{1} + t), t \ne 0
\end{equation}
Na~rysunku \ref{minmax} zaznaczone są~ekstrema funkcji spośród których możemy wyróżnić ekstrema lokalne oraz~globalne. Odpowiednio punkty:\\

\begin{itemize}
\item A, C, E są maksimami lokalnymi,
\item G jest maksimum globalnym,
\item B, F są minimami lokalnymi,
\item D jest minimum globalnym
\end{itemize}

\begin{figure}[h!]
\begin{center}
		\includegraphics[scale=0.6]{../../../../Screeny/mixmax.png}
		\caption{Przebieg funkcji wraz~z~zaznaczonymi ekstremami.}
		\label{minmax}		
\end{center}	
\end{figure}

Większość metod optymalizacyjnych ma zasięg lokalny, co znaczenie utrudnia poszukiwanie rozwiązania problemu, jako że~algorytmy kończą swoje działanie w~chwili, gdy wartość funkcji celu osiągnie wartość ekstremum lokalnego. Powodem tych problemów jest~zależność metod optymalizacyjnych od~pochodnych funkcji, które~to z~kolei nie~są~wystarczająco odporne na~nieciągłości czy~też rozległe wielomodalności.
\par

\subsection{Podstawowe pojęcia}
W~terminologii algorytmów genetycznych używa się~następujących pojęć zapożyczonych z~genetyki naturalnej \cite{michal}, \cite{maszynowe_sel}:\\

\textbf{Osobnik} - podstawowa jednostka charakteryzująca się pewnym przystosowaniem do środowiska w którym żyje czego miarą jest wartość funkcji celu obliczona dla tego osobnika. Dany osobnik, reprezentowany poprzez wektor liczb rzeczywistych, jest potencjalnym rozwiązaniem całego problemu optymalizacji.

\textbf{Populacja} - zbiór osobników ulegający zmianie w kolejnych iteracjach algorytmu jako, że słabsze osobniki są zastępowane przez nowe, lepiej przystosowane.

\textbf{Genotyp} - zbiór informacji przypisany do każdego osobnika indywidualnie, opisujący proponowane rozwiązanie problemu optymalizacyjnego.

\textbf{Fenotyp} - zbiór cech osobnika podlegających ocenie funkcji przystosowania.

\textbf{Chromosom} - jest to uporządkowany ciąg genów, który inaczej nazywany jest łańcuchem. W chromosomie zakodowany jest fenotyp i ewentualnie pewne informacje pomocnicze dla algorytmu genetycznego.

\textbf{Gen} - jest to pojedynczy element chromosomu.

\textbf{Grupa rozrodcza} - zbiór osobników służących jako podstawa do utworzenia osobnika potomnego.

\textbf{Funkcja celu} - funkcja pozwalająca określić miarę przystosowania i jakość konkretnego osobnika z punktu widzenia rozwiązywanego problemu. Jako argument przyjmuje ona wektor będący odpowiednikiem osobnika.

\textbf{Funkcja dopasowania} - jest to funkcja stanowiąca modyfikacje funkcji celu tak aby zachować zależność iż im większa wartość funkcji dopasowania dla danego osobnika tym osobnik jest lepszy.

\textbf{Selekcja} - technika mająca na celu wybór osobników, które przedostaną się do grupy rozrodczej tworzącej potomstwo.

\textbf{Mutacja} - technika mająca na celu utworzenie nowego osobnika na podstawie osobników znajdujących się w grupie rozrodczej .

\textbf{Krzyżowanie} - technika mająca na celu utworzenie nowego osobnika na podstawie osobnika wynikowego mutacji oraz osobnika z aktualnej populacji.

\textbf{Operacje genetyczne} - zbiór operacji takich jak selekcja, mutacja, krzyżowanie.

\subsection{Pseudokod}
\par
Większość algorytmów optymalizacyjnych stosuje procedurę postępowania, w~której to~z~każdym krokiem działania algorytmu coraz bardziej zbliżamy się~do~rozwiązania optymalnego. Algorytmy tego typu zazwyczaj rozpoczynają poszukiwanie startując z~konkretnego punktu, a~następnie, po~uwzględnieniu dodatkowych informacji, przemieszczają się~dalej dochodząc po~wielu iteracjach do~rozwiązania końcowego.\\
\par
Podobnie jak~w~algorytmach ewolucyjnych początkiem działania algorytmu ewolucji różnicowej jest~utworzenie pierwszej populacji osobników będących pierwszymi potencjalnymi rozwiązaniami problemu. W~każdej iteracji osobniki są~poddawane ocenie zgodnie z~przyjętą odgórnie funkcją celu. W~kontekście algorytmu ewolucji różnicowej osobnik zdefiniowany jest~jako n-elementowy wektor liczb rzeczywistych będący permutacją n-elementowego zbioru:\\
\begin{equation}
 \overrightarrow{x_{i}}=\{x_{i,1},x_{i,2},...x_{i,n}\}, j = 1,2,...,n
\end{equation}
Natomiast populacje można określić jako:\\
\begin{equation}
X=\{\overrightarrow{x_{1}},\overrightarrow{x_{2}},...\overrightarrow{x_{i}}\}, i~= 1,2,...,n
\end{equation}
Następnie przeprowadzana jest~reprodukcja, w~wyniku której~otrzymujemy grupę rozrodczą. Grupa rozrodcza stanowi podstawę do~przeprowadzenia operacji genetycznych, jakimi są~mutacja oraz~krzyżowanie. Operacja mutacji dokonuje przekształceń genetycznych na~danej grupie rozrodczej, w~wyniku których powstaje nowy osobnik. Poddawany jest~on operacji krzyżowania. Dopiero wynik tych dwóch operacji daje rezultat w~postaci osobnika potomnego, którego~współczynnik przystosowania zostaje oceniony. W~sytuacji gdy będzie on większy od~współczynnika przystosowania osobnika z~obecnej populacji, nastąpi ich podmiana, a~więc osobnik potomny zajmie w~populacji miejsce poprzednika. Warunkiem zakończenia algorytmu może być~na~przykład brak zmienności wyniku w~kolejnych $n$ iteracjach bądź~też osiągnięcie zadowalającego poziomu przystosowania.\\
\par
Opisane zależności można w~ogólności zaprezentować w~postaci pseudokodu algorytmu przedstawionego poniżej.

\begin{flushleft}
\scriptsize
\hspace{5cm}Przyjęcie określonych wartości $n,F,C_{r}$\\
\hspace{5cm}Inicjalizacja pierwszej populacji $X \leftarrow \{x_{1}, x_{2},...,x_{n}\}$\\
\hspace{5cm}while ( !warunek stopu) do\\
\hspace{5.5cm}	for i $\in \{1,2,...,n\}$ do\\
\hspace{6cm} \#reprodukcja\\
\hspace{6,5cm} $x_{i,1} \leftarrow$ reprodukcja $(i,X)$\\
\hspace{6,5cm} $x_{i,2} \leftarrow$ reprodukcja $(i,X)$\\
\hspace{6,5cm} $x_{i,3} \leftarrow$ reprodukcja $(i,X)$\\
\hspace{6cm} \#mutacja różnicowa\\
\hspace{6.5cm} $u_{i} \leftarrow$ mutacja różnicowa $(x_{i,1},x_{i,2},x_{i,3},F)$\\
\hspace{6cm} \#krzyżowanie\\
\hspace{6.5cm} $o_{i} \leftarrow$ krzyżowanie $(x_{i},u_{i},C_{r})$\\
\hspace{6cm} \#funkcja oceny\\
\hspace{6.5cm} if f($o_{i}$) $\leq$ f($(x_{i}$) then\\
\hspace{7cm} $x_{i} \leftarrow o_{i}$\\
\hspace{6.5cm} end if \\
\hspace{5.5cm} end for\\
\hspace{5cm}end while \\
\hspace{5cm}return $x_{minf(x_{i})}$\\
\end{flushleft}

Na~schemacie zauważalna staje się~odmienność algorytmu ewolucji różnicowej od~klasycznych algorytmów ewolucyjnych głównie w~kontekście sposobu dokonywania operacji mutacji.
Dobór metod reprodukcji, mutacji czy~też krzyżowania nie~może zostać jednoznacznie określony i~być uniwersalny dla~wszystkich problemów, które~ma on rozwiązywać. Poprawność i~jakość działania metod jest~ściśle zależna od~charakteru rozwiązywanego problemu. Z~tego również względu w~niektórych przypadkach konieczne jest~dokonanie zmian w~klasycznym algorytmie ewolucyjnym, w~wyniku czego otrzymujemy zmodyfikowany algorytm ewolucji różnicowej.

%---------------------------------------------------------------------------

\section{Opis kwadratowego zagadnienia przydziału}\label{sec:strukturaDokumentu}

Kwadratowe zagadnienie przydziału jest~jednym z~fundamentalnych problemów optymalizacyjnych zajmujących się~problemami rozlokowania obiektów. Problem należy do~klasy problemów NP-trudnych co oznacza, że~nie ma znanego i~jednoznacznie określonego algorytmu rozwiązującego ten typ problemu w~czasie wielomianowym. Sprawdzenie wszystkich opcji w~celu znalezienia rozwiązania najlepszego nawet małego rozmiaru problemu okazuje się~problematyczne i~długotrwałe.\\
\par
\textbf{Model matematyczny kwadratowego zagadnienia przydziału}\\
\par
Dany jest~zbiór $n$ budynków oraz~zbiór $n$ lokalizacji. Pomiędzy każdymi dwoma lokalizacjami znana jest~odległość oraz~wartość przepływu. Wartość przepływu można interpretować jako koszt transportu dóbr pomiędzy dwiema lokalizacjami. Celem zadania jest~takie przypisanie budynków do~danych lokalizacji, by~zminimalizować sumę odległości pomiędzy każdymi dwoma budynkami pomnożoną przez~odpowiadającą danej trasie wartość przepływu.


Formalna definicja kwadratowego zagadnienia przydziału jest~następująca \cite{chmiel} :

Dane są~dwie rzeczywiste macierze F = ($f_{ij}$) oraz~D = ($d_{ij}$) o~wymiarze $ n \times n $ każda, gdzie odpowiednio:\\
$f_{ij}$ - macierz przepływu pomiędzy punktami i~oraz j,\\
$d_{ij}$ - macierz odległości, określająca odległość pomiędzy punktami i~oraz j
Minimalizacja całkowitego kosztu działania systemu opisana jest~następującą zależnością :
\begin{equation}
\label {mathmodel}
\sum_{i=1}^{n}\sum_{j=1}^{n}f_{ij} \cdot d_{\pi(i)\pi(j)} \longrightarrow min
\end{equation}
Gdzie :\\
$\pi = \{\pi(1), ... , \pi(n)\}$ - permutacja zbioru N = \{1, ... , N\}\\
Należy zaznaczyć również fakt, iż~jeden budynek może zostać przypisany do~jednej lokalizacji oraz~do~jednej lokalizacji może zostać  przypisany tylko~jeden budynek. Poniżej zamieszczono rysunek obrazujący kwadratowe zagadnienie przydziału.

\begin{figure}[h!]
\begin{center}
		\includegraphics[scale=0.5]{../../../../Screeny/qap.png}
		\caption{Przedstawienie kwadratowego zagadnienia przydziału \cite{qap_rys}.}
		\label{qap}		
\end{center}	
\end{figure}

Opisana wzorem \ref{mathmodel} zależność stanowi model matematyczny i~funkcję celu będącą podstawą rozważań w~dalszych rozdziałach. Permutacja będąca wynikiem działania algorytmu ewolucji różnicowej interpretowana jest~jako pewien sposób rozlokowania $n$ budynków w~$n$ lokalizacjach. Poszczególne wartości genów odpowiadają konkretnym przypisanym do~nich na~stałe budynkom. Natomiast pozycje o~danych indeksach w~wektorze wynikowym interpretowane są~jako poszczególne lokalizacje.