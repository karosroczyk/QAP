\chapter{Fragmenty implementacji}\label{cha:pierwszyDokument}

\section*{Metody reprodukcji}\label{reporeprodukcji}
\subsection*{Klasyczna reprodukcja : losowy wybór}

\begin{lstlisting}[
 basicstyle=\scriptsize,]
        public List<int> RandomWitoutRestrictions()
        {
            List<int> WhichIndividuals = new List<int>();
            int number = 3;
            var rand = new System.Random();

            while (number != 0)
            {
                WhichIndividuals.Add(Kit.GiveRandomNumber(WhichIndividuals, Population.Count - 1, rand));
                number--;
            }
            return WhichIndividuals;
        }
\end{lstlisting}

\subsection*{Metoda rankingowa}

\begin{lstlisting}[
 basicstyle=\scriptsize,]
        public List<int> RankingMethond()
        {
            List<int> WhichIndividuals = new List<int>();
            List<int> WhichIndividualsIndex = new List<int>();
            List<int> WhichIndividualBeforeSelection = new List<int>();
            List<int> ResultsOfObjectiveFunction = inputdata.ObjectiveFunctionVector(Population);

            var rand = new System.Random();
            int size_tmp = size;
            int number = 3;

            while (size_tmp != 0)
            {
                var index = ResultsOfObjectiveFunction.IndexOf(ResultsOfObjectiveFunction.Min());
                ResultsOfObjectiveFunction[index] = Int32.MaxValue;

                int numberOfIter = size_tmp;
                while (numberOfIter != 0)
                {
                    WhichIndividualBeforeSelection.Add(index);
                    numberOfIter--;
                }
                size_tmp--;
            }

            while (number != 0)
            {
                var tmp = Kit.GiveRandomNumber(WhichIndividualsIndex, WhichIndividualBeforeSelection.Count - 1, rand);
                for (int i~= 0; i~< WhichIndividualBeforeSelection.Count; i++) // to~w~celu uniknięcia powtórzeń w~wektorach
                {
                    if (WhichIndividualBeforeSelection[i] == WhichIndividualBeforeSelection[tmp])
                        WhichIndividualsIndex.Add(i);
                }

                WhichIndividuals.Add(WhichIndividualBeforeSelection[tmp]);
                number--;
            }
            return WhichIndividuals;
        }
\end{lstlisting}

\subsection*{Metoda ruletki}

\begin{lstlisting}[
 basicstyle=\scriptsize,]
        public List<int> RouletteMethond()
        {
            List<int> WhichIndividuals = new List<int>();
            List<int> ResultsOfObjectiveFunction = inputdata.ObjectiveFunctionVector(Population);
            List<Tuple<int, int>> tuple = new List<Tuple<int, int>>();
            int range_begin = 0;
            int range_end = 0;
            var rand = new System.Random();
            int number = 3;

            foreach (var elem in ResultsOfObjectiveFunction)
            {
                range_end += elem;
                tuple.Add(new Tuple<int, int>(range_begin, range_end));
                range_begin = elem + 1;
            }

            while (number != 0)
            {
                var tmp = Kit.GiveRandomNumber(WhichIndividuals, ResultsOfObjectiveFunction.Sum(), rand);
                for (int i~= 0; i~< tuple.Count; i++)
                {
                    if (tuple[i].Item2 > tmp)
                    {
                        WhichIndividuals.Add(i);
                        number--;
                        break;
                    }
                }
            }
            return WhichIndividuals;
        }
\end{lstlisting}

\subsection*{Metoda turniejowa}

\begin{lstlisting}[
 basicstyle=\scriptsize,]
 public List<int> TournamentMethond()
        {
            List<int> WhichIndividuals = new List<int>();
            List<int> ResultsOfObjectiveFunction = inputdata.ObjectiveFunctionVector(Population);
            List<int> ResultsOfObjectiveFunction_const = new List<int>();
            ResultsOfObjectiveFunction_const.AddRange(ResultsOfObjectiveFunction);
            List<int> TournamentTable = new List<int>();
            int number = 3;

            while (number != 0)
            {
                int sizeOfTournament_const = ResultsOfObjectiveFunction.Count;
                while (sizeOfTournament_const > 1)
                {
                    for (int i~= 0; i~< sizeOfTournament_const; i++)
                    {
                        TournamentTable.Add(Math.Min(ResultsOfObjectiveFunction[i], ResultsOfObjectiveFunction[i + 1]));
                        i++;

                        if ((sizeOfTournament_const/2) == 1 && i~== sizeOfTournament_const - 2)
                        {
                            TournamentTable.Add(ResultsOfObjectiveFunction.Last());
                            i++;
                        }
                    }

                    if (TournamentTable.Count == 1)
                        break;
                    sizeOfTournament_const = TournamentTable.Count;
                    ResultsOfObjectiveFunction.RemoveAll(elem => ResultsOfObjectiveFunction.Contains(elem));
                    ResultsOfObjectiveFunction.AddRange(TournamentTable);
                    TournamentTable.RemoveAll(elem => TournamentTable.Contains(elem));
                }

                var index = ResultsOfObjectiveFunction_const.FindIndex(value => value == TournamentTable[0]);
                WhichIndividuals.Add(index);
                ResultsOfObjectiveFunction_const[index] = Int32.MaxValue;
                ResultsOfObjectiveFunction.RemoveAll(elem => ResultsOfObjectiveFunction.Contains(elem));
                ResultsOfObjectiveFunction.AddRange(ResultsOfObjectiveFunction_const);
                TournamentTable.RemoveAt(0);
                number--;
            }
            return WhichIndividuals;
        }
\end{lstlisting}

\subsection*{Metoda elitarna}

\begin{lstlisting}[
 basicstyle=\scriptsize,]
        public List<int> ElitistMethond()
        {
            List<int> WhichIndividuals = new List<int>();
            List<int> ResultsOfObjectiveFunction = inputdata.ObjectiveFunctionVector(Population);
            int number = 3;

            while (number != 0)
            {
                int index = ResultsOfObjectiveFunction.FindIndex(min => min == ResultsOfObjectiveFunction.Min());
                WhichIndividuals.Add(index);
                ResultsOfObjectiveFunction[index] = Int32.MaxValue;
                number--;
            }

            return WhichIndividuals;
        }
\end{lstlisting}

\section*{Metody mutacji}\label{repomutacji}

\subsection*{Klasyczna mutacja: DE/rand/1}

\begin{lstlisting}[
 basicstyle=\scriptsize,]
        public List<int> ToMutate()
        {
            List<int> MutatedIndividual = new List<int>(new int[Population.Count]);
            List<double> MutatedIndividual_tmp = new List<double>();
            List<int> Randoms = RandomWitoutRestrictions();
            double DefaultValueForMut = 0.8;

            for (int j = 0; j < Population.Count; j++)
            {
                double value = Population[Randoms[0]][j] + DefaultValueForMut * (Population[Randoms[1]][j] - Population[Randoms[2]][j]);
                MutatedIndividual_tmp.Add(value);
            }

            Normalize(MutatedIndividual_tmp, MutatedIndividual);
            return MutatedIndividual;
        }
\end{lstlisting}

\subsection*{Strategia II: DE/best/1}

\begin{lstlisting}[
 basicstyle=\scriptsize,]
        public List<int> ToMutate2()
        {
            List<int> MutatedIndividual = new List<int>(new int[Population.Count]);
            List<double> MutatedIndividual_tmp = new List<double>();

            List<int> Best = ElitistMethond(1);
            List<int> Difference = RandomWitoutRestrictions(2);

            double DefaultValueForMut = 0.8;

            for (int j = 0; j < Population.Count; j++)
            {
                double value = Population[Best[0]][j] + DefaultValueForMut * (Population[Difference[0]][j] - Population[Difference[1]][j]);
                MutatedIndividual_tmp.Add(value);
            }

            Normalize(MutatedIndividual_tmp, MutatedIndividual);
            return MutatedIndividual;
        }
\end{lstlisting}

\subsection*{Strategia III: DE/rand/$n_{v}$}

\begin{lstlisting}[
 basicstyle=\scriptsize,]
public List<int> ToMutate3(int number_of_vectors)
        {
            List<int> MutatedIndividual = new List<int>(new int[Population.Count]);
            List<double> MutatedIndividual_tmp = new List<double>();
            int n = 2 * number_of_vectors + 1;

            if (n > Population.Count)
            {
                number_of_vectors = (Population.Count - 1) / 2;
            }

            List<int> Randoms = RandomWitoutRestrictions(3);
            double DefaultValueForMut = 0.8;
            for (int j = 0; j < Population.Count; j++)
            {
                int difference = 0;
                int number = number_of_vectors;
                while (number != 0)
                {
                    difference += (Population[Randoms[1]][j] - Population[Randoms[2]][j]);
                    number--;
                }

                double value = Population[Randoms[0]][j] + DefaultValueForMut * difference;
                MutatedIndividual_tmp.Add(value);
            }
            Normalize(MutatedIndividual_tmp, MutatedIndividual);
            return MutatedIndividual;
        }
\end{lstlisting}

\subsection*{Strategia IV: DE/current to best/$n_{v} +1$}

\begin{lstlisting}[
 basicstyle=\scriptsize,]
public List<int> ToMutate4(int number_of_vectors, int iter)
        {
            List<int> MutatedIndividual = new List<int>(new int[Population.Count]);
            List<double> MutatedIndividual_tmp = new List<double>();
            int n = 2 * number_of_vectors + 1;

            if (n > Population.Count)
            {
                number_of_vectors = (Population.Count - 1) / 2;
            }

            List<int> Best = ElitistMethond(1);
            List<int> Randoms = RandomWitoutRestrictions(3);
            double DefaultValueForMut = 0.8;
            for (int j = 0; j < Population.Count; j++)
            {
                int difference = 0;
                int number = number_of_vectors;
                while (number != 0)
                {
                    difference += (Population[Randoms[1]][j] - Population[Randoms[2]][j]);
                    number--;
                }

                double value = Population[iter][j] + DefaultValueForMut * (Population[Best[0]][j] - Population[Randoms[0]][j]) + DefaultValueForMut * difference;
                MutatedIndividual_tmp.Add(value);
            }
            Normalize(MutatedIndividual_tmp, MutatedIndividual);
            return MutatedIndividual;
        }
\end{lstlisting}

\section*{Metody krzyżowania}\label{repocross}

\subsection*{Klasyczne krzyżowanie : krzyżowanie dwumianowe}

\begin{lstlisting}[
 basicstyle=\scriptsize,]
	public List<int> BinomialCrossver(List<int> ParentIndividual, List<int> MutatedIndividual)
        {
            List<int> CrossedIndividual = new List<int>();
            Random random = new Random();
            double Cr = random.NextDouble();
            Random random2 = new Random();

            int size = ParentIndividual.Count;
            for (int i~= 0; i~< size; i++){
                double Cr_tmp = random.NextDouble();

                if (Cr_tmp < Cr){
                    if (CrossedIndividual.Contains(MutatedIndividual[i])){
                        var ind = GiveRandomNumber(CrossedIndividual, MutatedIndividual.Count, random2);
                        CrossedIndividual.Add(ind);
                    }
                    else
                        CrossedIndividual.Add(MutatedIndividual[i]);
                }
                else //Parent{
                    if (CrossedIndividual.Contains(ParentIndividual[i])){
                        var ind = GiveRandomNumber(CrossedIndividual, ParentIndividual.Count, random2);
                        CrossedIndividual.Add(ind);
                    }
                    else
                        CrossedIndividual.Add(ParentIndividual[i]);
                }
            }
            return CrossedIndividual;
        }
\end{lstlisting}

\subsection*{Krzyżowanie OX}

\begin{lstlisting}[
 basicstyle=\scriptsize,]
	public List<int> OXCrossver(List<int> ParentIndividual, List<int> MutatedIndividual)
        {
            List<int> CrossedIndividual = new List<int>();
            for (int i~= 0; i~< ParentIndividual.Count; i++){
                CrossedIndividual.Add(0);
            }

            Random rnd = new Random();
            int offset = rnd.Next(0, ParentIndividual.Count);
            int length = rnd.Next(1, ParentIndividual.Count - offset);
            int iter = 0;
            while (iter != length){
                CrossedIndividual[offset + iter] = MutatedIndividual[offset + iter];
                iter++;
            }

            for (int i~= 0; i~< ParentIndividual.Count; i++){
                if (CrossedIndividual[i] == 0){
                    var i_tmp = 0;
                    while(CrossedIndividual.Contains(ParentIndividual[i_tmp])){
                        i_tmp++;
                    }
                    CrossedIndividual[i] = ParentIndividual[i_tmp];
                }
            }
            return CrossedIndividual;
        }
\end{lstlisting}

\subsection*{Krzyżowanie CX}

\begin{lstlisting}[
 basicstyle=\scriptsize,]
        public List<int> CXCrossver(List<int> ParentIndividual, List<int> MutatedIndividual)
        {
            List<int> CrossedIndividual = new List<int>();
            for (int i~= 0; i~< ParentIndividual.Count; i++){
                CrossedIndividual.Add(0);
            }

            CrossedIndividual[0] = MutatedIndividual[0];
            var index = 0;

            while (ParentIndividual[index] != MutatedIndividual[0]){
                index = MutatedIndividual.FindIndex(x => x == ParentIndividual[index]);
                CrossedIndividual[index] = MutatedIndividual[index];
            }

            for(int i~= 0; i~< ParentIndividual.Count; i++){
                if (CrossedIndividual[i] == 0){
                    var i_tmp = 0;
                    while (CrossedIndividual.Contains(ParentIndividual[i_tmp])){
                        i_tmp++;
                    }
                    CrossedIndividual[i] = ParentIndividual[i_tmp];
                }
            }
            return CrossedIndividual;
        }
\end{lstlisting}

\subsection*{Krzyżowanie PMX}
\begin{lstlisting}[
 basicstyle=\scriptsize,]
        public List<int> PMXCrossver(List<int> ParentIndividual, List<int> MutatedIndividual){
            List<int> MutatedIndividualSubstringCopy = new List<int>();
            List<int> ParentIndividualSubstringCopy = new List<int>();
            List<int> MutatedIndividualSubstring = new List<int>();
            List<int> ParentIndividualSubstring = new List<int>();
            Random rnd = new Random();
            int offset = rnd.Next(0, ParentIndividual.Count);
            int length = rnd.Next(1, ParentIndividual.Count - offset);
            int iter = 0;

            while (iter != length){
                MutatedIndividualSubstring.Add(MutatedIndividual[offset + iter]);
                ParentIndividualSubstring.Add(ParentIndividual[offset + iter]);
                iter++;}

            for (int i~= offset; i~< offset + MutatedIndividualSubstring.Count; i++){
                MutatedIndividual[i] = 0;
                ParentIndividual[i] = 0;}

            for (int i~= 0; i~< MutatedIndividualSubstring.Count; i++){
                MutatedIndividualSubstringCopy.Add(MutatedIndividualSubstring[i]);
                ParentIndividualSubstringCopy.Add(ParentIndividualSubstring[i]);}

            for (int i~= 0; i~< MutatedIndividualSubstring.Count; i++){
                if (ParentIndividualSubstring.Contains(MutatedIndividualSubstring[i])){
                    var index = ParentIndividualSubstring.FindIndex(x => x == MutatedIndividualSubstring[i]);
                    MutatedIndividualSubstring[i] = MutatedIndividualSubstring[index];
                    MutatedIndividualSubstring.RemoveAt(index);
                    ParentIndividualSubstring.RemoveAt(index);
                    i~= -1;}
            }
            for (int i~= 0; i~< MutatedIndividual.Count; i++){
                if (!MutatedIndividual.Contains(ParentIndividual[i])){
                    int  index_2 = MutatedIndividualSubstring.FindIndex(x => x == ParentIndividual[i]);
                    var value = ParentIndividualSubstring[index_2];
                    var index = MutatedIndividual.FindIndex(x => x == value);
                    MutatedIndividual[index] = ParentIndividual[i];
                    ParentIndividual[i] = value;}
            }
            var i_tmp = 0;
            var i_tmp2 = 0;

            for (int i~= 0; i~< MutatedIndividual.Count; i++){
                if (MutatedIndividual[i] == 0){
                    MutatedIndividual[i] = ParentIndividualSubstringCopy[i_tmp];
                    i_tmp ++;}
                if (ParentIndividual[i] == 0){
                    ParentIndividual[i] = MutatedIndividualSubstringCopy[i_tmp2];
                    i_tmp2 ++;}
            }
            return MutatedIndividual;
        }
\end{lstlisting}
