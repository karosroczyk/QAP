\chapter{Podsumowanie} \label{cha:podsumowanie}

W wyniku prac powstał interfejs do projektu, króry stanowi tylko jedną część całości.
Uzyskana dokładność jest wystarczająca do testowania całego projektu,
lecz może się okazać za mała do profesjonalnych zastosowań dla bronioznawców.
Do otrzymania idealnej precyzji konieczne będzie użycie dokładniejszych kontrolerów, jak np. Razer Hydra\footnote{
	Kontroler stworzony przez Sixense Entertainment, działający w oparciu o słabe pole magnetyczne. Według twórców osiąga dokładność pozycjonowania do 1mm i orientacji do 1 stopnia.
}
lub nawet Stem System -- rozwiązania podobnego do Razer Hydra, 
jednak pozwalającego śledzić całe ciało z użyciem pięciu czujników.
Użycie drugiego rozwiązania wyeliminowałoby także potrzebę korzystania z Kinecta i pozwoliłoby uniknąć jego ograniczeń.

Ponadto problemy z Oculusem dotyczą tylko kart graficznych w laptopach.
Z przeprowadzonych testów wynika, 
że na komputerach stacjonarnych będzie możliwe wykorzystanie wirtualnej rzeczywistości w pełni.

\section{Kierunki rozwoju} \label{sec:rozwoj}
%Połączenie projektów
Kolejnym krokiem będzie połączenie obu części projektu. 
W związku z tym pojawią się dodatkowe trudności do rozwiązania.
W walce gracza z wirtualnym szermierzem trzeba będzie uwzględnić "fizykę odwrotną", 
czyli reakcję szabli na uderzenie w inny przedmiot. 
W takim wypadku gracz będzie musiał wyobrazić sobie kolizję oraz opór z nią związany i zatrzymać ruch ręki 
lub nawet cofnąć ją do poprawnej pozycji.

%Kolizje, wyszczerbienia
Będzie konieczne dokładniejsze rozpatrywanie kolizji między samymi szablami.
Jeśli ostrze trafia w płaz\footnote{ płaz -- boczna część klingi} - ślizga się po nim. 
Natomiast jeśli ostrze zderza się z drugim ostrzem, 
działają dużo większe siły tarcia, 
krawędzie często sczepiają się. 
Poza tym w szablach powstają wyszczerbienia, 
co prowadzi do kolejnego poziomu szczegółowości obliczeń fizycznych.

Kolejnym pomysłem jest wykrywanie miejsca trafienia szablą. 
Do obliczenia obrażeń trzeba określić rodzaj pancerza w miejscu trafienia. 
Poza siłą i pancerzem należy uwzględnić szkody, jakie powoduje takie trafienie, 
wzorując się na tablicy Fairbairn\textquotesingle ea\footnote{
	Są w niej czasy zgonu lub stania się niezdolnym do walki po uszkodzeniu konkretnej części ciała.
}.

W planach rozwoju jest także dodanie innych rodzajów uzbrojenia. 
Symulowanie walki mieczem i tarczą byłoby problematyczne przy obecnej konfiguracji, 
ponieważ Kinect nie byłby w stanie określić pozycji ręki zasłoniętej przez tarczę. 
Dodanie drugiego kontrolera Move nie stanowi problemu.

%W końcu możliwe jest przekształcenie projektu w grę zręcznościową, odchodząc od koncepcji zwiększania realizmu przez integrację z prawdziwą bronią.\\
